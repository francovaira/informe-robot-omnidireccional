\newpage
\chapter{Desarrollo}

Este proyecto ilustra la concepción y construcción de un robot móvil pequeño. Durante todo el proyecto se orientará el diseño para la coexistencia de múltiples robots en el espacio, aun así, en el presente trabajo abordaremos el desarrollo de un solo robot.  Se detallarán todas las etapas de desarrollo del mismo, partiendo de los componentes básicos hasta el producto terminado.

Para el desarrollo de este proyecto, se realizarán iteraciones donde en cada una se sigue un ciclo que incluye los requerimientos que se aborda, el desarrollo, las pruebas y los resultados obtenidos. Un enfoque iterativo permite ajustar y mejorar continuamente el proyecto, asegurando que se cumplan todos los requerimientos y objetivos. A continuación, se detalla cada una de las etapas de una iteración típica.

En primera instancia se identifican los requerimientos que se abordan en la iteración. Estos requerimientos pueden repetirse en iteraciones sucesivas en caso de necesitar un ajuste o replanteo. Con los requerimientos en mente, se procede a elaborar una solución detallada y se planifica su implementación.

Para verificar y validar el diseño durante el proceso de desarrollo, se llevan a cabo las pruebas del componente o funcionalidad desarrollada y se documentan, garantizando que funcione correctamente y cumpla con las expectativas. Los resultados obtenidos se analizan para determinar si se ha alcanzado el objetivo de la iteración.

Finalmente, se elabora una conclusión de la iteración, en la que se resumen los resultados obtenidos y las lecciones aprendidas. Esta conclusión ayuda a identificar cualquier mejora necesaria o puntos que requieran revisión en futuras iteraciones.

\newpage
\section{Iteración 0: Decisiones preliminares}
\subsection{Introducción}
En esta iteración vamos plantear la elección que realizamos para todos los componentes que componen el robot, los opciones que contemplamos para la implementación y el motivo de las elecciones.

\subsection{Requerimientos}

En esta iteración abordaremos los siguientes requerimientos funcionales:

\begin{center} 
    \begin{tabular} {
        | >{\centering\arraybackslash}m{1cm}
        | >{\centering\arraybackslash}m{13cm} |}
        \hline
            ID & Descripción \\
        \hline
            RF1 & El robot debe contar con un sistema de control para las 4 ruedas. \\
        \hline
            RF2 & El robot debe tener un sistema de locomoción omnidireccional. \\
        \hline
            RF3 & El robot debe poder medir la distancia recorrida. \\
        \hline
            RF4 & El robot debe poder realizar trayectorias en línea recta y curvas. \\
        \hline
            RF5 & El robot debe poder corregir su trayectoria mediante el uso de sensores. \\
        \hline
            RF6 & El robot debe recibir y enviar información mediante comunicaciones inalámbricas. \\
        \hline
    \end{tabular}
\end{center}
  
   Por otra parte, el requerimiento no funcional que abordaremos es:
  
\begin{center}
    \begin{tabular} {
        | >{\centering\arraybackslash}m{1cm}
        | >{\centering\arraybackslash}m{13cm} |}
        \hline
        ID & Descripción \\
        \hline
        RNF1 & Debería tener tiempos de respuesta aceptables para el buen funcionamiento del sistema de control. \\
        \hline
    \end{tabular}
\end{center}

\subsection{Desarrollo}

\subsection{Relevamiento del Proyecto Integrador Hermes III}

El robot Hermes III es la última versión de la familia de robots desarrollada dentro del laboratorio de arquitectura de computadoras. Este también incorpora sistemas del modelo anterior y plantea una mejora sobre ellos, en este caso sobre el sistema de control que comanda el robot móvil y la implementación de un sistema operativo robótico (ROS) el cual es un framework para Linux que está diseñado exclusivamente para el desarrollo de robots. \cite{micolini2022hermes}

El uso del sistema operativo ROS abre la posibilidad de usar software dedicado al desarrollo de robots móviles y también a la incorporación de componentes como ser el módulo de cámara Kinect desarrollado por Microsoft para la consola Xbox.

El conjunto ya mencionado da paso a la implementación e innovación más importante del proyecto, la localización y mapeo en simultáneo (SLAM), brindándole al robot la posibilidad de moverse sobre un superficie y crear un mapa de su entorno, provocando que en cada nueva iteración pueda mejorar su desplazamiento y localización en la superficie de movimiento.

El software se ejecuta sobre una placa NVIDIA Jetson TK1, la cual ofrece un gran poder de cómputo que es necesario para sostener y correr de forma efectiva tanto el sistema operativo como el algoritmo de SLAM. La desventaja es su alto precio, lo que aumenta el costo de construcción del robot y por lo tanto disminuye la posibilidad de pensar en un flota de estos equipos.

\subsubsection{Elección de microcontrolador}

Dentro de las características del modelo anterior de la familia Hermes, se destacaba la gran cantidad hardware por el cual estaba compuesto, y por ende, el gran costo para su replicación e implementación.

Para esta nueva edición se decidió reducir la cantidad de componentes que lo integran, en cuanto al procesamiento se refiere, y llevarlo a un modelo que se acerque mas a la Industria 4.0 donde los dispositivos se conectan a Internet y es allí donde mandan reportes y reciben instrucciones. Es por eso que se optó por la utilización del microcontrolador ESP-32 el cual proporciona un API bastante completa y por en ende un gran abanico para que el desarrollador despliegue todo su potencial. \cite{kolban2017kolban}

\begin{figure}[H]
   \centering
   \includegraphics[width=0.7\linewidth]{images/esp32-micro.jpg}
   \caption{Microcontrolador ESP32}
   \label{fig:microcontrolador}
\end{figure}

\subsubsection{Sistema Operativo} \mbox{} \vspace{10pt} \\
La decisión de utilizar FreeRTOS como sistema operativo para el desarrollo de este proyecto se fundamenta en varias consideraciones claves. Primero, FreeRTOS está integrado de manera nativa en el framework ESP-IDF de Espressif, lo que facilita enormemente el desarrollo y la integración de funcionalidades en tiempo real. Al utilizar la API de Espressif, se aprovechan componentes y servicios que ya están optimizados para trabajar con FreeRTOS, lo que se traduce en una mayor estabilidad y rendimiento en aplicaciones críticas.

FreeRTOS permite una gestión eficiente de tareas, permitiendo la concurrencia y la sincronización de procesos de manera robusta y controlada. Esto es esencial para aplicaciones que requieren una respuesta rápida a eventos externos, como la gestión de sensores, la comunicación inalámbrica y la realización de múltiples operaciones simultáneamente. La modularidad y escalabilidad que ofrece FreeRTOS permite desarrollar soluciones complejas sin incurrir en sobre costos de recursos, lo que es especialmente valioso en sistemas embebidas con limitaciones de hardware.

El uso de FreeRTOS está respaldado por un amplio ecosistema de documentación y soporte, facilitando el desarrollo, la depuración y el mantenimiento de aplicaciones en el microcontrolador ESP32. Esta sinergia entre el hardware y el sistema operativo garantiza una integración fluida y optimizada para las exigencias de aplicaciones modernas en sistemas embebidos y soluciones IoT, lo que se ajusta perfectamente al desarrollo de este proyecto.

\subsubsection{Elección de motores}

El sistema de locomoción es el responsable de la traslación del robot. Las configuraciones más comunes son las siguientes: diferencial, sincrónico, triciclo, ackerman y omnidireccional.

Como nosotros tomamos como base la estructura ya definida para el robot Hermes III, adoptamos el sistema mecánico de locomoción omnidireccional, el cual permite mayor libertad de movimiento que los sistemas de ruedas clásicos, Los robots que implementan este sistema pueden moverse en cualquier dirección sobre el plano y en cualquier momento sin la necesidad de hacer movimientos previos para modificar su trayectoria. Requiere ruedas que permitan movimientos en más de una dirección. Este sistema puede ser implementado con tres o cuatro ruedas.

Las ruedas omnidireccionales ruedan en el sentido de avance, pero, también se pueden desplazar lateralmente con gran facilidad como se observa en la siguiente figura:

\begin{figure}[H]
    \centering
    \includegraphics[width=0.25\linewidth]{images/omni-wheel-wikipedia.png}
    \caption{Rueda omnidireccional}
    \label{fig:rueda_omnidireccional}
\end{figure}

Los actuadores tienen por misión generar el movimiento de los elementos del robot según las órdenes dadas por la unidad de control. De manera general, los actuadores utilizados en robótica pueden emplear energía neumática, hidráulica o eléctrica. Las características de control, sencillez y precisión de los accionamientos han hecho que los actuadores eléctricos sean los más usados. Dentro de los actuadores eléctricos pueden distinguirse tres tipos diferentes: \\

\textbf{Motores de corriente continua (DC):}
\begin{itemize}
   \item Controlados por inducción
   \item Controlados por excitación
\end{itemize}

\textbf{Motores de corriente alterna (AC):}
\begin{itemize}
   \item Síncronos
   \item Asíncronos
\end{itemize}

Se opta por motores de corriente continua, como el que se muestra en la figura siguiente, pues el torque generado es proporcional a la diferencia de potencial aplicado a los terminales de alimentación y el sentido de giro depende de la polaridad, facilitando de este manera el control. El modelo de motor elegido se muestra en la imagen debajo, es un motorreductor de $12V\ @\ 100\ RPM$.

\begin{figure}[H]
    \centering
    \includegraphics[width=0.35\linewidth]{images/motorreductor.png}
    \caption{Motorreductor}
\end{figure}


Este tipo de motores pequeños y de bajo costo generalmente presentan como inconveniente que no publican sus curvas. Para resolver este inconveniente se realizaron distintos experimentos. Esencialmente se construyó un montaje que consta de una polea con un peso conocido, que no es más que un recipiente con agua como muestra la siguiente figura, la polea se fija al eje del motor para realizar los experimentos.

Se realizaron distintas corridas variando la tensión y corriente, se tomaron mediciones de velocidad y torque. Con estos datos se construyeron las curvas de rpm-torque y corriente-torque, como se muestra la siguiente tabla:

\begin{center} \begin{tabular}{|c|c|c|}
   \hline \rowcolor{test_header_color}
       Tensión & Velocidad & Torque \\
   \hline
       3V & 25 RPM & 0,3 Kg*cm\\
   \hline
       6V & 67 RPM & 1 Kg*cm\\
   \hline
       12V & 96 RPM & 2 Kg*cm\\
   \hline
\end{tabular} \end{center}

Con estos experimentos, también, se midió la respuesta del sistema a un escalón, detallado en la siguiente iteración.

\begin{figure}[H]
    \centering
    \includegraphics[width=0.25\linewidth]{images/medicion_rpm.jpg}
    \caption{Montaje para medir el torque}
    \label{fig:medicion_torque}
\end{figure}

Para la alimentación de los motores es necesario un circuito integrado especial que permite manipular de manera segura la corriente eléctrica de los motores y a su vez brinde la posibilidad de controlar la polaridad de los bornes donde se conectan los mismos para cambiar su sentido de giro.

Esta solución se ofrece en el mercado en el integrado comúnmente denominado "Driver Dual para Motores L298N", también conocido como "puente H", el cual posee a su vez un regulador de voltaje LM7805 para alimentar la parte lógica del integrado L298N.

El sentido de giro de un motor estará definido por dos pines cuyos valores establecerán la polaridad de los terminales de alimentación del motor respectivamente. Existen dos pares de pines por cada motor.

Para el control de estos integrados vamos a usar cuatro canales PWM del microcontrolador ESP-32, los cuales van a establecer el sentido de giro y fuerza del motor.

\subsubsection{Sensores rotativos}

Un sensor es un dispositivo eléctrico y/o mecánico capaz de convertir magnitudes físicas, como la luz, velocidad, aceleración, presión, temperatura, etc. En otra magnitud, normalmente eléctrica, que sea posible manipular y cuantificar.

Para nuestro caso en particular, usamos los sensores para medir la velocidad de los motores. Como el objetivo es construir un sistema de lazo cerrado que controle la velocidad de los motores con el fin de aproximar la trayectoria calculada.

Para medir la velocidad en revoluciones por minuto (RPM), se colocó, en el eje del motor, una rueda con patrón impreso el cual es detectado por un sensor infrarrojo (opto acoplador de ranura). El patrón impreso consta de líneas perforadas, ranuras, en la rueda con el fin de generar interrupciones. Se realizaron 24 ranuras en cada rueda para poder realizar el cálculo de la distancia recorrida.

Para realizar la medición de RPM se experimentó tres métodos diferentes:

\begin{itemize}
   \item Generando una interrupción con el flanco de señal obtenida del sensor infrarrojo. Las RPM se determinan en función de la cantidad de interrupciones contadas en el periodo de una base de tiempo implementada para este fin.
   \item Tomando el tiempo del sistema entre dos interrupciones consecutivas, y luego calcular las RPM.
   \item Midiendo la cantidad de pulsos generada por una base de tiempo externa entre dos interrupciones. Para lo cual se toma la lectura de un contador en la primera interrupción y se toma la lectura nuevamente en la interrupción siguiente. Con la diferencia entre las lecturas se calculan las RPM.
   \item Utilizando el contador de pulso incorporado en el microcontrolador ESP32, y para tener en cuenta las bajas revoluciones acumular los pulsos para luego calcular el promedio y obtener un resultado más preciso.
\end{itemize}

\begin{figure}[H]
  \centering
  \includegraphics[width=0.5\linewidth]{images/encoder.png}
  \caption{Modelo 3D del encoder rotativo y opto acoplador de ranura}
  \label{fig:encoder}
\end{figure}

Por último se debe destacar que es necesario filtrar el rebote de los flancos de subida y bajada en cada interrupción.

\subsubsection{Elección de lenguaje para interfaz} \mbox{} \vspace{10pt} \\
En cuanto a las elección del lenguaje se refiere, teníamos dos caminos para elegir, ya que nos encontrábamos limitados por el microcontrolador que soporta tanto C como microPython. Habiendo hecho un análisis y una lectura de las distintas APIs que el fabricante nos ofrece, optamos por C ya que es un lenguaje conocido por nosotros. Esto porque durante el transcurso de la carrera nos tocó desarrollar varios trabajos prácticos en este lenguaje, por lo que sentíamos que no necesitabamos adquirir muchos nuevos conocimientos, es un entorno cómodo para trabajar y la documentación es que se encuentra es amplia por lo que los problemas que podían llegar a surgir los íbamos a poder sortear con cierta facilidad.
En el escenario de seguimiento del robot nos encontramos con el desafió de realizar una interfaz amigable, entendible y que pueda desarrollarse de la forma mas rápida posible sin la necesidad de tener una curva de aprendizaje muy grande. Es por ello que optamos por Python para poder cumplir todos los requisitos ya mencionados. Ambos integrantes ya habíamos trabajado con este lenguaje, y ademas, últimamente, se lo encuentra seguido entre los lenguajes mas utilizados por toda la comunidad de la programación, esto por ser un lenguaje sumamente adaptable ante cualquier necesidad y contar con una variedad muy extensa de librerías. Nos pareció una buena elección que nos daba la opción de sortear cualquier obstáculo que surgiera en el camino.

\subsubsection{Comunicación inalámbrica}

Nuestro proyecto al estar pensado para cumplir con los estándares de la Industria 4.0, nos vimos en la necesidad de incorporar la comunicación inalámbrica como medio principal para poder conectar y establecer comunicación entre los distintos componentes que forman parte del proyecto. Así como muestra el diagrama de alto nivel, los componentes se comunican utilizando un router como medio para poder llegar uno hacia otro.

Los microcontroladores elegidos vienen incorporados con los integrados necesarios para hacer uso de esta tecnología y ademas cuentan con la posibilidad de incorporarles una antena, lo que mejora la calidad de la señal y amplia su rango de alcance por lo que los vuelve mas efectivos aun si es que se necesita enviar gran cantidad de información y asegurar que los paquetes van a llegar a destino.

Existe la posibilidad de usar protocolos propios de estos microcontroladores, como ser ESP-NOW, para establecer una comunicación mas segura entre los microcontroladores, pero, optamos por hacer uso del protocolo MQTT que es usado ampliamente en la industria para enviar información de sensores y mensajes cortos. Esto debido a la liviandad y eficiencia de sus mensajes, además, de no atarnos por completo a un protocolo privativo de la marca ESP y poder usar lo que hasta ahora viene siendo un estándar para los dispositivos IoT.

\begin{figure}[H]
   \centering
   \includegraphics[width=0.7\linewidth]{images/com_inalambrica.jpg}
   \caption{Protocolos usados comúnmente en entornos IoT}
   \label{fig:mqtt}
\end{figure}

\subsection{Pruebas y testing}

\begin{testtableformat}
    \hline \rowcolor{test_header_color}
        Test ID             & TC\_01\_00 \\
    \hline
        Tipo de test        & Test unitario \\
    \hline
        Objeto de prueba    & Determinar cuanto se desplazó el robot\\
    \hline
        Requerimiento       & RF3\\
    \hline
        Nombre              & Cálculo de la distancia recorrida \\
    \hline
        Descripción         & Se pretende calcular la distancia recorrida por el robot contando las cantidad de ranuras que se atraviesan utilizando el contador de pulsos del microcontrolador ESP32 \\
    \hline
        Precondición        & PRECOND\_A \\
    \hline
        Pasos del test      & \begin{enumerate}
                                \item Crear un programa que habilite el contador de pulsos incorporado en el microcontrolador ESP32
                                \item Compilar el proyecto
                                \item Grabar el programa en el microcontrolador
                                \item Contar la cantidad de ranuras que atraviesa el opto acoplador y mediante la formula $distancia_recorrida = 2.\pi.r\ /\ ranuras_del_encoder$ obtener la distancia recorrida por la rueda del robot
                            \end{enumerate} \\
    \hline
        Resultado esperado  & Obtener de forma precisa el desplazamiento realizado por la rueda del robot \\
    \hline
        Resultado obtenido  & Se pudo determinar cuanto es el desplazamiento de la rueda mediante el contador de pulsos pero en frecuencias baja pierde mucha precisión por lo que se decidió realizar varias lecturas de la medición y realizar el promedio de las mismas \\
    \hline
        Observaciones       & - \\
    \hline
 \end{testtableformat}

\begin{testtableformat}
   \hline \rowcolor{test_header_color}
       Test ID             & TC\_01\_01 \\
   \hline
       Tipo de test        & Test unitario \\
   \hline
       Objeto de prueba    & Ejecutar un programa simple en el microcontrolador ESP32 \\
   \hline
       Requerimiento       & RF6 \\
   \hline
       Nombre              & Programa en C que logra establecer conexión WiFi \\
   \hline
       Descripción         & Crear, compilar y grabar un programa simple que logre establecer una conexión inalambrica usando el protocolo WiFi \\
   \hline
       Precondición        & PRECOND\_A \\
   \hline
       Pasos del test      & \begin{enumerate}
                               \item Crear un proyecto en C para el microcontrolador ESP32 usando la librería provista por Espressif
                               \item Compilar el proyecto
                               \item Grabar el programa en el microcontrolador
                               \item Establecer una conexión inalambrica WiFi
                           \end{enumerate} \\
   \hline
       Resultado esperado  &  El microcontrolador debe establecer la conexión inalámbrica sin errores \\
   \hline
       Resultado obtenido  & El microcontrolador establece la conexión WiFi \\
   \hline
       Observaciones       & - \\
   \hline
\end{testtableformat}

\subsection{Resultados}
El resultado es satisfactorio debido a que pudimos recolectar gran cantidad de información de todos los componentes y armamos el principio de lo que será el desarrollo de todo el proyecto.

\subsection{Riesgos superados}
\begin{center}
    \begin{tabular} {
        | c| c |}
        \hline \rowcolor{test_header_color}
            ID & Riesgo \\
        \hline
            RI-04 & Modificación de los requerimientos del proyecto\\
        \hline
    \end{tabular}
\end{center}

\subsection{Conclusiones}
En base a los información recolectada y leída, vemos que planteamos una base por donde comenzar el proyecto, definiendo los sistemas y componentes que se van a usar para su implementación, por lo tanto nos sentimos cómodos con las decisiones tomadas y vemos con buenos ojos el comienzo del desarrollo.

\newpage
\section{Iteración 1: Prototipo del robot}

\subsection{Introducción}
En la iteración anterior obtuvimos como resultado los principios y bases fundamentales sobre las que desarrollaremos esta iteración para lograr un prototipo funcional.

\subsection{Requerimientos}
En esta iteración abordaremos los siguientes requerimientos funcionales:

\begin{center}
\begin{tabular}{
    | >{\centering\arraybackslash}m{1cm}
    | >{\centering\arraybackslash}m{13cm} |
}
\hline \rowcolor{test_header_color}
    ID & Descripción \\
\hline
    RF1 & El robot debe contar con un sistema de control para las 4 ruedas. \\ 
\hline
    RF2 & El robot debe tener un sistema de locomoción omnidireccional. \\ 
\hline
    RF3 & El robot debe poder medir la distancia recorrida. \\ 
\hline
    RF4 & El robot debe poder realizar trayectorias en línea recta y curvas. \\ 
\hline
    RF5 & El robot debe poder corregir su trayectoria mediante el uso de sensores. \\  
\hline
    RF6 & El robot debe recibir y enviar información mediante comunicaciones inalámbricas. \\ 
\hline
\end{tabular}
\end{center}

Por otra parte, los requerimientos no funcionales que trataremos son:

\begin{center}
\begin{tabular}{
    | >{\centering\arraybackslash}m{1cm}
    | >{\centering\arraybackslash}m{13cm} |
}
\hline \rowcolor{test_header_color}
    ID & Descripción \\
\hline
    RNF1 & Debería tener tiempos de respuesta aceptables para el buen funcionamiento del sistema de control. \\
\hline
    RNF2 & El software debería contar con pruebas unitarias y de integración. \\
\hline
    RNF4 & El código debería contar con documentación.\\
\hline
\end{tabular}
\end{center}

\subsection{Desarrollo}

\subsubsection{Sistema de control PID}
Un controlador PID (Proporcional-Integral-Derivativo) es un tipo de controlador utilizado en sistemas de control automático para mantener una variable de algún proceso lo más cercana posible a un valor deseado, conocido como "setpoint", a pesar de las perturbaciones encontradas. La acción proporcional responde al error actual, que es la diferencia entre el valor deseado y el valor medido, ajustando la salida del controlador proporcionalmente a este error. Por otro lado, la acción integral se enfoca en la acumulación de errores pasados para eliminar cualquier error que sea persistente, asegurando que con el tiempo la variable de proceso converja al setpoint establecido. Finalmente, la acción derivativa considera la tasa de cambio del error, anticipando y corrigiendo cualquier futura tendencia del error, sumando a la estabilidad del sistema.

El objetivo principal de un controlador PID es proporcionar una respuesta rápida y estable en diferentes aplicaciones. Además de ser utilizado en procesos industriales como plantas químicas y refinerías, también se lo utiliza en el control de velocidad de motores eléctricos, asegurando que el motor funcione a la velocidad deseada sin fluctuaciones por mas que se perciban alteraciones, por lo que en robótica resultan útiles para el control de posición y movimiento, permitiendo acciones precisas y controladas.

\paragraph{Función de transferencia del motor} \mbox{} \vspace{8pt}

La función de transferencia en un sistema de control se representa como $G(s)$ y modela matemáticamente el comportamiento de un actuador, al cual se le aplica un estimulo o señal $R(s)$ y se obtiene una respuesta $C(s)$ por parte de él.

Para controlar la variable en cuestión debemos tener una noción sobre la salida producida. Por un lado tenemos la función $H(s)$, que es una función que toma como entrada una medición sobre la salida del actuador o sistema para producir un valor $B(s)$. Éste se suma negativamente al setpoint $R(s)$ para obtener el error entre ellas $E(s)$, que es la señal efectivamente que se introduce a $G(s)$ para logar aproximarse al setpoint y compensar el sistema.

\begin{figure}[H]
    \centering
    \includegraphics[width=0.4\linewidth]{sistema_de_control}
    \caption{Sistema de control realimentado}
    \label{fig:sistcontrolrealim}
\end{figure}

Para lograr controlar cada motor de cada rueda es necesario conocer la función de transferencia del mismo para lograr modelarlo. Al no contar con la hoja de datos del fabricante por ser un motor genérico, se realizaron una serie de experimentos para aproximar la función $G(s)$. En nuestro caso debemos obtener una función de transferencia que tenga como entrada $[Volts]$ y su salida sea $[RPM]$.

Para hacer esto se propone hacer uso del método de la constante de tiempo $\tau$. En primer lugar, consideraremos que la curva alcanza el $63.2\%$  del valor final cuando ha transcurrido un tiempo $t=\tau$. En la gráfica el valor final de la curva es 1, es decir, $y(\infty)=1$. Por lo que debemos identificar el instante $t$ para el cual se cumple que $y(t)=0.632$.

El paso siguiente es trazar una recta paralela al eje de las abscisas (eje $t$) que corresponda al $63.2\%$ del valor final de $y(t)$. Desde ese punto, se traza una recta paralela al eje de las ordenadas (eje $y$) hasta cortar el eje $t$. Este punto de intersección corresponde al valor de $\tau$.

\begin{figure}[H]
    \centering
    \includegraphics[width=0.625\linewidth]{metodo_const_tiempo_func_transf}
    \caption{Método de la constante de tiempo}
    \label{fig:metodoctetiempo}
\end{figure}

Obtenido esto, se procede a elaborar un sistema de primer orden mediante la siguiente expresión:

$$ G(s) = \frac{k}{\tau \cdot s + 1} $$

Donde $k$ representa la ganancia estática, que es el valor al que tiende la salida del sistema cuando la entrada es una señal constante y el tiempo tiende a infinito. En otras palabras, es el factor de escala entre la entrada y la salida en estado estacionario.

Para identificar la ganancia estática $k$, partimos de que en estado estacionario la salida alcanza un valor constante $y(\infty)$. Además, se considera que estimulamos al actuador con una señal escalón, denominada $u(\infty)$. La ganancia estática se puede calcular como:

$$ k=\frac{y(\infty)}{u(\infty)} $$

Para medir la respuesta del motor, primero conectamos mecánicamente el eje del motorreductor a medir junto con el eje de otro motor testigo, el cual esta conectado a un osciloscopio. Con esta experiencia buscamos estimular al motorreductor con una señal escalón unitario y que el motor testigo genere una tensión que se puede registrar en el osciloscopio. De este modo obtuvimos los valores de tensión generada en el motor testigo y podemos estimar la respuesta del motorreductor. En la Figura \ref{fig:exprespmotor} se muestra un diagrama de la experiencia.

\begin{figure}[H]
    \centering
    \includegraphics[width=0.55\linewidth]{experim_resp_motor}
    \caption{Experiencia para medición de RPM}
    \label{fig:exprespmotor}
\end{figure}

En el osciloscopio logramos obtener la gráfica que se muestra en la Figura \ref{fig:curvarespmotor}. En amarillo se representa la entrada escalón de 12V hacia el motor y en verde la tensión generada en el motor testigo.

\begin{figure}[H]
    \centering
    \includegraphics[width=0.525\linewidth]{curva_resp_motor_osciloscopio}
    \caption{Curva de respuesta del motorreductor}
    \label{fig:curvarespmotor}
\end{figure}

Luego procedimos a extraer los puntos de datos y suavizar la curva para facilitar el análisis. Esto lo hicimos mediante una media ponderada con un factor $\alpha = 0.2$, obtenido experimentalmente:

$$ S_{t} = \alpha \cdot Y_{t-1} + (1 - \alpha) \cdot Y_{t} $$

A partir de ello obtuvimos la gráfica de la Figura \ref{fig:curvarespmotorsuaviz}, donde en el eje horizontal están representados instantes de tiempo equivalentes a $10[ms]$ cada uno y en el eje vertical el voltaje (expresado en $[volts]$) sensado en el motor testigo.

Ahora bien, para linealizar el sistema podemos suponer que el valor de tensión generado en el motor testigo es proporcional a las revoluciones por minuto que gira su eje. Dado que en la iteración anterior se logró la implementación de un medidor de RPM, obtuvimos en este caso que el motorreductor conectado a 12V continuos produce 92 [RPM]. En este momento podemos calcular la ganancia estática $k$:

$$ k=\frac{y(\infty)}{u(\infty)}=\frac{92[RPM]}{12[V]}=7.66[RPM/V] $$

\begin{figure}[H]
    \centering
    \includegraphics[width=1.0\linewidth]{resp_motor_suavizado}
    \caption{Curva de respuesta del motor suavizada}
    \label{fig:curvarespmotorsuaviz}
\end{figure}

Por otra parte, determinamos que $\tau=70[ms]$ por alcanzar en esa marca el valor de tensión $0,564[V]$, rondando el $63.2\%$ del valor en estado estacionario. Aplicando la expresión para obtener un sistema de primer orden obtenemos:

$$ G(s) = \frac{7.66}{0.07 \cdot s + 1} $$

% https://dademuchconnection.wordpress.com/2021/06/26/aproximacion-teorica-de-una-curva-de-respuesta-real/
% https://controlautomaticoeducacion.com/control-realimentado/ziegler-nichols-sintonia-de-control-pid/

\paragraph{Diseño del controlador} \mbox{} \vspace{8pt}

Es tarea del controlador intervenir para corregir las fluctuaciones que sufre el sistema. Por lo general los controladores PID basan su funcionamiento en 3 parámetros fundamentales, los cuales son $K_p$, $K_i$ y $K_d$; correspondiéndose con el factor de acción proporcional, integrativa y derivativa, respectivamente.

En nuestro caso, optamos por utilizar un PID aditivo por ser de sencilla implementación y por ser mas rápida la convergencia a los coeficientes óptimos. En la Figura \ref{fig:pidaditivo} se muestra un diagrama del mismo.

\begin{figure}[H]
    \centering
    \includegraphics[width=1.0\linewidth]{images/pid_aditivo}
    \caption{PID aditivo}
    \label{fig:pidaditivo}
\end{figure}

\paragraph{Implementación} \mbox{} \vspace{8pt}

Cada uno de los cuatro motores del robot tiene su propio controlador PID, todos con coeficientes idénticos. Se implementan dentro del microcontrolador ESP32 haciendo uso de los puertos de salida PWM para controlar la velocidad y sentido de cada una de las ruedas. Además se utilizan los módulos contadores de pulsos asociados a los puertos GPIO donde se conecta cada encoder rotativo.

Como requerimiento, debe contar con tiempos de respuesta aceptables para que el sistema de control funcione adecuadamente, por lo que hicimos pruebas para descubrir el mejor $\Delta T$ de actualización del PID, en concordancia con el período de medición de RPM. Pudimos determinar que el controlador cumple con el requerimiento teniendo un periodo mínimo de $T=100[ms]$.

Por cada motor se tiene una estructura como la siguiente:

\begin{figure}[H]
    \centering
    \hspace*{-0.75cm}
    \includegraphics[width=1.1\linewidth]{images/diag_comp_esp32_pid_solo.png}
    \caption{Diagrama de componentes del firmware de la ESP32}
    \label{fig:diagcomponentesp32}
\end{figure}

En primer lugar, se tiene una interrupción periódica de $T=100[ms]$ donde se toman los valores de cada uno de los módulos contadores de pulso (PCNT) del microcontrolador, vinculados a los encoders rotativos de cada rueda. Existe una cola por cada MotorTask para recibir esta información desde la interrupción. Por otro lado, la tarea MasterTask cuenta con una cola (queue) única que comparte con todas las MotorTask para recibir el feedback de las mismas. Asimismo, cada tarea MotorTask tiene una cola que comparte con la tarea MasterTask, donde se envía un setpoint independiente a cada rueda.

Al ser el robot omnidireccional, si se establecen todas las ruedas a la misma velocidad y en sentidos determinados, es posible lograr que el robot de mueva a lo largo de un vector en linea recta sobre el plano. De este modo podemos llevar a cabo pruebas con las que encontramos los coeficientes del controlador.

A continuación, incurrimos en la utilización de la técnica de Ziegler-Nichols a lazo abierto y logramos obtener valores aproximados de $K_p$, $K_i$ y $K_d$. Introdujimos estos valores en el controlador y colocamos el robot en el suelo para realizar una serie de pruebas en línea recta. Con estos coeficientes no notamos un comportamiento óptimo, por lo que nos enfocamos en iterar sobre los valores y ajustar las constantes.

De este modo se realizaron iteraciones aumentando o disminuyendo cada uno de los coeficientes para buscar el punto óptimo de funcionamiento. Comenzamos con un valor de $K_p=10$ y un setpoint de $65[RPM]$, los demás coeficientes en cero. De este modo se fue aproximando el valor hasta lograr un arranque rápido pero sin demasiado sobrepaso. Una vez obtenido lo anterior, se procedió a ajustar $K_i$ observando cómo variaba el error acumulado al alcanzar las RPM deseadas. Para finalizar, se determinó el factor $K_d$ mediante observación de la respuesta del robot ante picos de perturbaciones.

Finalmente, obtenemos la siguiente estructura para el control de las 4 ruedas independientes del robot:

\begin{figure}[H]
    \centering
    \hspace*{-0.75cm}
    \includegraphics[width=1.05\linewidth]{images/diag_comp_esp32_pid_solo_todos_los_motores.png}
    \caption{Estructura de control de los motores}
    \label{fig:diagcommpesp32pidmotores}
\end{figure}

A continuación se detalla el funcionamiento de una de las tareas de los motores con la tarea principal. El proceso es similar para las demás MotorTask.

\begin{figure}[H]
    \centering
    \includegraphics[width=1.1\linewidth]{images/diag_secuencia_pid_solo.png}
    \caption{Diagrama de secuencia de las tareas de control para los motores}
    \label{fig:diagsecuenciapidsolo}
\end{figure}


% fuentes de esta seccion 
% https://www.mdpi.com/2076-3417/12/5/2606
% https://tesis.ipn.mx/jspui/bitstream/123456789/18688/1/Sistema%20de%20control%20para%20el%20desplazamiento%20omnidireccional%20de%20un%20robot%20m%C3%B3vil.pdf


\subsubsection{Modelo cinemático}

La cinemática se define como la ciencia que estudia el movimiento de objetos sin tomar en cuenta sus inercias. Dentro de la misma se estudia la posición, la velocidad, la aceleración y todas las demás derivadas de alto orden de las variables de posición con respecto al tiempo.

Para lograr representar matemáticamente al robot se utilizan ecuaciones que relacionan las velocidades de las ruedas con la velocidad lineal y angular del robot en el plano. Estas ecuaciones se derivan de la disposición geométrica de las ruedas y las características de las ruedas omnidireccionales. Utilizando matrices de transformación, es posible describir cómo las velocidades de las ruedas se combinan para producir el movimiento deseado del robot. Para ello se definen dos elementos fundamentales, la ecuación cinemática directa y la ecuación cinemática inversa. \cite{tzafestas2013introduction}

En el contexto de un robot omnidireccional de 4 ruedas, la cinemática directa permite traducir un vector de movimiento lineal del robot en velocidades específicas para cada rueda independiente. Cada rueda se mueve independientemente y por su configuración, permite que el robot se pueda desplazar lateralmente, hacia adelante, hacia atrás, gire sobre su propio eje y realice trayectorias curvas. \cite{rijalusalamkinematics}

Es importante que tengamos en cuenta que para tener un sistema de control acotado y predecible, es necesario compensarlo de algún modo. En este caso debemos ser capaces de obtener el vector de movimiento del robot en base a las velocidades angulares medidas en las ruedas y así lograr detectar diferencias con el vector de movimiento deseado. Para ello nos resulta útil la cinemática inversa, que implica medir las velocidades angulares de cada una de las cuatro ruedas para obtener el vector de movimiento que el robot realiza. Esto es crucial para el control y navegación del robot dado que corrige las alteraciones de un entorno dinámico.

El desarrollo de estas ecuaciones implica calcular cómo las velocidades de cada rueda contribuyen al movimiento global del robot y cómo se deben ajustar estas velocidades para lograr una trayectoria específica. En otras palabras, estas ecuaciones toman en cuenta la disposición y orientación de las ruedas y cómo contribuyen al movimiento global del robot, aprovechando al máximo la capacidad omnidireccional del robot.

Una vez logrado un modelo cinemático que aproxime bien el sistema real, deberíamos poder establecer un vector de movimiento deseado y obtener la velocidad a la que se debe colocar cada rueda para poder transcurrirlo. No solo debería poder hacer movimientos rectos en cualquier dirección, sino que también sería capaz de hacer movimientos rotatorios con traslación sobre el plano, formando trayectorias elípticas. \cite{rijalusalamkinematics}

En nuestro caso se trata de un robot cuadrado con 1 rueda por cada lado cuya orientación respecto al robot es fija. Ademas de ello, se utilizan ruedas de tipo Omni-wheel, las cuales son similares a las ruedas Mecanum. Son ruedas que cuentan con pequeños discos (llamados rodillos) alrededor de la circunferencia perpendiculares a la dirección de giro. El efecto es que la rueda puede moverse con toda su fuerza, pero también se deslizará lateralmente con gran facilidad.

\begin{figure}[H]
    \centering
    \includegraphics[width=0.3\linewidth]{omni-wheel-wikipedia}
    \caption{Rueda omnidireccional Omni-Wheel}
    \label{fig:ruedaomniwheel}
\end{figure}

Teniendo en cuenta lo anterior, para la construcción del modelo cinemático se consideran las siguientes limitaciones:

\begin{itemize}
    \item El robot se mueve sobre una superficie plana lisa.
    \item No existen elementos flexibles en la estructura del robot.
    \item El eje de direccionamiento de las ruedas siempre es perpendicular al suelo.
    \item No se consideran ningún tipo de fricciones contra el suelo.
\end{itemize}

\paragraph{Desarrollo} \mbox{} \vspace{8pt} \\
Al suponerse la rueda como un elemento rígido, se establece el principio de que las ruedas en contacto con el suelo se comportan como una articulación planar de tres grados de libertad, con lo que se propone el sistema de referencia descrito en la siguiente figura:

\begin{figure}[H]
    \centering
    \includegraphics[width=0.35\linewidth]{rueda_modelo_cinematico}
    \caption{Vectores actuantes en una rueda}
    \label{fig:vectoresrueda}
\end{figure}

El eje $V_y$ determina el sentido normal de avance de la rueda, el eje $V_x$ indica los desplazamientos laterales y $\omega_z$ la velocidad rotacional que se produce cuando el vehículo realiza un giro.

Definimos al robot sobre el plano cartesiano, donde se establece el marco de referencia global representado por $oxy$ y el marco de referencia local del robot $o_rx_ry_r$, donde el marco de referencia local se encuentra alineado con el marco de referencia global.

\begin{figure}[H]
    \centering
    \includegraphics[width=0.5\linewidth]{robot_en_el_plano_mod_cinem}
    \caption{Marco de referencia del robot y del espacio}
    \label{fig:marcorefrobotenelplano}
\end{figure}

Además podemos representar el robot y la distribución de ruedas del siguiente modo:

\begin{figure}[H]
    \centering
    \includegraphics[width=0.5\linewidth]{images/modelo_cinematico_robot_ruedas.png}
    \caption{Descomposición de vectores del robot}
    \label{fig:vectoresrobotmodelocinem}
\end{figure}

Se establece que el angulo entre las ruedas y el cuerpo del robot es fijo y está dado por:

$$ \alpha_1 = \frac{\pi}{4} = 45^{\circ} $$
$$ \alpha_2 = \frac{3\pi}{4} = 135^{\circ} $$
$$ \alpha_3 = \frac{5\pi}{4} = 225^{\circ} $$
$$ \alpha_4 = \frac{7\pi}{4} = 315^{\circ} $$

\textbf{Cinemática Inversa} \mbox{} \vspace{8pt}

Para obtener el modelo cinemático, partimos del desarrollo de cómo afecta cada una de las ruedas al movimiento total del robot. Para ello comenzamos con la descripción del movimiento de un cuerpo rígido descrito en la Figura \ref{fig:movimientocuerporigido} \cite{islassistcontrolomni}.

$$ V_p = V_Q + W \times L $$

Utilizando el modelo de movimiento de cuerpo rígido, podemos expresar para el robot:

$$ V_{R1} = V_{01} + \omega_1 \times r_1 $$
$$ V_{R2} = V_{02} + \omega_2 \times r_2 $$
$$ V_{R3} = V_{03} + \omega_3 \times r_3 $$
$$ V_{R4} = V_{04} + \omega_4 \times r_4 $$

\begin{figure}[H]
    \centering
    \includegraphics[width=0.3\linewidth]{images/movimiento_cuerpo_rigido.png}
    \caption{Movimiento de un cuerpo rígido}
    \label{fig:movimientocuerporigido}
\end{figure}

Se consideran $ V_{01}, V_{02}, V_{03}, V_{04} $ nulos dado que no existe deslizamiento entre las ruedas y el piso, además si todas las ruedas tienen el mismo radio, podemos expresar:

$$ V_{R1} = \omega_1 \times r $$
$$ V_{R2} = \omega_2 \times r $$
$$ V_{R3} = \omega_3 \times r $$
$$ V_{R4} = \omega_4 \times r $$

Ahora, podemos analizar el marco de referencia de la rueda respecto al marco de referencia del robot. Para ello:

\begin{figure}[H]
    \centering
    \includegraphics[width=0.6\linewidth]{images/modelo_cinematico_robot_vector.png}
    \caption{Vectores del robot en el marco de referencia local}
    \label{fig:robotmarcoreflocal}
\end{figure}

Se plantea el caso para una de las ruedas. Para obtener el vector $V_{R1}$ en base al marco de referencia denotado por $U_1$, $U_2$ y $U_3$, podemos hacer:

$$ V_{R1} = V_{M1} + U_3 \times L $$

De modo que podemos expresar $V_{M1}$ en función de $U_1$ y $U_2$:

$$ V_{M1} = V_{RU_1} + V_{RU_2} $$
$$ V_{M1} = -U_1 \cdot sen(\alpha_1) + U_2 \cdot cos(\alpha_1) $$

Obteniendo finalmente que:

$$ V_{R1} = -U_1 \cdot sen(\alpha_1) + U_2 \cdot cos(\alpha_1) + U_3 \times L $$

Al realizar el mismo procedimiento para las demás ruedas obtenemos:

$$ V_{R1} = -U_1 \cdot sen(\alpha_1) + U_2 \cdot cos(\alpha_1) + U_3 \times L $$
$$ V_{R2} = -U_1 \cdot sen(\alpha_1) + U_2 \cdot cos(\alpha_1) + U_3 \times L $$
$$ V_{R4} = -U_1 \cdot sen(\alpha_1) + U_2 \cdot cos(\alpha_1) + U_3 \times L $$
$$ V_{R3} = -U_1 \cdot sen(\alpha_1) + U_2 \cdot cos(\alpha_1) + U_3 \times L $$

Ahora bien, si $U_1$, $U_2$ y $U_3$ están alineados respecto al marco de referencia global $0XY$, podemos expresar:

$$ \begin{bmatrix} U_1 \\ U_2 \\ U_3 \end{bmatrix} = \begin{bmatrix} V_x \\ V_y \\ V_\theta \end{bmatrix} $$

Entonces ahora podemos hacer la siguiente igualdad con lo obtenido:

$$ \omega_1 \times r = -V_x \cdot sen(\alpha_1) + V_y \cdot cos(\alpha_1) + V_\theta \times L $$
$$ \omega_2 \times r = -V_x \cdot sen(\alpha_2) + V_y \cdot cos(\alpha_2) + V_\theta \times L $$
$$ \omega_3 \times r = -V_x \cdot sen(\alpha_3) + V_y \cdot cos(\alpha_3) + V_\theta \times L $$
$$ \omega_4 \times r = -V_x \cdot sen(\alpha_4) + V_y \cdot cos(\alpha_4) + V_\theta \times L $$

Originalmente partimos de que necesitamos una matriz de conversión de modo que podamos obtener las velocidades de las ruedas en base a un vector dado, donde la entrada es $V_x, V_y$ expresadas en $[m/seg]$ y $V_\theta$ expresada en $[RPM]$. Por otra parte, la salida $\omega_n$ se determina en $[rad/seg]$:

$$ \begin{bmatrix} w_1 \\ w_2 \\ w_3 \\ w_4 \\ \end{bmatrix} = IK \cdot \begin{bmatrix} V_x \\ V_y \\ V_\theta \\ \end{bmatrix} $$

Con las expresiones anteriores, obtenemos que la matriz cinemática inversa se puede expresar como:

$$ IK = 
    \frac{1}{r}
    \cdot
    \begin{bmatrix}
        {-sen(\alpha_1)} & {cos(\alpha_1)} & L \\
        {-sen(\alpha_2)} & {cos(\alpha_2)} & L \\
        {-sen(\alpha_3)} & {cos(\alpha_3)} & L \\
        {-sen(\alpha_4)} & {cos(\alpha_4)} & L \\
    \end{bmatrix} $$

De modo que podemos expresar la ecuación para obtener las velocidades de las ruedas expresadas en $[rad/seg]$ como se muestra debajo.

$$ \begin{bmatrix} w_1 \\ w_2 \\ w_3 \\ w_4 \\ \end{bmatrix} = \frac{1}{r} \cdot \begin{bmatrix}
    {-sen(\frac{\pi}{4})} & {cos( \frac{\pi}{4})} & L \\
    {-sen(\frac{3\pi}{4})} & {cos(\frac{3\pi}{4})} & L \\
    {-sen(\frac{5\pi}{4})} & {cos(\frac{5\pi}{4})} & L \\
    {-sen(\frac{7\pi}{4})} & {cos(\frac{7\pi}{4})} & L \\
\end{bmatrix} \cdot
\begin{bmatrix} V_x \\ V_y \\ V_\theta \\ \end{bmatrix} $$

Dado que las MotorTask de cada una de las ruedas recibe el setpoint en $[RPM]$, debemos hacer una conversión a esa unidad. Por lo que planteamos una matriz que convierta los valores de las dos primeras columnas de la matriz cinemática, que son los coeficientes que nos interesa convertir a $[RPM]$, dado que los de la última columna ya están expresados en esa unidad.

$$ \begin{bmatrix} w_1 \\ w_2 \\ w_3 \\ w_4 \\ \end{bmatrix} =
    \frac{1}{r}
    \cdot
    \begin{bmatrix}
        {-sen(\frac{\pi}{4})} & {cos( \frac{\pi}{4})} & L \\
        {-sen(\frac{3\pi}{4})} & {cos(\frac{3\pi}{4})} & L \\
        {-sen(\frac{5\pi}{4})} & {cos(\frac{5\pi}{4})} & L \\
        {-sen(\frac{7\pi}{4})} & {cos(\frac{7\pi}{4})} & L \\
    \end{bmatrix}
    \cdot
    \begin{bmatrix}
        {\frac{60}{2 \pi}} & {0} & {0} \\
        {0} & {\frac{60}{2 \pi}} & {0} \\
        {0} & {0} & {1}                \\
    \end{bmatrix}
    \cdot
    \begin{bmatrix} V_x \\ V_y \\ V_\theta \\ \end{bmatrix} $$


\textbf{Cinemática Directa} \mbox{} \vspace{8pt}

Para lograr el control del robot sobre el plano es necesario hacer una comparación entre el vector de movimiento deseado y un vector de movimiento inferido en base a la medición de la velocidad de las ruedas, de modo que se cierra el lazo de control. El error existente entre el vector real y el deseado es debido a que los motores no establecen las RPM inmediatamente por cuestiones de inercia e imperfecciones en el terreno.

Para obtener el vector de velocidad lineal real que efectivamente realiza el robot, hacemos uso de las mediciones de velocidad angular en cada rueda y se ingresan a la ecuación cinemática directa. Esta matriz se obtiene a partir de la pseudoinversa de la matriz cinemática inversa. \cite{islassistcontrolomni}

De modo que, en primer lugar obtenemos la matriz de conversión que nos permite obtener el vector lineal en base a las velocidades de las ruedas expresadas en $[rad/seg]$:

$$ \begin{bmatrix} V_x \\ V_y \\ V_\theta \\ \end{bmatrix} = DK \cdot \begin{bmatrix} w_1 \\ w_2 \\ w_3 \\ w_4 \\ \end{bmatrix} $$

Luego, al realizar pseudoinversa de la matriz $IK$:

$$  DK = 
    r
    \cdot 
    \begin{bmatrix}
        {\frac{-sen(\alpha_1)}{2}} & {\frac{-sen(\alpha_2)}{2}} & {\frac{-sen(\alpha_3)}{2}} & {\frac{-sen(\alpha_4)}{2}} \\
        {\frac{cos(\alpha_1)}{2}}  & {\frac{cos(\alpha_2)}{2}}  & {\frac{cos(\alpha_3)}{2}}  & {\frac{cos(\alpha_4)}{2}}  \\
        {\frac{1}{4L}}  & {\frac{1}{4L}}  & {\frac{1}{4L}}  & {\frac{1}{4L}}  \\
    \end{bmatrix} $$

Por lo que obtenemos la siguiente expresión para obtener el vector de movimiento del robot dependiendo las velocidades de las ruedas:

$$ \begin{bmatrix} V_x \\ V_y \\ V_\theta \\ \end{bmatrix} = 
    r
    \cdot 
    \begin{bmatrix}
        {\frac{-sen(\alpha_1)}{2}} & {\frac{-sen(\alpha_2)}{2}} & {\frac{-sen(\alpha_3)}{2}} & {\frac{-sen(\alpha_4)}{2}} \\
        {\frac{cos(\alpha_1)}{2}}  & {\frac{cos(\alpha_2)}{2}}  & {\frac{cos(\alpha_3)}{2}}  & {\frac{cos(\alpha_4)}{2}}  \\
        {\frac{1}{4L}}  & {\frac{1}{4L}}  & {\frac{1}{4L}}  & {\frac{1}{4L}}  \\
    \end{bmatrix}
    \cdot
    \begin{bmatrix} w_1 \\ w_2 \\ w_3 \\ w_4 \\ \end{bmatrix} $$

Esta expresión toma las velocidades angulares de las ruedas en $[rad/seg]$, por lo que necesitamos convertir la entrada para que pueda ser $[RPM]$. Para ello planteamos una matriz que solo convierta los valores de las dos primeras filas, que son los coeficientes que nos interesa convertir a $[RPM]$, dado que los de la última fila ya están expresados en esa unidad.

$$ \begin{bmatrix} V_x \\ V_y \\ V_\theta \\ \end{bmatrix} = 
    r
    \cdot 
    \begin{bmatrix}
        {\frac{2\pi}{60}} & {0} & {0}  \\
        {0} & {\frac{2\pi}{60}} & {0} \\
        {0} & {0} & {1} \\
    \end{bmatrix}
    \cdot
    \begin{bmatrix}
        {\frac{-sen(\alpha_1)}{2}} & {\frac{-sen(\alpha_2)}{2}} & {\frac{-sen(\alpha_3)}{2}} & {\frac{-sen(\alpha_4)}{2}} \\
        {\frac{cos(\alpha_1)}{2}}  & {\frac{cos(\alpha_2)}{2}}  & {\frac{cos(\alpha_3)}{2}}  & {\frac{cos(\alpha_4)}{2}}  \\
        {\frac{1}{4L}}  & {\frac{1}{4L}}  & {\frac{1}{4L}}  & {\frac{1}{4L}}  \\
    \end{bmatrix}
    \cdot
    \begin{bmatrix} w_1 \\ w_2 \\ w_3 \\ w_4 \\ \end{bmatrix} $$


\paragraph{Implementación} \mbox{} \vspace{8pt}

Para la implementación se propuso que el modelo cinemático sea contenido dentro de la MasterTask dado que es quien recibe el feedback de todas las tareas MotorTask y conoce el estado actual de cada uno de los motores. Un diagrama de secuencia se detalla más adelante en este capítulo.

\begin{figure}[H]
    \centering
    \hspace*{-0.75cm}
    \includegraphics[width=1.1\linewidth]{images/diag_comp_esp32_modelo_cinem.png}
    \caption{Estructura de control de los motores con el Modelo Cinemático}
    \label{fig:diagcomponentesp32modelocinem}
\end{figure}


\subsubsection{Odometría}

La odometría es el proceso mediante el cual un robot estima su posición y orientación en el espacio a lo largo del tiempo. Esta técnica se utiliza para obtener una aproximación de la trayectoria recorrida por el robot, a partir de los desplazamientos medidos en sus ruedas o actuadores, generalmente utilizando sensores como pueden ser encoders rotativos o lineales.

Dado que necesitamos estimar la posición y orientación del robot en el espacio a lo largo del tiempo, nos podemos basar en las lecturas de los sensores de las ruedas. Para inferir la distancia recorrida podemos medir las velocidades angulares o la distancia recorrida de cada una de las cuatro ruedas en simultáneo. Estos datos se obtienen a partir de los encoders rotativos instalados en las ruedas.

Se probaron distintas alternativas para el proceso de odometría. Primeramente se probó de modo que cada rueda recibe la distancia que debe recorrer el robot y de modo independiente mide cuanto recorre para determinar si debe detenerse o no. Concluimos que esta técnica tiene algunos problemas porque no considera a nivel global el movimiento del robot y como influyen las perturbaciones de las demás ruedas sobre una de ellas. Es por ello que optamos por hacer uso del modelo cinemático para realizar la odometría. Este método no excluye de errores al proceso, pero tiene en consideración todas las ruedas al momento de determinar la distancia recorrida por el robot dada la sumatoria de la acción de todas ellas. \cite{palacinodometry}

La odometría se basa en el uso del modelo cinemático para convertir estas velocidades angulares en velocidades lineales y angulares del robot en el plano, es decir, obtener un vector de movimiento en base a las mediciones. Para ello, el primer paso en el proceso es utilizar las ecuaciones del modelo cinemático directo para calcular las velocidades lineales y angulares en el instante $t$ a partir de las velocidades de las ruedas: ($V_xt$, $V_yt$, $V_\theta t$). Posteriormente se pueden integrar estas velocidades a lo largo del tiempo para estimar la posición ($x, y$) y la orientación ($\theta$) del robot. La integración se realiza mediante métodos numéricos, típicamente utilizando un enfoque de integración discreta, como el método de Euler. Por lo que en cada instante de tiempo que se toma una medición, se actualiza la posición y orientación del robot utilizando las siguientes ecuaciones, donde ($\Delta$t) es el intervalo de tiempo entre dos mediciones consecutivas:

$$ x_{t+1} = x_t + V_xt \cdot \Delta t $$
$$ y_{t+1} = y_t + V_yt \cdot \Delta t $$
$$ \theta_{t+1} = \theta_t + V_\theta t \cdot \Delta t $$

Es importante considerar que la odometría realizada para ruedas puede acumular errores debido a factores como el deslizamiento de las ruedas, inexactitudes en las mediciones de los encoders o imprecisiones en los cálculos. Por lo tanto, es común combinar la odometría con otros métodos de localización, como el uso de sensores adicionales (LIDAR, cámaras, etc.) y técnicas de fusión de datos (como los filtros de Kalman), para mejorar la precisión y robustez de la estimación de la posición del robot.

Para la realización de las pruebas colocamos marcas en el piso a 1 metro de distancia entre sí y realizamos iteraciones para verificar que la odometría opera dentro de los rangos de precisión establecidos.


\subsubsection{Envío y recepción de comandos}

Se implementó un sistema de comunicación basado en el protocolo MQTT (Message Queuing Telemetry Transport) para enviar y recibir informacion del robot. Se envían vectores a recorrer con una estructura de datos a modo de tupla de valores que representa la distancia a recorrer, las velocidades lineales y angulares del robot: $distancia [m]$, $V_x [m/seg]$, $V_y [m/seg]$ y $V_\theta [RPM]$.

Además, el robot es capaz de enviar información por medio de MQTT sobre la odometría y el estado en tiempo real. Esto resulta útil dado que posibilita monitorear el estado del robot y realizar ajustes si es necesario.

Por un lado, las velocidades $V_x$ y $V_y$ determinan un sentido de movimiento en linea recta sobre el plano. Por otro lado, $V_\theta$ controla la rotación del robot sobre su propio eje, permitiéndole girar en el plano horizontal.

Cuando se envía una tupla de valores de velocidad mediante MQTT, el robot calcula la velocidad para cada rueda usando el modelo cinemático y establece los motores a la velocidad estipulada, luego se detiene cumplida la distancia recorrida. En la Figura \ref{fig:diagcomponentesp32conmodelocinem} se muestra un diagrama de secuencia con los componentes integrados hasta el momento.

\begin{figure}[htb]
    \centering
    \includegraphics[width=1\linewidth]{images/diag_secuencia_full_modelo_cinematico.png}
    \caption{Diagrama de secuencia de la ESP32 con el Modelo Cinemático}
    \label{fig:diagcomponentesp32conmodelocinem}
\end{figure}


\subsubsection{Campo de pruebas}

En cualquier estudio experimental, es vital controlar todas las variables posibles para aislar el efecto de las variables independientes. La superficie sobre la que se desplaza el robot puede influir significativamente en su comportamiento, ya que las irregularidades de la misma pueden inducir deslizamientos o atascos.

La elección de una superficie determinada, consistente y constante en cada experimento es crucial para asegurar la fiabilidad de los resultados obtenidos, al proporcionar un entorno controlado. Mantener una superficie constante permite replicar las condiciones experimentales y comparar resultados de manera precisa y predecible. Si la superficie varía, se introducen variables adicionales que pueden afectar el desempeño del robot, dificultando la comparación entre experimentos.

En primera instancia, la superficie sobre la cual comenzamos las pruebas del prototipo fue el suelo del Laboratorio. En los sucesivos experimentos notamos cierta inconsistencia entre cada iteración, manteniendo los parámetros de funcionamiento constantes. En búsqueda de mejorar la consistencia entre los experimentos, hicimos algunas pruebas sobre un tablón de madera lisa y notamos que la mejoría es sustancial. En consecuencia, se dispone de una superficie de madera sobre la cual realizamos las subsiguientes pruebas.


\subsection{Testing y pruebas}

% https://en.wikibooks.org/wiki/LaTeX/Tables

\begin{testtableformat}
    \hline \rowcolor{test_header_color}
        Test ID             & TC\_01\_00 \\
    \hline
        Tipo de test        & Test unitario \\
    \hline
        Objeto de prueba    & Comunicación inalámbrica \\
    \hline
        Requerimiento       & RF6 \\
    \hline
        Nombre              & Comunicación inalámbrica para monitoreo y envío de comandos por MQTT \\
    \hline
        Descripción         & Verificar que los comandos son recibidos correctamente en el robot y que el robot envía información de estado en el formato correcto \\
    \hline
        Precondición        & PRECOND\_A \\
    \hline
        Pasos del test      & \begin{enumerate}
                                \item Enviar un nuevo setpoint con parámetros de distancia [50cm $\sim$ 200cm] y velocidad $\pm$[0.25m/seg $\sim$ 0.75m/seg]
                                \item Verificar que el robot recibe el comando correctamente y que envía reportes
                                \item Repetir desde el paso 1) con diferentes valores
                            \end{enumerate} \\
    \hline
        Resultado esperado  & El robot recibe e interpreta los comandos, al mismo tiempo que reporta información sobre su estado periódicamente \\
    \hline
        Resultado obtenido  & El robot realiza el comportamiento esperado, envía, recibe e interpreta comandos \\
    \hline
        Observaciones       & - \\
    \hline
\end{testtableformat}


\begin{testtableformat}
    \hline \rowcolor{test_header_color}
        Test ID             & TC\_01\_01 \\
    \hline
        Tipo de test        & Test unitario \\
    \hline
        Objeto de prueba    & PWM - Medidor de RPM \\
    \hline
        Requerimiento       & RF1 - RF5 \\
    \hline
        Nombre              & Medidor de RPM \\
    \hline
        Descripción         & Comprobar que al establecer el motor a determinada velocidad ésta se corresponda con la medición obtenida por el medidor de RPM \\
    \hline
        Precondición        & PRECOND\_A \\
    \hline
        Pasos del test      & \begin{enumerate}
                                \item Enviar un valor de PWM al controlador del motor entre [0 $\sim$ 1023]
                                \item Verificar que el motor se establece a la velocidad deseada y que el medidor de RPM informa el valor correcto
                                \item Repetir desde el paso 1) con diferentes valores
                            \end{enumerate} \\
    \hline
        Resultado esperado  & Se obtiene el valor correcto de RPM medido en cada iteración \\
    \hline
        Resultado obtenido  & En todas las iteraciones las mediciones reportan una velocidad cercana a la establecida \\
    \hline
        Observaciones       & La velocidad máxima que podemos obtener del motor sin carga son 92 RPM (PWM = 1023) y el motor comienza a girar continuamente cuando las RPM se establecen en una velocidad mínima de 55 RPM (PWM $\cong$ 420) \\
    \hline
\end{testtableformat}


\begin{testtableformat}
    \hline \rowcolor{test_header_color}
        Test ID             & TC\_01\_02 \\
    \hline
        Tipo de test        & Test de integración \\
    \hline
        Objeto de prueba    & PID - PWM - Medidor de RPM \\
    \hline
        Requerimiento       & RF1 - RF5 \\
    \hline
        Nombre              & PID sin carga \\
    \hline
        Descripción         & Verificar que el PID establece correctamente la velocidad de la rueda al recibir el setpoint en RPM sin carga vinculada a la rueda \\
    \hline
        Precondición        & PRECOND\_A \\
    \hline
        Pasos del test      & \begin{enumerate}
                                \item Enviar un valor de RPM al controlador PID del motor entre [0 $\sim$ 92] RPM
                                \item Verificar que el motor se establece a las RPM deseadas y que el medidor de RPM informa el valor correcto
                                \item Repetir desde el paso 1) con diferentes valores
                            \end{enumerate} \\
    \hline
        Resultado esperado  & Se obtiene el valor de RPM correcta en cada iteración \\
    \hline
        Resultado obtenido  & En todas las iteraciones se obtuvo que la rueda gira a una velocidad cercana a la establecida \\
    \hline
        Observaciones       & La velocidad máxima que podemos obtener del motor sin carga son 92 RPM y el motor comienza a girar continuamente cuando las RPM se establecen en una velocidad mínima de 55 RPM \\
    \hline
\end{testtableformat}


\begin{testtableformat}
    \hline \rowcolor{test_header_color}
        Test ID             & TC\_01\_03 \\
    \hline
        Tipo de test        & Test de integración \\
    \hline
        Objeto de prueba    & PID - PWM - Medidor de RPM \\
    \hline
        Requerimiento       & RF1 - RF5 \\
    \hline
        Nombre              & PID con carga \\
    \hline
        Descripción         & Verificar que el PID establece correctamente la velocidad de la rueda al recibir el setpoint en RPM con una carga aproximadamente igual al peso del robot \\
    \hline
        Precondición        & PRECOND\_A \\
    \hline
        Pasos del test      & \begin{enumerate}
                                \item Enviar un valor de RPM al controlador PID del motor entre [0 $\sim$ 92] RPM
                                \item Verificar que el motor se establece a las RPM deseadas y que el medidor de RPM informa el valor correcto
                                \item Repetir desde el paso 1) con diferentes valores
                            \end{enumerate} \\
    \hline
        Resultado esperado  & Se obtiene el valor de RPM correcta en cada iteración \\
    \hline
        Resultado obtenido  & En todas las iteraciones se obtuvo que la rueda gira a una velocidad cercana a la establecida \\
    \hline
        Observaciones       & La velocidad máxima que podemos obtener del motor con una carga presente son 88 RPM y el motor comienza a girar continuamente cuando las RPM se establecen en una velocidad mínima de 63 RPM \\
    \hline
\end{testtableformat}


\begin{testtableformat}
    \hline \rowcolor{test_header_color}
        Test ID             & TC\_01\_04 \\
    \hline
        Tipo de test        & Test unitario \\
    \hline
        Objeto de prueba    & Modelo Cinemático \\
    \hline
        Requerimiento       & RF2 - RF4 \\
    \hline
        Nombre              & Modelo Cinemático en línea recta \\
    \hline
        Descripción         & Comprobar que el Modelo Cinemático calcula adecuadamente las velocidades de las ruedas según un setpoint en línea recta \\
    \hline
        Precondición        & PRECOND\_B \\
    \hline
        Pasos del test      & \begin{enumerate}
                                \item Enviar al Modelo Cinemático un vector de velocidad en linea recta con valores entre $\pm$[0.25m/seg $\sim$ 0.75m/seg]
                                \item Colocar cada una de las ruedas a la velocidad calculada por el Modelo Cinemático y verificar que el robot se mueve a lo largo del vector definido
                                \item Repetir desde el paso 1) con diferentes valores
                            \end{enumerate} \\
    \hline
        Resultado esperado  & El robot se mueve en línea recta en la dirección del vector dado por $V_x$ y $V_y$ \\
    \hline
        Resultado obtenido  & Se observa que el robot realiza el comportamiento esperado \\
    \hline
        Observaciones       & - \\
    \hline
\end{testtableformat}


\begin{testtableformat}
    \hline \rowcolor{test_header_color}
        Test ID             & TC\_01\_05 \\
    \hline
        Tipo de test        & Test unitario \\
    \hline
        Objeto de prueba    & Modelo Cinemático \\
    \hline
        Requerimiento       & RF2 - RF4 \\
    \hline
        Nombre              & Modelo Cinemático en trayectorias curvas \\
    \hline
        Descripción         & Comprobar que el Modelo Cinemático calcula adecuadamente las velocidades de las ruedas según un setpoint con velocidad rotacional distinta de cero \\
    \hline
        Precondición        & PRECOND\_B \\
    \hline
        Pasos del test      & \begin{enumerate}
                                \item Enviar al Modelo Cinemático un vector de velocidad lineal nula y velocidad rotacional distinta de cero con valores entre $\pm$[0RPM $\sim$ 30RPM]
                                \item Colocar cada una de las ruedas a la velocidad calculada por el Modelo Cinemático y verificar que el robot gira sobre su eje según la velocidad rotacional dada
                                \item Repetir desde el paso 1) con diferentes valores
                            \end{enumerate} \\
    \hline
        Resultado esperado  & El robot gira sobre su eje a distintas velocidades \\
    \hline
        Resultado obtenido  & Se observa que el robot realiza el comportamiento esperado \\
    \hline
        Observaciones       & - \\
    \hline
\end{testtableformat}


\begin{testtableformat}
    \hline \rowcolor{test_header_color}
        Test ID             & TC\_01\_06 \\
    \hline
        Tipo de test        & Test unitario \\
    \hline
        Objeto de prueba    & Modelo Cinemático \\
    \hline
        Requerimiento       & RF2 - RF4 \\
    \hline
        Nombre              & Modelo Cinemático en trayectorias elípticas (lineal y curva en simultáneo) \\
    \hline
        Descripción         & Comprobar que el Modelo Cinemático calcula adecuadamente las velocidades de las ruedas según un vector de movimiento dado por velocidades lineales y velocidades angulares al mismo tiempo \\
    \hline
        Precondición        & PRECOND\_B \\
    \hline
        Pasos del test      & \begin{enumerate}
                                \item Enviar al Modelo Cinemático un vector de velocidad lineal y rotacional distintas de cero con valores para $V_x$ y $V_y$ entre $\pm$[0.25m/seg $\sim$ 0.75m/seg] y $V_r$ entre $\pm$[0RPM $\sim$ 30RPM]
                                \item Colocar cada una de las ruedas a la velocidad calculada por el Modelo Cinemático y verificar que el robot describe una trayectoria elíptica que se corresponde con el vector dado
                                \item Repetir desde el paso 1) con diferentes valores
                            \end{enumerate} \\
    \hline
        Resultado esperado  & El robot realiza trayectorias elípticas a distintas velocidades y radios de movimiento \\
    \hline
        Resultado obtenido  & El robot describe una trayectoria elíptica variable en radio y velocidad según se modifique el setpoint \\
    \hline
        Observaciones       & Al inicio hasta su convergencia, se observa un patrón en espiral y luego se torna un recorrido constante \\
    \hline
\end{testtableformat}


\begin{testtableformat}
    \hline \rowcolor{test_header_color}
        Test ID             & TC\_01\_07 \\
    \hline
        Tipo de test        & Test de integración \\
    \hline
        Objeto de prueba    & Odometría - Modelo cinemático \\
    \hline
        Requerimiento       & RF2 - RF3 - RF4 - RF5 \\
    \hline
        Nombre              & Odometría en línea recta \\
    \hline
        Descripción         & Verificar que el robot recorre la distancia establecida \\
    \hline
        Precondición        & PRECOND\_B \\
    \hline
        Pasos del test      & \begin{enumerate}
                                \item Enviar al robot un setpoint en linea recta de distancia entre [50cm $\sim$ 400cm] y velocidad $\pm$[0.25m/seg $\sim$ 0.75m/seg]
                                \item Verificar que el robot recorre el vector dado a lo largo de la distancia determinada
                                \item Repetir desde el paso 1) con diferentes valores
                            \end{enumerate} \\
    \hline
        Resultado esperado  & El robot recorre la distancia establecida \\
    \hline
        Resultado obtenido  & El robot a distancias menores a 20cm no logra una buena precisión. Con distancias de al menos 35cm se obtiene una buena precisión en la medición, de alrededor de +-3cm. \\
    \hline
        Observaciones       & Se probó hasta recorridos de 4 metros por limitaciones de espacio.  \\
    \hline
\end{testtableformat}


\begin{testtableformat}
    \hline \rowcolor{test_header_color}
        Test ID             & TC\_01\_08 \\
    \hline
        Tipo de test        & Test de integración \\
    \hline
        Objeto de prueba    & Odometría - Modelo cinemático \\
    \hline
        Requerimiento       & RF2 - RF3 - RF4 - RF5 \\
    \hline
        Nombre              & Odometría en trayectorias curvas \\
    \hline
        Descripción         & Verificar que el robot recorre la distancia establecida \\
    \hline
        Precondición        & PRECOND\_B \\
    \hline
        Pasos del test      & \begin{enumerate}
                                \item Enviar al robot un setpoint de trayectoria curva con distancia entre [50cm $\sim$ 400cm], velocidad lineal entre $\pm$[0.25m/seg $\sim$ 0.75m/seg] y velocidad rotacional entre $\pm$[0RPM $\sim$ 30RPM]
                                \item Verificar que el robot recorre el vector dado a lo largo de la distancia determinada
                                \item Repetir desde el paso 1) con diferentes valores
                            \end{enumerate} \\
    \hline
        Resultado esperado  & El robot recorre la distancia establecida \\
    \hline
        Resultado obtenido  & El robot a distancias menores a 32cm no logra una buena precisión. Con distancias de al menos 40cm se obtiene una buena precisión en la medición, de alrededor de +-6cm. \\
    \hline
        Observaciones       & Se probó hasta recorridos de 4 metros por limitaciones de espacio.  \\
    \hline
\end{testtableformat}


\begin{testtableformat}
    \hline \rowcolor{test_header_color}
        Test ID             & TC\_01\_09 \\
    \hline
        Tipo de test        & Test de sistema \\
    \hline
        Objeto de prueba    & Comunicación inalámbrica - PID - Modelo cinemático - Odometría \\
    \hline
        Requerimiento       & RF1 - RF2 - RF3 - RF4 - RF5 - RF6 \\
    \hline
        Nombre              & Prueba de sistema integrado \\
    \hline
        Descripción         & Comprobar que el robot realiza trayectorias en una dirección y longitud determinadas, además de reportar información de estado \\
    \hline
        Precondición        & PRECOND\_B \\
    \hline
        Pasos del test      & \begin{enumerate}
                                \item Enviar al robot un setpoint con distancia entre [50cm $\sim$ 400cm] y velocidad lineal entre $\pm$[0.25m/seg $\sim$ 0.75m/seg]
                                \item Verificar que el robot recorre el vector dado a lo largo de la distancia determinada y que reporta periódicamente mediciones y estado actual
                                \item Repetir desde el paso 1) con diferentes valores
                            \end{enumerate} \\
    \hline
        Resultado esperado  & El robot responde correctamente al vector y la distancia establecida, además reporta periódicamente el estado de mediciones de distancia y velocidad \\
    \hline
        Resultado obtenido  & El robot recibe comandos de trayectorias con vectores y distancias determinadas, se observa que realiza las trayectorias de manera acorde dentro de los límites observados en las pruebas unitarias y de integración. Al mismo tiempo se reciben los reportes de estado periódicos por parte del robot \\
    \hline
        Observaciones       & Se probó hasta recorridos de 4 metros por limitaciones de espacio. \\
    \hline
\end{testtableformat}

\subsection{Resultados}

En esta iteración se logró efectivamente el desarrollo de un prototipo funcional del robot. Es capaz de recibir comandos, realizar trayectorias rectas y curvas, teniendo también mediciones de odometría.

Las cuatro ruedas son compensadas mediante un controlador PID cada una y se dispone del modelo cinemático para determinar el vector de movimiento.

\begin{figure}[H]
    \centering
    \includegraphics[trim={0 1cm 0 1cm}, clip, width=0.55\linewidth]{images/robot_sin_imanes_prototipo.png}
    \caption{Prototipo del robot}
    \label{fig:primerprototiporobot}
\end{figure}

\subsection{Riesgos superados}

En esta iteración se logro avanzar sobre el riesgo RI-03, pero para poder superarlo debemos explotar aun mas las capacidades del robot en las siguientes iteraciones.

Por otro lado, el riesgo RI-05 se logra superar en parte porque en esta iteración se incorporaron la mayoría de componentes esenciales.

\begin{center} \begin{tabular}{|p{0.10\linewidth}|p{0.65\linewidth}|}
\hline \rowcolor{test_header_color}
    ID & Descripción \\ 
\hline
    RI-03 & Prestaciones insuficientes de componentes. \\
\hline
    RI-05 & Dificultad en conseguir determinados componentes. \\
\hline
\end{tabular} \end{center}

\subsection{Conclusiones}

En primer lugar, se logró el control individual de velocidad y sentido de cada una de las ruedas y por otro lado, se desarrolló un modelo matemático del robot teniendo en cuenta el efecto de todas las ruedas.
En la presente iteración se obtuvo un prototipo que realiza mediciones de velocidad y distancia y las puede reportar inalámbricamente. Con todo esto podemos concluir que se alcanzó la meta de lograr un prototipo funcional superando varios desafíos para la cinemática y mecánica del robot.

\newpage
\section{Iteración 2: Prototipo mejorado}

\subsection{Introducción}
En el marco de la iteración anterior, se desarrolló un prototipo funcional que permitió validar la viabilidad técnica del sistema propuesto y sentar las bases para su evolución. Este trabajo destacó por identificar áreas clave de mejora que podrían optimizar su desempeño. Con base en estos hallazgos, la presente iteración se enfoca en mejorar el prototipo inicial, explorando la implementación de compensaciones en el sistema de control. Este enfoque tiene como objetivo principal incrementar la precisión, robustez y eficiencia del sistema, marcando un hito significativo en su desarrollo.

\subsection{Requerimientos}
En esta iteración abordaremos los siguientes requerimientos funcionales:

\begin{center} \begin{tabular}{|c|c|}
\hline
    ID & Descripción \\
\hline
    RF4 & El robot debe poder realizar trayectorias en línea recta y curvas. \\ 
\hline
    RF5 & El robot debe poder corregir su trayectoria mediante el uso de sensores. \\ 
\hline
\end{tabular} \end{center}

Por otra parte, los requerimientos no funcionales que trataremos son:

\begin{center} \begin{tabular}{|p{0.10\linewidth}|p{0.65\linewidth}|}
\hline
    ID & Descripción \\
\hline
    RNF1 & Debería tener tiempos de respuesta aceptables para el buen funcionamiento del sistema de control. \\
\hline
    RNF2 & El software debería contar con pruebas unitarias y de integración. \\
\hline
    RNF4 & El código debería contar con documentación.\\
\hline
\end{tabular} \end{center}

\subsection{Desarrollo}

\subsubsection{Compensación del modelo cinemático}

Dado el potencial del modelo cinemático, procederemos a compensar el sistema mediante la información provista por éste.

\paragraph{Compensación de movimiento lineal} \mbox{} \vspace{6pt}

En un caso real, existen variaciones que pueden provocar desviaciones en la ruta planeada debido a la textura o inclinación del suelo, imprecisiones o ruido en los sensores y el desgaste de las ruedas, así como otros factores mecánicos que pueden alterar la respuesta del robot a las órdenes de control.

Para abordar estos desafíos, se implementa un procedimiento de control que ajusta continuamente las velocidades de las ruedas en función de la retroalimentación recibida de los sensores y del modelo cinemático del robot. Se utilizan sensores para medir la posición y orientación actual del robot en el plano, inferida a partir de la odometría y las velocidades de las ruedas, para comparar estos datos con la trayectoria deseada y estimar los errores de posición y orientación. El modelo cinemático nos ayuda a  comprender cómo las velocidades de las ruedas individuales afectan el movimiento general del robot a través de ecuaciones matriciales que describen el comportamiento cinemático del robot.

Basado en los errores estimados y el modelo cinemático, se calculan las correcciones necesarias para las velocidades de las ruedas y se envían a los actuadores de las ruedas, logrando que el robot ajuste su movimiento de manera inmediata. \cite{rijalusalamkinematics} El proceso de medición, estimación, cálculo y aplicación de correcciones se repite continuamente a una frecuencia de $500[ms]$, equivalentes a 5 períodos de medición de RPM de cada rueda.

\begin{figure}[H]
    \centering
    \includegraphics[width=0.9\linewidth]{images/diag_compensacion_modelo_cinem.png}
    \caption{Diagrama del Modelo Cinemático compensado}
    \label{fig:diagramamodelocinemcompensado}
\end{figure}

Los experimentos se realizaron en un entorno consistente con la iteración anterior y los resultados muestran una reducción significativa en los errores de trayectoria, demostrando la robustez del enfoque propuesto, de modo que el robot puede seguir trayectorias con desviaciones mínimas.

\paragraph{Compensación de movimiento rotacional} \mbox{} \vspace{6pt}

El enfoque propuesto es el de tener un acumulador basado en la odometría de $\theta$ que mide la distancia recorrida angularmente. El principio de funcionamiento es tal que aplica ajustes en la velocidad rotacional para compensar el desplazamiento, de modo que el robot intente llevar la distancia rotacional a cero nuevamente. Por ejemplo, si el robot detecta un desplazamiento hacia la derecha, se aumenta la velocidad rotacional hacia la izquierda para compensar el desfase en la cantidad determinada por la distancia rotacional. De este modo es posible corregir la orientación de movimiento.

\begin{figure}[H]
    \centering
    \includegraphics[width=1\linewidth]{images/diag_secuencia_modelo_cinematico_compensado.png}
    \caption{Diagrama de secuencia del Modelo Cinemático compensado}
    \label{fig:diagsecuenciamodcinemcompens}
\end{figure}


\subsubsection{Seguidor de línea}

A pesar de ser una herramienta fundamental para estimar la posición y orientación del robot, la odometría está sujeta a errores. Por lo que se hace necesario contar con un mecanismo adicional para poder detectar desviaciones en la trayectoria. Para esto se propone un seguidor de línea colocado de modo que el robot pueda detectar cuando se sale por fuera de las trayectorias determinadas por las líneas.

Un seguidor de línea es un sistema basado en sensores que detectan marcas en el entorno y utilizan esta información para ajustar la trayectoria del robot de manera automática.

La implementación de un seguidor de línea en robots omnidireccionales ofrece varias ventajas. En primer lugar, permite corregir desviaciones causadas por errores acumulativos en la odometría, ya que los sensores del seguidor de línea proporcionan información directa sobre la posición del robot respecto a la línea de referencia y en tiempo real.


\paragraph{Elección de método de seguimiento} \mbox{} \vspace{8pt}

Existen distintas técnicas para detectar una linea que un robot puede seguir. \\

\textbf{Seguidor de Línea Ultrasónico} \mbox{} \vspace{6pt} \\
Utiliza sensores ultrasónicos para detectar la presencia de una línea guía de la cual el robot debe mantenerse equidistante. Estos sensores emiten ondas sonoras de alta frecuencia y miden el tiempo que estas tardan en reflejarse al encontrar una superficie, lo que permite al robot seguir una trayectoria de manera precisa.

Entre sus ventajas, destaca su robustez, ya que no se ven afectados por factores como la suciedad o la iluminación, y su precisión, al detectar con alta exactitud la distancia hacia la línea guía. Sin embargo, estos sistemas presentan algunas desventajas, como la mayor complejidad y costo asociados con los sensores ultrasónicos, así como un alcance limitado en comparación con otros tipos de sensores. \cite{venkateshrfidultrasonic} \cite{aungultrasonic} \\


\textbf{Seguidor de Línea por Cámara} \mbox{} \vspace{6pt} \\
El seguidor de línea por cámara utiliza cámaras y algoritmos de procesamiento de imágenes para identificar y seguir líneas en el suelo de manera eficiente. Estas tecnologías permiten que el sistema sea altamente versátil, adaptándose a distintos tipos de líneas y patrones, al tiempo que proporcionan información adicional sobre el entorno que rodea al robot.

No obstante, esta solución presenta ciertos desafíos. Por un lado, requiere algoritmos avanzados de procesamiento de imágenes, lo que puede complicar su implementación. Por otro, su desempeño puede verse afectado por cambios en las condiciones de iluminación, lo que limita su eficacia en entornos variables. \cite{inianlinefollowcamera} \\


\textbf{Seguidor de Línea Ópticos} \mbox{} \vspace{6pt} \\
Los seguidores de línea ópticos son una de las implementaciones más comunes, ya que emplean sensores ópticos para detectar el contraste entre una línea dibujada en el suelo, generalmente negra, y el fondo, que suele ser blanco. Sensores como los fotodiodos o fototransistores envían señales al controlador del robot, permitiéndole ajustar su trayectoria para mantenerse sobre la línea. Entre las opciones disponibles, las versiones láser o infrarrojas son las más utilizadas.

Este sistema tiene como principales ventajas su simplicidad y bajo costo, ya que es relativamente sencillo de implementar, así como la amplia disponibilidad de estos dispositivos en el mercado. Sin embargo, también presenta algunas desventajas. Su rendimiento puede verse afectado por factores ambientales como la iluminación variable, el polvo, la suciedad y las superficies reflectantes, además de mostrar menor precisión en superficies no uniformes. Adicionalmente, las líneas pintadas en el suelo pueden desgastarse con el tiempo, requiriendo mantenimiento constante. \\


\textbf{Seguidor de Línea Magnético} \mbox{} \vspace{6pt} \\
El seguidor de línea magnético utiliza bandas magnéticas adheridas al suelo o a una pared, las cuales son detectadas por sensores magnéticos. Estos sensores identifican cambios en el campo magnético, permitiendo al robot seguir con precisión la trayectoria marcada por las bandas magnéticas.

Entre las ventajas de este sistema se encuentra su inmunidad al entorno, ya que no es afectado por factores como la suciedad, el polvo, la iluminación variable o las superficies reflectantes. Además, las bandas magnéticas destacan por su durabilidad, resistiendo condiciones ambientales adversas y requiriendo poco mantenimiento, mientras que ofrecen alta precisión en la navegación, especialmente en entornos industriales. Sin embargo, este sistema también presenta desventajas, como el mayor costo asociado a los sensores y bandas magnéticas en comparación con sistemas ópticos, así como el tiempo y esfuerzo necesarios para instalar dichas bandas. \\


\textbf{Seguidor de Línea con Etiquetas RFID} \mbox{} \vspace{6pt} \\
El seguidor de línea con etiquetas RFID utiliza etiquetas RFID incrustadas en el suelo o las paredes, junto con lectores RFID instalados en el robot, para detectar la presencia y posición de dichas etiquetas. Este sistema permite al robot interpretar su entorno de forma más inteligente al acceder a la información adicional almacenada en las etiquetas.

Entre las ventajas de este enfoque se encuentran la capacidad de las etiquetas RFID para proporcionar información adicional que facilita una navegación más eficiente, su inmunidad a factores ambientales como suciedad, polvo, variaciones en la iluminación o superficies reflectantes, y su flexibilidad para diseñar trayectorias más complejas y adaptativas. Sin embargo, presenta algunas desventajas, como el alto costo asociado a los sistemas RFID debido al precio tanto de los lectores como de las etiquetas, así como la mayor complejidad requerida para su implementación y configuración en comparación con otros sistemas. \cite{venkateshrfidultrasonic} \\


Al evaluar las diferentes alternativas, optamos por un seguidor de línea magnético dado que es el que mejor se adapta a los requerimientos de funcionamiento del robot. Una de las principales ventajas es la inmunidad a las condiciones del entorno y su relativo bajo costo. No se ven afectados por elementos que pueden interferir con los sensores ópticos, como la suciedad, el polvo, la iluminación variable, superficies reflectantes o las sombras. Esta característica aumenta la robustez del sistema en entornos industriales o comerciales donde las condiciones pueden ser menos que ideales.

Otra ventaja importante es la durabilidad y el mantenimiento reducido. Las bandas magnéticas son resistentes al desgaste y pueden soportar condiciones ambientales variables, reduciendo la necesidad de mantenimiento frecuente y aumenta la vida útil del sistema. En comparación, en caso de un sensor óptico, las líneas pintadas o adhesivas pueden desgastarse con el tiempo y requerir repintado o reemplazo periódico. Además, se pueden colocar de manera discreta dado que no es necesario que sean visibles.


\paragraph{Implementación} \mbox{} \vspace{8pt}

Se hizo una pieza impresa en 3D y se colocaron los sensores una determinada distancia entre sí obtenida experimentalmente. El arreglo de sensores Hall se coloca en el frente del robot y se mejoro la pista adhiriendo los imanes.

El procedimiento para compensar el robot se basa en que al tocar un imán de los extremos significa que existe una determinada distancia angular que difiere del vector de dirección deseado. Se representa esta distancia mediante $d_{r\theta}$:

\begin{figure}[H]
    \centering
    \includegraphics[width=0.4\linewidth]{images/robot_desplazamiento_angular_toca_iman.png}
    \caption{Escenario de detección de un imán}
    \label{fig:deteccionimanrobot}
\end{figure}

El enfoque propuesto es el de tener un acumulador que mide la distancia recorrida angularmente y ajusta la velocidad rotacional para compensar el desfase. Cada vez que se detecta un imán se suma o se resta un determinado valor de distancia angular de modo que el controlador intente llevarla a 0 nuevamente. Por ejemplo, si el robot detecta un desplazamiento hacia la derecha, se aumenta la velocidad rotacional hacia la izquierda.

El algoritmo de control opera de la siguiente manera:

\begin{enumerate}
    \item Los sensores magnéticos son monitoreados continuamente para detectar la presencia de imanes. Cada vez que se detecta un imán en alguno de los extremos, se actualiza un acumulador de distancia angular recorrida sumando o restando un valor determinado. Este valor depende de la distancia de los sensores al centro del robot y de la distancia entre los mismos.

    \item El objetivo es que el acumulador se mantenga en cero, lo que indicaría que el robot sigue la dirección de la trayectoria. El sistema de control ajusta las velocidades de las ruedas del robot en tiempo real en función al desplazamiento angular detectado.    
\end{enumerate}

Al realizar las pruebas, se lanzó al robot en distintas circunstancias, comenzando alineado con los imanes y comenzando con una desviación detectable. A medida que se avanzaron los ensayos se fue iterando sobre los coeficientes del seguidor de línea para compensar el error.

En una primera instancia, la implementación de la corrección del desplazamiento angular estaba dado por una velocidad fija. En las sucesivas experiencias, notamos que el robot a bajas velocidades se comportaba de un modo esperado y que a medida que se incrementaba la velocidad era menos estable. Es por ello que se cambió la velocidad constante por una que sea variable según la velocidad lineal del robot, es decir, ahora la compensación es adaptativa. De modo que a mayor velocidad lineal, mayor velocidad rotacional para realizar la compensación. Esto se debe a la inercia del robot, implicando que a mayor velocidad requiere mayor compensación para vencer el momento del mismo.

En la Figura \ref{fig:diagsecuencialinefollowmodcin} se detalla un diagrama de secuencia del funcionamiento del modelo cinemático compensado con el seguidor de línea.

\begin{figure}[H]
    \centering
    \hspace*{-1.25cm}
    \includegraphics[width=1.2\linewidth]{images/diag_secuencia_seguidor_linea_magnetica_modelo_cinem_compensado.png}
    \caption{Diagrama de secuencia del seguidor de linea con el Modelo Cinemático}
    \label{fig:diagsecuencialinefollowmodcin}
\end{figure}



\subsection{Testing y pruebas}

Las pruebas realizadas en esta iteración parten del prototipo logrado en la iteración anterior.

\begin{testtableformat}
    \hline \rowcolor{test_header_color}
        Test ID             & TC\_02\_00 \\
    \hline
        Tipo de test        & Test de integración \\
    \hline
        Objeto de prueba    & Modelo Cinemático \\
    \hline
        Requerimiento       & RF4 - RF5 \\
    \hline
        Nombre              & Modelo Cinemático con compensación en línea recta \\
    \hline
        Descripción         & Comprobar que el Modelo Cinemático compensa adecuadamente las velocidades de las ruedas según un setpoint en línea recta \\
    \hline
        Precondición        & PRECOND\_B \\
    \hline
        Pasos del test      & \begin{enumerate}
                                \item Enviar al robot un setpoint con distancia entre [50cm $\sim$ 400cm] y velocidad lineal entre $\pm$[0.25m/seg $\sim$ 0.75m/seg]
                                \item Verificar que el robot recorre el vector dado a lo largo de la distancia determinada y que su movimiento es compensado
                                \item Repetir desde el paso 1) con diferentes valores
                            \end{enumerate} \\
    \hline
        Resultado esperado  & El robot se mueve en línea recta en la dirección del vector dado por $V_x$ y $V_y$. Debe ser notable la mejora en la estabilidad del vector a realizar. \\
    \hline
        Resultado obtenido  & Se observa que el robot mejora sustancialmente el desempeño al realizar la trayectoria. \\
    \hline
        Observaciones       & - \\
    \hline
\end{testtableformat}


\begin{testtableformat}
    \hline \rowcolor{test_header_color}
        Test ID             & TC\_02\_01 \\
    \hline
        Tipo de test        & Test de integración \\
    \hline
        Objeto de prueba    & Seguidor de línea magnética - Modelo Cinemático \\
    \hline
        Requerimiento       & RF5 \\
    \hline
        Nombre              & Compensación de linea magnética \\
    \hline
        Descripción         & Verificar que el robot compensa su trayectoria al comenzar centrado en la línea de imanes \\
    \hline
        Precondición        & PRECOND\_C \\
    \hline
        Pasos del test      & \begin{enumerate}
                                \item Colocar al robot centrado respecto a la línea de imanes
                                \item Enviar al robot un setpoint con distancia entre [50cm $\sim$ 400cm] y velocidad lineal entre $\pm$[0.25m/seg $\sim$ 0.75m/seg]
                                \item Verificar que el robot recorre el vector dado a lo largo de la distancia determinada, que el Modelo Cinemático compensa el movimiento y que al detectar un imán en un lateral se corrige la orientación del robot
                                \item Repetir desde el paso 1) con diferentes valores
                            \end{enumerate} \\
    \hline
        Resultado esperado  & El robot se mantiene dentro de los límites de la línea magnética \\
    \hline
        Resultado obtenido  & El robot logra mantenerse centrado con la línea de imanes. Se observa en varias ocasiones que el robot toca un imán, a lo que el robot aplica una velocidad rotacional contraria para centrarlo \\
    \hline
        Observaciones       & - \\
    \hline
\end{testtableformat}


\begin{testtableformat}
    \hline \rowcolor{test_header_color}
        Test ID             & TC\_02\_02 \\
    \hline
        Tipo de test        & Test de integración \\
    \hline
        Objeto de prueba    & Seguidor de línea magnética  - Modelo Cinemático \\
    \hline
        Requerimiento       & RF5 \\
    \hline
        Nombre              & Compensación de linea magnética \\
    \hline
        Descripción         & Verificar que el robot compensa su trayectoria al comenzar tocando un imán \\
    \hline
        Precondición        & PRECOND\_C \\
    \hline
        Pasos del test      & \begin{enumerate}
                                \item Colocar al robot descentrado respecto a la línea de imanes y de modo que se detecte un imán al inicio
                                \item Enviar al robot un setpoint con distancia entre [50cm $\sim$ 400cm] y velocidad lineal entre $\pm$[0.25m/seg $\sim$ 0.75m/seg]
                                \item Comprobar que al iniciar corrige su trayectoria de inmediato y que luego el robot se mantiene centrado a lo largo de la línea magnética, compensando con el Modelo Cinemático y al detectar imanes
                                \item Repetir desde el paso 1) con diferentes valores
                            \end{enumerate} \\
    \hline
        Resultado esperado  & El robot corrige el desfase y vuelve a posicionarse dentro de los límites de la línea \\
    \hline
        Resultado obtenido  & El robot logra mantenerse centrado con la línea. Se observa que al tocar un imán se aplica una velocidad rotacional contraria para centrarlo \\
    \hline
        Observaciones       & - \\
    \hline
\end{testtableformat}


\begin{testtableformat}
    \hline \rowcolor{test_header_color}
        Test ID             & TC\_02\_03 \\
    \hline
        Tipo de test        & Test de sistema \\
    \hline
        Objeto de prueba    & Comunicación inalámbrica - PID - Modelo cinemático compensado - Odometría - Seguidor de línea magnética \\
    \hline
        Requerimiento       & RF1 - RF2 - RF3 - RF4 - RF5 - RF6 \\
    \hline
        Nombre              & Prueba de sistema integrado \\
    \hline
        Descripción         & Comprobar que el robot realiza trayectorias en una dirección y longitud determinadas \\
    \hline
        Precondición        & PRECOND\_C \\
    \hline
        Pasos del test      & \begin{enumerate}
                                \item Enviar al robot un setpoint con distancia entre [50cm $\sim$ 400cm] y velocidad lineal entre $\pm$[0.25m/seg $\sim$ 0.75m/seg]
                                \item Comprobar que el robot se mantiene centrado a lo largo de la línea magnética y que es compensado por el Modelo Cinemático. Además debe reportar mediciones de velocidad y distancia
                                \item Repetir desde el paso 1) con diferentes valores
                            \end{enumerate} \\
    \hline
        Resultado esperado  & El robot responde correctamente al vector y la distancia establecida, reporta información de mediciones de velocidad y distancia \\
    \hline
        Resultado obtenido  & El robot realiza las trayectorias de manera acorde dentro de los límites observados en las pruebas unitarias y de integración. Se logra recolectar la información enviada por el robot \\
    \hline
        Observaciones       & Se probó hasta recorridos de 4 metros por limitaciones de espacio. \\
    \hline
\end{testtableformat}

\subsection{Resultados}

En el presente proyecto, se lograron avances significativos respecto al prototipo funcional desarrollado en la etapa anterior. Uno de los principales logros fue la implementación exitosa de compensaciones en varios niveles del sistema de control, lo cual permitió optimizar el rendimiento global del sistema. Estas compensaciones del modelo cinemático y de un seguidor de línea contribuyeron a mejorar la precisión y la estabilidad en condiciones de operación variables, reduciendo de manera notable los márgenes de error detectados previamente.

Asimismo, se llevaron a cabo pruebas exhaustivas que validaron la robustez del sistema mejorado, evidenciando un incremento en su capacidad para adaptarse a distintos escenarios operativos.

\begin{figure}[H]
    \centering
    \includegraphics[trim={0 2cm 0 7.5cm}, clip, width=1.1\linewidth]{images/prototipo_robot_con_sens_mag.png}
    \caption{Prototipo del robot con seguidor de linea}
    \label{fig:prototiporobotlinef}
\end{figure}

\subsection{Riesgos superados}

Dado el desarrollo de esta iteración, se logro avanzar sobre el riesgo RI-03 y RI-01 por demostrar que los componentes elegidos aun son viables para escalar el proyecto pero existe lugar para mayor aprovechamiento de los mismos.

Por otra parte, se trabajó en superar el riesgo RI-02 por lograr una comunicación efectiva entre los componentes del sistema y se supera en parte el riesgo RI-05 al ser los nuevos componentes agregados de alta disponibilidad.

\begin{center} \begin{tabular}{|c|c|}
    \hline
        ID & Riesgo \\
    \hline
        RI-01 & Incompatibilidad o avería de componentes \\
    \hline
        RI-02 & Intercomunicación de componentes ineficiente o ineficaz \\
    \hline
        RI-03 & Prestaciones insuficientes de componentes \\
    \hline
        RI-05 & Dificultad en conseguir determinados componentes. \\
    \hline
\end{tabular} \end{center}

\subsection{Conclusiones}

Con la realización de esta iteración, obtuvimos que la implementación de compensaciones en el sistema de control marcó un hito al optimizar aspectos clave como la precisión, la estabilidad y la adaptabilidad del sistema frente a distintas condiciones de operación. Estos logros no solo demuestran la efectividad de las estrategias propuestas, sino que también refuerzan la importancia de un enfoque continuo hacia la mejora y la innovación del proyecto.


\newpage
\section{Iteración 3: Modelo del mapa e interfaz}

\subsection{Introducción}
Durante el desarrollo de la iteración anterior se logró construir y optimizar el robot, integrando compensaciones en el sistema de control para mejorar su estabilidad y precisión operativa.

En la presente iteración se busca avanzar aún más en las capacidades del robot mediante el modelado del espacio en el que operará, esto mediante la creación de una representación del mapa de su entorno.

Además, se buscará desarrollar una interfaz de usuario que facilitará la interacción y monitoreo del sistema.

\subsection{Requerimientos}
En esta iteración abordaremos los siguientes requerimientos funcionales:

\begin{center} \begin{tabular}{|c|c|}
\hline
    ID & Descripción \\
\hline
    RF7 & Debe existir un modo de calcular trayectorias automáticamente. \\
\hline
    RF10 & Debe existir una interfaz de usuario para control y monitoreo. \\
\hline
\end{tabular} \end{center}

Por otra parte, los requerimientos no funcionales que trataremos son:

\begin{center} \begin{tabular}{|p{0.10\linewidth}|p{0.65\linewidth}|}
\hline
    ID & Descripción \\
\hline
    RNF1 & Debería tener tiempos de respuesta aceptables para el buen funcionamiento del sistema de control. \\
\hline
    RNF2 & El software debería contar con pruebas unitarias y de integración. \\
\hline
    RNF4 & El código debería contar con documentación.\\
\hline
\end{tabular} \end{center}

\subsection{Desarrollo}

\subsubsection{Modelado del mapa}

Para empezar, es crucial tener un diseño inicial del entorno donde se moverá el robot, que incluya paredes, obstáculos y puntos de interés. Dado que el robot solo se mueve en líneas rectas, este proceso puede ser más sencillo en comparación con robots que tienen libertad de movimiento en todas las direcciones.

El siguiente paso fue la creación del modelo del mapa, donde se divide el entorno en una cuadrícula (grid) y cada celda se marca como libre, ocupada, obstáculo o borde. Cada celda es cuadrada y tiene un lado de $0.5[m]$. Además de ello, el control de movimiento del robot debe estar alineado con los ejes de la grid, y se utilizan algoritmos como A* o Dijkstra para planificar rutas en línea recta.

Mientras el robot se mueve y los sensores recopilan datos, las celdas de la grid se deben actualizar continuamente, indicando si están libres u ocupadas. Esta actualización constante del mapa es esencial para mantener la precisión y eficiencia en el mapeo. Al mismo tiempo, es importante validar que el mapa generado es preciso y coincide con el entorno real.

Debajo se muestra el modelo del mapa que consideramos para el proyecto. Decidimos que los colores de referencia son:

\begin{itemize}
    \item Rojo: borde o límite del mapa
    \item Blanco: obstáculo
    \item Gris: celda ocupable libre
    \item Otro: robot ocupando una celda
\end{itemize}

El punto verde en la imagen de debajo simboliza el origen efectivo del mapa, es la primera celda ocupable del mapa. Además en el centro existe un obstáculo que torna a esa celda no ocupable. Las coordenadas representadas en el mapa son coordenadas ordinales de las celdas, al realizar la odometría se toma el valor de medición en metros. Por lo que si el robot está en la celda $(3, 2)$, la posición reportada por la odometría será $(1.5, 1.0)$, dado que cada celda tiene $0.5[m]$ de lado.

\begin{figure}[H]
    \centering
    \includegraphics[width=0.8\linewidth]{images/modelo_del_mapa.png}
    \caption{Modelo del mapa}
    \label{fig:modelomapa}
\end{figure}


\subsubsection{Path Finder}

Dentro de los algoritmos de planificación de trayectorias, elegimos utilizar A* por su gran eficiencia en encontrar el camino mas corto y ser ampliamente conocido y estudiada su efectividad. Su objetivo es encontrar el camino más corto entre dos puntos en un mapa combinando las ventajas de los algoritmos de búsqueda de coste uniforme y heurística. \cite{sariffpathplan} \cite{cuevaspathfinding}

$A*$ funciona evaluando cada celda del mapa utilizando una función de costo:

$$ f(n) = g(n) + h(n) $$

Donde $g(n)$ es el costo acumulado desde el punto inicial hasta la celda actual, y $h(n)$ es una heurística que estima el costo restante desde la celda actual hasta el destino. La heurística debe ser admisible, lo que significa que nunca sobreestima el costo real para garantizar que el algoritmo encuentre el camino más corto.

El proceso comienza con la celda inicial, que se agrega a una lista abierta (open list) de celdas por explorar. En cada paso, el algoritmo selecciona la celda con el valor $f(n)$ más bajo de la lista abierta y la mueve a una lista cerrada (closed list) de celdas ya exploradas. Luego, evalúa las celdas vecinas de la celda actual. Si una celda vecina no está en la lista cerrada y no hay una ruta mejor a esa celda en la lista abierta, se actualizan sus valores de $g$, $h$ y $f$, y se agrega a la lista abierta.

Este proceso se repite, seleccionando y evaluando celdas hasta que se alcanza la celda destino. A medida que se avanza, $A*$ construye el camino más corto de regreso desde el destino hasta el origen siguiendo los valores de $g$. La clave del éxito del algoritmo $A*$ es su capacidad para equilibrar de manera eficiente el costo acumulado $g(n)$ y la heurística $h(n)$, permitiéndole encontrar rutas óptimas de manera efectiva.

La eficiencia de $A*$ depende en gran medida de la heurística utilizada. La heurística más común es la distancia de Manhattan para movimientos en una cuadrícula ortogonal, o también la distancia euclidiana para movimientos en cualquier dirección. En nuestro caso la técnica mas conveniente es utilizar la distancia de Manhattan, también conocida como distancia L1 o distancia de taxista, que esta dada por la suma de las diferencias absolutas entre las coordenadas de dos puntos en un espacio multidimensional. Es equivalente a calcular la distancia viajando en línea recta, primero en una dirección ($x$) y luego en la otra ($y$), como si se caminara por las calles de Manhattan que tiene un patrón cuadriculado.



\subsubsection{Interfaz de usuario}

En la pantalla principal de la interfaz se muestra un mapa interactivo del entorno, con una cuadrícula que representa el mapa. La interfaz muestra información en tiempo real sobre la celda actual en la que se ubica el robot en el mapa, actualizándose conforme el robot se desplaza. Además, se proporciona información sobre la dirección de movimiento.

El usuario comienza estableciendo los puntos de inicio y destino haciendo click en las celdas disponibles. Cuando se da la orden de inicio, se calcula la ruta óptima y se envían los comandos al robot. Paso seguido, la interfaz muestra continuamente en tiempo real la celda del robot y se envían los comandos de movimiento hacia las celdas subsiguientes.

\begin{figure}[H]
    \centering
    \includegraphics[width=0.85\linewidth]{images/interfaz_de_usuario}
    \caption{Interfaz de usuario}
    \label{fig:interfazusuario}
\end{figure}



\subsubsection{Diseño de la pista}

El diseño de la pista debe tener en cuenta las limitaciones impuestas a movimientos que puede y no puede hacer el robot en el plano. Dado que tendrá montada una cámara fija y ésta deberá tomar imágenes que son perpendiculares a la dirección de movimiento, es por lo que se establece una restricción al movimiento del robot sobre el plano. Aunque el robot tiene la capacidad de realizar movimientos rectos en cualquier ángulo e incluso hacer trayectorias curvas, se ha decidido que solo realizará trayectorias en línea recta a 90° y que puede rotar sobre su eje, pero no realizar trayectorias curvas.

Se procede a diseñar la pista dibujando un plano detallado del recorrido y la ubicación de los imanes considerando la distancia entre ellos.

\begin{figure}[H]
    \centering
    \includegraphics[width=0.5\linewidth]{images/pista_diseño_full.png}
    \caption{Diseño de la pista}
    \label{fig:disenopista}
\end{figure}

Procedimos a realizar un corte láser sobre madera MDF de 5mm. Se instalaron los imanes en la pista de modo que están correctamente alineados todos en la misma polaridad para que el robot pueda seguir la línea magnética sin problemas.

Finalmente, se realizó la calibración iterativa del robot en la pista diseñada. Se ajustan los sensores magnéticos del robot y se realizan múltiples pruebas en diferentes condiciones para evaluar su desempeño y hacer los ajustes necesarios.


\subsection{Testing y pruebas}

\begin{testtableformat}
    \hline \rowcolor{test_header_color}
        Test ID             & TC\_03\_00 \\
    \hline
        Tipo de test        & Test unitario \\
    \hline
        Objeto de prueba    & Calculador de trayectorias \\
    \hline
        Requerimiento       & RF7 \\
    \hline
        Nombre              & Cálculo de trayectorias en un mapa con obstáculos, trayectoria resoluble \\
    \hline
        Descripción         & Calcular la secuencia de coordenadas a seguir para llegar a un punto A a un punto B del mapa, con obstáculos dispuestos de modo que existe al menos una trayectoria posible \\
    \hline
        Precondición        & PRECOND\_E \\
    \hline
        Pasos del test      & \begin{enumerate}
                                \item Enviar al PathFinder una coordenada origen y una destino de modo que exista un camino soluble entre ellas
                                \item Verificar que la salida del PathFinder consiste en la secuencia de coordenadas del mapa que se deben visitar para llegar del origen al destino
                                \item Repetir desde el paso 1) con diferentes valores
                            \end{enumerate} \\
    \hline
        Resultado esperado  & Se obtiene la secuencia adecuada para los puntos dados y se cumple la restricción de solo movimientos horizontales o verticales \\
    \hline
        Resultado obtenido  & La secuencia de coordenadas calculadas se corresponden a las más óptimas para cada par de puntos y se cumple la restricción de movimiento \\
    \hline
        Observaciones       & - \\
    \hline
\end{testtableformat}


\begin{testtableformat}
    \hline \rowcolor{test_header_color}
        Test ID             & TC\_03\_01 \\
    \hline
        Tipo de test        & Test unitario \\
    \hline
        Objeto de prueba    & Calculador de trayectorias \\
    \hline
        Requerimiento       & RF7 \\
    \hline
        Nombre              & Cálculo de trayectorias en un mapa con obstáculos, trayectoria irresoluble \\
    \hline
        Descripción         & Calcular la secuencia de coordenadas a seguir para llegar a un punto A a un punto B del mapa, con obstáculos dispuestos de modo que no existe ninguna trayectoria posible \\
    \hline
        Precondición        & PRECOND\_F \\
    \hline
        Pasos del test      & \begin{enumerate}
                                \item Enviar al PathFinder una coordenada origen y una destino de modo que no exista camino posible entre ellas
                                \item Verificar que la salida del PathFinder es un error
                                \item Repetir desde el paso 1) con diferentes valores
                            \end{enumerate} \\
    \hline
        Resultado esperado  & Se obtiene un mensaje de error informando que no existe trayectoria posible \\
    \hline
        Resultado obtenido  & El error se presenta solo cuando se define el mapa de modo que entre un punto A y un punto B no existe camino posible \\
    \hline
        Observaciones       & - \\
    \hline
\end{testtableformat}


\begin{testtableformat}
    \hline \rowcolor{test_header_color}
        Test ID             & TC\_03\_02 \\
    \hline
        Tipo de test        & Test de integración \\
    \hline
        Objeto de prueba    & Interfaz de usuario y calculador de trayectorias \\
    \hline
        Requerimiento       & RF7 - RF10 \\
    \hline
        Nombre              & Interacción con la interfaz y cálculo de trayectorias con obstáculos \\
    \hline
        Descripción         & Verificar que al interactuar con la interfaz es posible establecer puntos de origen y de destino, para luego calcular la secuencia de coordenadas a seguir en base a esos puntos \\
    \hline
        Precondición        & PRECOND\_E \\
    \hline
        Pasos del test      & \begin{enumerate}
                                \item Establecer un punto de origen y un punto de destino en el mapa de la interfaz
                                \item Dar la orden de calcular la secuencia de coordenadas y verificar que son validas
                                \item Repetir desde el paso 1) para otra combinación de coordenadas origen y destino
                            \end{enumerate} \\
    \hline
        Resultado esperado  & En todos los casos se obtiene una lista con las coordenadas a recorrer desde el punto de origen hasta el final \\
    \hline
        Resultado obtenido  & La lista de coordenadas obtenidas representa las coordenadas del mapa por las cuales se debe pasar para llegar a destino \\
    \hline
        Observaciones       & - \\
    \hline
\end{testtableformat}


\begin{testtableformat}
    \hline \rowcolor{test_header_color}
        Test ID             & TC\_03\_03 \\
    \hline
        Tipo de test        & Test de sistema \\
    \hline
        Objeto de prueba    & Comunicación inalámbrica - PID - Modelo cinemático compensado - Odometría - Seguidor de línea magnética - Modelo del Mapa - Calculador de trayectorias - Interfaz de usuario \\
    \hline
        Requerimiento       & RF1 - RF2 - RF3 - RF4 - RF5 - RF6 - RF7 - RF10 \\
    \hline
        Nombre              & Prueba de sistema integrado \\
    \hline
        Descripción         & Comprobar que el robot realiza trayectorias en una dirección y longitud determinadas \\
    \hline
        Precondición        & PRECOND\_C - PRECOND\_E \\
    \hline
        Pasos del test      & \begin{enumerate}
                                \item Establecer en la interfaz un par de puntos origen y destino y calcular la trayectoria
                                \item Enviar al robot un setpoint por cada coordenada a recorrer
                                \item Repetir desde el paso 1) con diferentes valores
                            \end{enumerate} \\
    \hline
        Resultado esperado  &  El robot se mantiene centrado a lo largo de la línea magnética y que es compensado por el Modelo Cinemático. Además, la interfaz debe mostrar en tiempo real la celda que ocupa el robot y debe reportar mediciones de velocidad y distancia \\
    \hline
        Resultado obtenido  & El robot realiza las trayectorias de manera acorde dentro de los límites observados en las pruebas unitarias y de integración. Se logra recolectar la información enviada por el robot y se visualiza la posición del robot en la interfaz \\
    \hline
        Observaciones       & - \\
    \hline
\end{testtableformat}

\subsection{Resultados}
Uno de los principales resultados fue el desarrollo de una interfaz de usuario simple que facilita el control y la supervisión del robot. Aparejado con ello, se introdujo un calculador de trayectorias que optimiza los movimientos del robot en el espacio modelado.

Por último se logró implementar una pista fabricada con madera MDF, con imanes integrados, la cual garantiza estabilidad y uniformidad en las pruebas del sistema. Esta pista, además, fue modelada y representada tanto en el mapa desarrollado como en la interfaz, manteniendo la coherencia del entorno del robot.


\begin{figure}[H]
    \centering
    \includegraphics[width=1\linewidth]{images/robot_en_pista_final.png}
    \caption{Robot en la pista con imanes}
    \label{fig:robotpistaconimanes}
\end{figure}

\subsection{Riesgos superados}
En esta iteración pudimos demostrar que los nuevos componentes integrados al sistema interactúan efectivamente con los ya desarrollados en las iteraciones previas, por lo que se avanza sobre el riesgo RI-02.

Del mismo modo, se supera en parte el riesgo RI-05 por tener una alta disponibilidad de servicios de corte láser en MDF y los imanes son muy accesibles.

\begin{center}
\begin{tabular}{|c|c|} 
    \hline
        ID & Riesgo \\
    \hline
        RI-02 & Intercomunicación de componentes ineficiente o ineficaz \\
    \hline
        RI-05 & Dificultad en conseguir determinados componentes. \\
    \hline
\end{tabular}
\end{center}

\subsection{Conclusiones}

En esta iteración del proyecto se han alcanzado avances fundamentales, como la inclusión de un calculador de trayectorias que optimiza los movimientos del robot dentro de la pista, marcando un avance fundamental para su rendimiento.

Asimismo, se avanzó con la creación de una interfaz de usuario funcional y sencilla permitiendo un control y monitoreo más efectivo del sistema. 

Por su parte, la implementación de una pista de superficie consistente que representa el espacio de movimiento del robot ha proporcionado un entorno operacional estable, ideal para las pruebas del robot.

En definitiva, el proyecto avanza hacia su consolidación como una solución robusta, eficiente y adaptable.

\newpage
\section{Iteración 4: Red de Petri y monitor}

\subsection{Introducción}
El control de navegación de un robot omnidireccional es un desafío complejo que requiere la coordinación de múltiples tareas, como la planificación de trayectorias, el eludir obstáculos y la ejecución de movimientos precisos. Las redes de Petri son una herramienta matemática y gráfica que permite modelar, analizar y controlar sistemas discretos y concurrentes, lo que las hace particularmente útiles para gestionar la navegación de robots en entornos dinámicos. A continuación se establece la relación entre el control de navegación y las redes de Petri.

\subsection{Requerimientos}

En esta iteración abordaremos los siguientes requerimientos funcionales:

\begin{center}
    \begin{tabular} {
        | >{\centering\arraybackslash}m{1cm}
        | >{\centering\arraybackslash}m{13cm} |}
        \hline \rowcolor{test_header_color}
            ID & Descripción \\
        \hline
            RF4 & El robot debe poder realizar trayectorias en línea recta y curvas. \\
        \hline
            RF6 & El robot debe recibir y enviar información mediante comunicaciones inalámbricas. \\
        \hline
            RF8 & Debe poder ubicarse al robot en el plano de forma precisa. \\
        \hline
    \end{tabular}
\end{center}

   Por otra parte, el requerimiento no funcional que abordaremos es:

\begin{center}
    \begin{tabular} {
        | >{\centering\arraybackslash}m{1cm}
        | >{\centering\arraybackslash}m{13cm} |}
        \hline \rowcolor{test_header_color}
            ID & Descripción \\
        \hline
            RNF1 & Debería tener tiempos de respuesta aceptables para el buen funcionamiento del sistema de control. \\
        \hline
    \end{tabular}
\end{center}

\subsection{Desarrollo}

\subsubsection{Modelo del mapa con Red de Petri}

Como este trabajo tiene por objetivo el control, seguimiento del comportamiento y modelado de un sistema compuesto por una flota de robots donde cada uno tiene que seguir una trayectoria definida y calculada previamente utilizando el algoritmo $A*$ determinando el camino más corto hacia su destino. Teniendo esto en mente es necesaria la implementación de un método para evitar los problemas inherentes en los sistemas multi-robot: las colisiones entre los robots y los bloqueos del sistema, haciendo que algunos robots no puedan terminar su trayectoria y por lo tanto entorpecer todo el sistema.

Al tener el conjunto de trayectorias definidas para cada robot, se modela el sistema y el mapa mediante el uso de Redes de Petri, que no son más que representaciones matemáticas con una representación gráfica de un sistema de eventos discretos. El entorno en el cual se mueven los robots se considera particionado el mapa en regiones (celdas) y cada región del mapa se modela como una plaza en la Red de Petri. Para evitar las colisiones entre los robots, se establecen regiones con capacidad finita (recursos), donde, no pueden pasar por ellas mas robots de los recursos definidos. \cite{mahulea2020multi}

Si más de un robot desea pasar por la misma región en donde ya se encuentra posicionado un robot, el segundo debe esperar que la celda se libere y por lo tanto es necesario añadir lugares de espera. Estos lugares pueden conducir a bloqueos en la Red de Petri, ocasionando que los robots no lleguen a su destino. Los bloqueos se pueden caracterizar en la Red de Petri utilizando algunos elementos estructurales denominados sifones. Controlando que estos elementos no se vacíen, la red no se bloquea.

\begin{figure}[H]
   \centering
   \includegraphics[trim={1.35cm 0 0.9cm 1.5cm}, clip, width=0.6\linewidth]{images/mapa_py.png}
   \caption{Representación de un mapa en Python}
   \label{fig:mapa_py}
\end{figure}

La capacidad de las regiones, es decir, el número de robots que pueden estar simultáneamente en un lugar, se modela mediante los lugares de recursos, a los cuales se les asigna una marca (recurso), cuando el robot entra en una región específica y que se libera cuando el robot deja esa región.

Como ya se ha comentado la aplicación inmediata de los algoritmos implementados para este trabajo es la creación de una plataforma para el control de un sistema multi-robot con el objetivo de evitar colisiones entre los robots móviles. Sin embargo, se puede aplicar a cualquier sistema que sea modelable mediante redes de Petri. Por ello este proyecto tiene trascendencia en todas las áreas del control mediante sistemas de eventos discretos.

Antes de la implementación de uno de estos procesos, es muy importante modelarlo con un sistema discreto, comprobar su funcionamiento y corregirlo si hace falta. Las redes de Petri son una herramienta ampliamente utilizada cuando se trata de sistemas concurrentes, y como en cualquier otro sistema con recursos compartidos pueden aparecer bloqueos, así que este proyecto puede servir para resolver estos bloqueos haciendo el correcto uso de los recursos compartidos y gestionando de forma adecuada su utilización.

Un paso a realizar para obtener el modelo discreto del sistema multi-robot es dividir el mapa en regiones, ya que cada región del mapa se modelará como un lugar en el modelo de la red de Petri o como un nodo en el modelo de un autómata finito determinista. Para resolver el problema mencionado de dividir el mapa en regiones, se utiliza el método de la \textit{Descomposición en celdas}. Este método consiste en dividir las zonas sin obstáculos del plano mediante un cuadrado. Esto se puede ver claramente en la imagen anterior donde las celdas en color gris son las regiones disponibles o habitables por el robot mientras que las celdas rojas representan los límites del mapa.

Una vez descompuesto el mapa en regiones, queda calcular las trayectorias de cada robot para saber porque regiones ha de pasar para alcanzar de la forma más óptima el objetivo prefijado. Las trayectorias son obtenidas de forma automática a partir de algoritmos de cálculo de trayectorias individualmente para cada uno de los robots ignorando al resto de ellos utilizando algoritmos de búsqueda de caminos mínimos en grafos, o también pueden calcularse mediante algoritmos de planificación multi-robot ignorando las posibles colisiones entre los mismos utilizando programación matemática y modelos de tipo redes de Petri. En nuestro caso optamos por utilizar el algoritmo de $A*$, el cual busca el camino más corto para el robot hacía su destino último.

Si los dos robots individualmente siguen sus propias trayectorias, estos pueden colisionar (plazas ocupadas) ya que sus trayectorias pasan por esas regiones que tienen capacidad unitaria (un solo robot puede haber dentro de ellas en cualquier momento). Para evitar las posibles colisiones, se introducen modos de espera, es decir, se impone que por estas regiones no pueda pasar más de un robot al mismo tiempo, de modo que el segundo en llegar a esas regiones deberá esperar a que el primero salga de esa área.

Para sostener este concepto de regiones con capacidad limitada a las redes de Petri, se introducen lugares adicionales, a los que se les llama lugares de capacidades o de recursos compartidos. Estos lugares contienen inicialmente tantas marcas como capacidad tenga la región, en este caso se asignaran recursos con una marca a los lugares que corresponden a regiones con capacidad unitaria. Esta marca se utiliza para disparar la transición de entrada al lugar con capacidad limitada, y cuando el robot abandona esa zona, la transición de salida de ese lugar libera la marca para que vuelva a ser utilizada por el otro robot. Utilizando este concepto para todos los lugares con capacidad finita, la red quedaría como la siguiente.

\begin{figure}[H]
   \centering
   \includegraphics[width=0.8\linewidth]{images/rdp_no_grid.png}
   \caption{Representación del mapa en Python en una red de Petri}
   \label{fig:rdp_no_grid}
\end{figure}

Cabe destacar, que conforme va evolucionando el sistema y los robots se van moviendo de una región a otra, el marcado de la red va cambiando hasta que finalmente cuando un robot termina su trayectoria, la marca vuelve al lugar de reposo, sin embargo, el modelo de  red de Petri no podría volver a ser usado para supervisar el movimiento hasta que el robot sea colocado de nuevo en la posición de inicio del modelo.

Esta simple idea lo que hace es generar modos de espera de modo que nunca puedan coincidir en una región con capacidad restringida más de un robot. De modo que escogiendo correctamente estas regiones en las zonas donde las trayectorias de los robots coinciden, se pueden evitar las colisiones entre robots. Esto es debido a que los recursos compartidos producen modos de espera.

\paragraph{Transformación de un mapa a una red de Petri} \mbox{} \vspace{8pt}

Dentro del programa en Python existe una clase que se encarga de realizar esta transformación, el proceso consiste en definir la estructura que va a tener el mapa, es decir, cuáles van a ser sus espacios habitables y cuáles los límites. Entendiendo cómo límites a las paredes por fuera de la superficie de desplazamiento (su perímetro), y a los obstáculos que pueden existir dentro del mapa (paredes interiores, columnas, objetos, etc). Esto se hace en un archivo de texto que va a ser leído e interpretado por el algoritmo para luego ser convertido en una red de Petri.

\begin{figure}[H]
   \centering
   \includegraphics[width=0.5\linewidth]{images/map_definition_matriz.png}
   \caption{Matriz que define el mapa y sus elementos para el programa en Python}
   \label{fig:map_definition}
\end{figure}

La figura \ref{fig:map_definition} muestra el contenido del archivo de texto definido como una matriz de elementos donde, los números $-1$ representan los límites del mapa, los números $1$ los obstáculos y por último los números $0$ los espacios habitables.

El contenido del archivo que se muestra en la imagen anterior es el interpretado por el algoritmo y transformado. De este archivo se obtienen dos definiciones fundamentales para representar de forma correcta y suficiente a la red de Petri. Por un lado la matriz de incidencia, la cual nos describe la relación que existe entre las plazas y las transiciones de la red, es decir, como va a ser el movimiento de los tokens a medida que se disparen las transiciones, y por el otro lado el marcado inicial de la red, es decir, la distribución de los tokens en la red cuando todavía no se disparó ninguna transición. Cómo nosotros buscamos que la red cumpla ciertas condiciones para su adecuado comportamiento (por ejemplo no ser bloqueante), añadimos un lugar de recurso y un lugar de ocupación para celda habitable del mapa, es por ello que la imagen \ref{fig:rdp_no_grid} luce de esa manera.

Todo el proceso de transformación se encuentra descrito en la siguiente imagen, desde la lectura del archivo donde se define la estructura del mapa, pasando por su conversión a las matrices de incidencia y marcado, y por último en cómo el monitor toma esa red de Petri para decidir sobre el avance de los robots en el mapa.

\begin{figure}[H]
   \centering
   \includegraphics[width=1.0\linewidth]{images/diagrama_clase_monitor.jpg}
   \caption{Diagrama de alto nivel de la transformación de un mapa a una red de Petri}
   \label{fig:mapa_to_rdp}
\end{figure}

\subsubsection{Concurrencia en Python} \mbox{} \vspace{5pt}

Python ofrece diversas formas de manejar la concurrencia, permitiendo que múltiples tareas se ejecuten de manera simultánea. Las principales herramientas para lograrlo incluyen hilos, procesos y asincronía. Sin embargo, la concurrencia en Python está influenciada y controlada por el Global Interpreter Lock (GIL), el cual impide que múltiples hilos ejecuten código Python puro en paralelo dentro de un mismo proceso. Esto influye en su performance ya que todos los hilos, excepto en algunas ocasiones como en las operaciones de entrada/salida, se van a ejecutar en un solo procesador y por lo tanto no se aprovecha todo el potencial de un computador.

El módulo threading permite la ejecución concurrente de tareas dentro del mismo proceso, pero debido al GIL, los hilos no pueden aprovechar múltiples núcleos para realizar cómputo intensivo. Aun así, el threading sigue siendo útil, y sobre todo en el enfoque que estamos tomando para su utilización. Aclarando, los hilos pueden aprovechar el tiempo en el que un proceso está inactivo esperando información, permitiendo la ejecución de otras tareas en paralelo.

Por otro lado, cuando se requiere ejecutar tareas que demandan un alto uso de CPU, la mejor alternativa es multiprocessing, que crea procesos independientes con su propio intérprete de Python, evitando la restricción del GIL y permitiendo una ejecución realmente paralela.

Teniendo en cuenta que este proyecto está pensado para que escale en cantidad de robots, y se pueda controlar una flota de ellos, se ha optado por utilizar threading debido a que cada hilo va a representar un robot, cuya gestión va a estar coordinada a través del monitor basado en una Red de Petri. Dado que los robots no requieren cómputo intensivo, sino que dependen principalmente de operaciones de I/O y sincronización de eventos, threading resulta adecuado a pesar de las limitaciones del GIL.

Para garantizar un control ordenado de los hilos, se ha implementado un mecanismo de sincronización mediante Lock, permitiendo bloquear el hilo de un robot cuando este no puede avanzar y continuar con los demás. Este enfoque asegura una ejecución eficiente dentro del marco definido por la red de Petri, optimizando el flujo de trabajo sin afectar la estabilidad del sistema.

\subsubsection{Monitor}

La manera en que controlamos la ocupación de los espacios en el mapa, es decir, cómo evoluciona la trayectoria de cada robot es mediante el disparo de un Monitor implementado en Python. Este es el va a controlar, guiar y realizar la toma de decisiones para que cada robot pueda seguir su trayectoria sin interrumpir a los demás y por lo tanto no bloquear todo el sistema.

Para poder guiar a cada robot en su trayectoria, cada uno va a estar representado por un hilo lógico en el programa en Python. Más precisamente cuando existe un robot en el mundo real y se realiza el proceso de sincronización con el software, este va a crear un objeto (representación de un ente en la Programación Orientada a Objetos) el cual va a contener un conjunto de métodos implementados y entre ellos, la creación de un hilo en Python.

A su vez definimos cuál va a ser el punto de inicio y el punto final, es decir el comienzo del recorrido y el final del mismo, y el mapa es un modelo discreto representado por una Red de Petri. Podemos determinar cuales van a ser la plazas que el robot va a ocupar en un preciso momento y las transiciones que se van a disparar para que este pueda llegar a su destino. Con todos estos elementos le damos al monitor las herramientas para realizar su trabajo y efectuar los métodos que le permitan modificar el marcado de la Red de Petri.

\paragraph{Cola de cortesía} \mbox{} \vspace{8pt}

La manera en que el Monitor ejerce el control de la exclusión mutua dentro de la Red de Petri es mediante el uso de colas, entonces, cuando un hilo (robot) está dentro del monitor y aparece otro hilo que intenta ejecutar otro o el mismo procedimiento, el acceso al Monitor se bloquea e inserta el segundo hilo en una cola de cortesía usando una política FIFO. Cuando el primer hilo abandona el monitor, el segundo hilo (que se encuentra en el primer lugar de la cola) es el seleccionado por el Monitor para ejecutar sus tareas. Si la cola está vacía entonces el Monitor se encuentra libre y cualquier hilo que intente tomar el control de éste último podrá hacerlo sin intervención alguna.

\paragraph{Política del monitor} \mbox{} \vspace{8pt}

Así como mencionamos más arriba que hacemos uso de las colas del monitor para controlar la exclusión mutua, existe la posibilidad de que dos hilos (robots) intenten acceder a la misma plaza ya que su secuencia de disparos así lo determina. En ese caso vamos a tener que tomar una decisión de qué hilo es que va a lograr apoderarse de la plaza en primer lugar. Para esto hacemos uso de una Política definida por nosotros mismos, a partir de ella un hilo va a tener preferencia sobre otro. La política decide sobre todos los hilos que tengan una transición sensibilizada correspondiente a la celda que quiere ocupar.

Poniendo el caso en concreto de los robots que comparten un mapa, nos parece más acertada la idea de que un robot pueda terminar su recorrido de forma rápida y efectiva. Es por ello que cada robot va guardando el camino que va atravesando, es decir, las celdas por las cuales se desplazó y aumentando un contador interno. Este es el parámetro que se usa para determinar qué robot va a tener preferencia sobre otro, entonces, al momento de ocurrir un conflicto entre dos robots que intentan ocupar la misma celda, el Monitor va a evaluar que robot le falta menos camino por recorrer y va a elegirlo para ocupar el lugar en disputa. 

\paragraph{Solución de conflictos entre robots} \mbox{} \vspace{8pt}

Acá se representa el posible conflicto que puede ocurrir cuando dos robots intentan pasar por el mismo lugar (celda) ya que se su trayectoria así lo define. La figura \ref{fig:conflicto_map} muestra entonces que el robot A tiene como destino la celda $14$ mientras que el robot B tiene como destino la celda $16$ y la celda en conflicto es la $8$.

\begin{figure}[H]
    \centering
    \includegraphics[trim={1.6cm 0.3cm 0 1.5cm}, clip, width=0.7\linewidth]{images/conflicto_map.png}
    \caption{Representación de un conflicto entre dos robots que intentan pasar por la misma celda}
    \label{fig:conflicto_map}
\end{figure}

Esta misma situación se la plantea en la figura \ref{fig:rdp_no_grid_conflicto} con una red de Petri con sus respectivos recursos que controlan la exclusión mutua de las celdas y las transiciones sensibilizadas marcadas en rojo $\{T_2, T_4, T_5, T_7, T_{11}, T_{13}\}$ que muestran los posibles caminos que puede tomar el robot dado el marcado actual de la red. El robot A intenta disparar la transición sensibilizada $T_7$ mientras que el robot B intenta hacer lo mismo pero con la transición $T_{11}$.

\begin{figure}[H]
    \centering
    \includegraphics[width=0.6\linewidth]{images/rdp_no_grid_conflicto.png}
    \caption{Representación del conflicto de la figura \ref{fig:conflicto_map} en una red de Petri}
    \label{fig:rdp_no_grid_conflicto}
\end{figure}

El conflicto es resuelto mediante la política que se definió mas arriba en esta misma sub sección, y da como resolución que el robot A es el que avance en la ocupación de la celda en disputa tal como muestra la figura \ref{fig:conflicto_map_solucionado}. Esto significa que una vez que el robot A finalice su recorrido en la celda $14$, el robot B va a poder ocupar la celda $8$ y también terminar con su recorrido.

\begin{figure}[H]
    \centering
    \includegraphics[trim={1.6cm 0.3cm 0 1.5cm}, clip, width=0.7\linewidth]{images/conflicto_map_solucionado.png}
    \caption{Resolución del conflicto de dos robots intentando pasar por la misma celda}
    \label{fig:conflicto_map_solucionado}
\end{figure}

La solución del conflicto también se ve representada en la red de Petri que muestra un marcado diferente en la figura \ref{fig:rdp_no_grid_conflicto_solucionado}, colocando al robot A en la plaza $PE$ y también revelando que las transiciones sensibilizas ahora son $\{T_4, T_8, T_{12}, T_{13}, T_{15}, T_{17}\}$ que por supuesto se diferencian al estar en color rojo. Ya resuelto el conflicto se puede observar que ahora el robot A tiene como posibilidad el disparo de la transición $T_{17}$ para llegar a su destino mientras que el robot B no puede avanzar hacia la plaza $PE$ porque se encuentra ocupada.

\begin{figure}[H]
    \centering
    \includegraphics[width=0.7\linewidth]{images/rdp_no_grid_conflicto_solucionado.png}
    \caption{Resolución del conflicto entre los robots representado en una red de Petri y con el nuevo marcado de la red}
    \label{fig:rdp_no_grid_conflicto_solucionado}
\end{figure}

\newpage
\paragraph{Funcionamiento del monitor} \mbox{} \vspace{8pt}

El funcionamiento del módulo del Monitor implementado en Python queda representado en el diagrama de secuencia de la figura \ref{fig:diagrama_monitor}.
Una de las cosas a tener en cuenta en el diagrama es que al Python estar limitado por la integración de $GIL$ y el manejo de los hilos, se vio la necesidad de implementar una condición para cada hilo en donde se le asigna una prioridad y por lo tanto el Monitor puede elegir cual va a ser el hilo a ejecutar. Esto se puede interpretar como la imposición sobre el mecanismo $GIL$ y forzarlo a elegir el hilo a ejecutar.

\begin{figure}[H]
    \centering
    \hspace*{-0,8cm}
    \includegraphics[width=1.2\linewidth]{images/diagrama_monitor.jpg}
    \caption{Diagrama de secuencia del monitor}
    \label{fig:diagrama_monitor}
\end{figure}

Puede darse la situación en que un hilo no dispare una transición ya que no cumple las condiciones para hacerlo. En este caso el hilo es agregado a la $Cola\ de\ hilos\ bloqueados$.

También puede surgir la situación en que dos robots entran en un conflicto donde pretenden intercambiar lugares, es decir que un robot intenta ocupar la celda del otro y viceversa cómo lo muestra la figura \ref{fig:conflicto_path}. Acá se ve involucrada la $Cola\ de\ hilos\ en\ conflicto$ donde se añaden los robots en cuestión. Este conflicto se soluciona mediante la implementación de una nueva política de re-calculo de trayectorias que toma como elementos de entrada la cola ya mencionada.

\begin{figure}[H]
    \centering
    \includegraphics[trim={1.6cm 0.3cm 0 1.5cm}, clip, width=0.7\linewidth]{images/conflicto_path.png}
    \caption{Situación de conflicto en donde dos robots intentan intercambiar lugares}
    \label{fig:conflicto_path}
\end{figure}

Cabe destacar que el robot elegido por la política es penalizado dado que debe modificar su secuencia de coordenadas original (la más corta calculada por el algoritmo de $A*$) para darle lugar al otro robot. Esto implica que también se modifican las transiciones a disparar.

A continuación en la figura \ref{fig:diagrama_monitor_recalculate_path} se muestra en detalle la situación ya descrita y amplía el estado del Monitor cuando $k == false$ en el diagrama de secuencia \ref{fig:diagrama_monitor}.

\begin{figure}[H]
   \centering
   \hspace*{-2,0cm}
   \includegraphics[width=1.3\linewidth]{images/diagrama_monitor_recalculate_path.jpg}
   \caption{Diagrama de secuencia del monitor que contempla la lógica de re-calcular la trayectoria de un robot bloqueado}
   \label{fig:diagrama_monitor_recalculate_path}
\end{figure}

\subsection{Testing y pruebas} \mbox{} \vspace{5pt}

Las pruebas que se realizaron para que efectivamente podamos determinar el correcto funcionamiento del Monitor, la gestión de la cola de cortesía y la resolución de conflictos mediante la aplicación de una Política fue con la ayuda de la Red de Petri sumamente conocida como Productor-Consumidor. Esta se tomó cómo referencia ya que permite evaluar los problemas de sincronización, concurrencia y gestión de los recursos, entonces, nos pareció un buen puntapié para validar el funcionamiento con un modelo conocido y sumamente usado.

\begin{figure}[H]
   \centering
   \includegraphics[width=0.9\linewidth]{images/rdp_prod_cons.png}
   \caption{Red de Petri productor-consumidor}
   \label{fig:rdp_prod_cons}
\end{figure}

Se plantearon los siguientes casos:

\begin{testtableformat}
   \hline \rowcolor{test_header_color}
       Test ID             & TC\_04\_00 \\
   \hline
       Tipo de test        & Test unitario \\
   \hline
       Objeto de prueba    & Disparar las transiciones sensibilizadas del Productor \\
   \hline
       Nombre              & Ejecución del productor \\
   \hline
       Descripción         & Realizar el disparo de las transiciones sensibilizadas del productor teniendo como máximo 3 (tres) tokens disponibles para producir\\
   \hline
       Precondición        & PRECOND\_D \\
   \hline
       Pasos del test      & \begin{enumerate}
                             \item Instanciar a la clase monitor con la red productor-consumidor
                             \item Obtener el vector de transiciones sensibilizadas 
                             \item Ejecutar un disparo de la transición sensibilizada 
                             \item Adquirir el marcado saliente y compararlo con un programa de análisis de redes de Petri 
                             \end{enumerate}\\
   \hline
       Resultado esperado  & El marcado en ambos casos debe coincidir \\
   \hline
       Resultado obtenido  & El marcado de la red obtenido por la clase monitor y el programa de análisis de redes de Petri son iguales, por lo tanto, se verifica su adecuado funcionamiento \\
   \hline
       Observaciones       & -\\
   \hline
\end{testtableformat}

\begin{testtableformat}
   \hline \rowcolor{test_header_color}
       Test ID             & TC\_04\_01 \\
   \hline
       Tipo de test        & Test unitario \\
   \hline
       Objeto de prueba    & Disparar las transiciones sensibilizadas del consumidor \\
   \hline
       Nombre              & Ejecución del consumidor \\
   \hline
       Descripción         & Realizar el disparo de las transiciones sensibilizadas del consumidor teniendo como máximo 3 (tres) recursos disponibles generados por el productor\\
   \hline
       Precondición        & PRECOND\_D \\
   \hline
       Pasos del test      & \begin{enumerate} 
                             \item Instanciar a la clase monitor con la red productor-consumidor 
                             \item Obtener el vector de transiciones sensibilizadas 
                             \item Ejecutar un disparo de la transición sensibilizada 
                             \item Adquirir el marcado saliente y compararlo con un programa de analisis de redes de petri 
                             \end{enumerate}\\
   \hline
       Resultado esperado  & El marcado en ambos casos debe coincidir \\
   \hline
       Resultado obtenido  & El marcado de la red obtenido por la clase monitor y el programa de análisis de redes de Petri son iguales, por lo tanto, se verifica su adecuado funcionamiento \\
   \hline
       Observaciones       & -\\
   \hline
\end{testtableformat}

\begin{testtableformat}
   \hline \rowcolor{test_header_color}
       Test ID             & TC\_04\_02 \\
   \hline
       Tipo de test        & Test unitario \\
   \hline
       Objeto de prueba    & Disparar una transición con política de decisión \\
   \hline
       Nombre              & Ejecución de una transición con política definida \\
   \hline
       Descripción         & En en el caso donde tanto el productor cómo el consumidor pueden disparar una transición, la idea es evaluar la política de la transición del consumidor para que tenga prevalencia sobre el productor ante una situación indeterminista \\
   \hline
       Precondición        & PRECOND\_E \\
   \hline
       Pasos del test      & \begin{enumerate} 
                             \item Instanciar a la clase monitor con la red productor-consumidor 
                             \item Generar una situación de equidad donde ambos recursos puedan disparar su transición 
                             \item Obtener el vector de transiciones sensibilizadas 
                             \item Evaluar la política de las transiciones sensibilizada y elegir cual disparar 
                             \item Ejecutar un disparo de la transición elegida 
                             \item Adquirir el marcado saliente y compararlo con un programa de análisis de redes de petri 
                             \end{enumerate}\\
   \hline
       Resultado esperado  & El marcado actual en ambos casos debe coincidir y la transición disparada debe ser la del consumidor \\
   \hline
       Resultado obtenido  & El marcado de la red obtenido por la clase monitor y el programa de análisis de redes de Petri son iguales, por lo tanto, se verifica su adecuado funcionamiento \\
   \hline
       Observaciones       & -\\
   \hline
\end{testtableformat}

\begin{testtableformat}
   \hline \rowcolor{test_header_color}
       Test ID             & TC\_04\_03 \\
   \hline
       Tipo de test        & Test integración \\
   \hline
       Objeto de prueba    & Realizar movimientos con el robot dentro del mapa, el robot sólo va a avanzar cuando la red de Petri haya hecho la solicitud de disparo y haya verificado que el robot no se encuentra bloqueado.\\
   \hline
       Nombre              & Desplazamientos del robot controlado por el Monitor\\
   \hline
       Descripción         & Para que el robot avance de una celda a otra y realice el desplazamiento definido (punto inicial y final) debe ser autorizado y coordinado por el monitor que controla la red de Petri.\\
   \hline
       Precondición        & PRECOND\_E \\
   \hline
       Pasos del test      & \begin{enumerate}
                             \item Indicar los puntos de inicio y final del robot
                             \item Esperar que el Monitor dispare la transición sensibilizada
                             \item El robot recibe la orden de avanzar a la celda libera
                             \item El robot envía una señal de llegada a la celda (plaza)
                             \item Se dispara la siguiente transición sensibilizada
                             \item Este proceso se repite hasta que el robot llegue a su destino
                             \end{enumerate}\\
   \hline
       Resultado esperado  & El robot pueda completar su desplazamiento definido por el usuario\\
   \hline
       Resultado obtenido  & La red de Petri al no bloquearse permite que todas las transiciones sensibilizadas se puedan disparar y por ende que el robot pueda desplazarse por todas las celdas (plazas) involucradas en su recorrido\\
   \hline
       Observaciones       & -\\
   \hline
\end{testtableformat}

\begin{testtableformat}
    \hline \rowcolor{test_header_color}
        Test ID             & TC\_04\_04 \\
    \hline
        Tipo de test        & Test de sistema \\
    \hline
        Objeto de prueba    & Comunicación inalámbrica - PID - Modelo cinemático - Odometría - Seguidor de línea magnética - Modelo del mapa - Calculador de trayectorias - Red de Petri - Monitor\\
    \hline
        Nombre              & Prueba de sistema integrado\\
    \hline
        Descripción         & Comprobar que el robot se puede desplazar por el mapa gestionado por el marcado de la red de Petri\\
    \hline
        Precondición        & PRECOND\_G \\
    \hline
        Pasos del test      & \begin{enumerate}
                              \item Indicar los puntos de inicio y final del robot en el mapa
                              \item Calcular la secuencia de disparos del Monitor
                              \item Esperar que el Monitor dispare la transición sensibilizada
                              \item El robot recibe la orden de avanzar a la celda (plaza) liberada
                              \item El robot envía una señal de llegada a la celda (plaza) destino
                              \item Se dispara la siguiente transición sensibilizada
                              \item Este proceso se repite hasta que el robot llegue a su destino
                              \end{enumerate}\\
    \hline
        Resultado esperado  & El robot pueda completar su desplazamiento definido por el usuario\\
    \hline
        Resultado obtenido  & La red de Petri al no bloquearse permite que todas las transiciones sensibilizadas se puedan disparar y por ende que el robot pueda desplazarse por todas las celdas (plazas) involucradas en su recorrido\\
    \hline
        Observaciones       & Se probó recorridos de 4 de metros por limitaciones de espacio\\
    \hline
 \end{testtableformat}

\subsection{Resultados}
Como muestran las pruebas realizadas del módulo Monitor implementado en Python, el algoritmo se comporta de forma correcta y las validaciones realizadas sobre la red de Petri, productor-consumidor, entregaron los resultados suficientes para poder llevar este procedimiento a las redes que representan los mapas donde los robots se van a desplazar y tener la validez que el monitor va a permitir que estos se muevan de forma coordinada y puedan llegar a su destino.

\subsection{Riesgos superados}

\begin{center}
    \begin{tabular} {
        | c| c |}
        \hline \rowcolor{test_header_color}
            ID & Riesgo \\
        \hline
            RI-02 & Intercomunicación de componentes ineficiente o ineficaz \\
        \hline
            RI-03 & Prestaciones insuficientes de componentes \\
        \hline
    \end{tabular}
\end{center}

\subsection{Conclusiones}
El desarrollo del sistema de control para robots omnidireccionales, basado en redes de Petri, demostró ser altamente efectivo en la gestión de trayectorias y prevención de colisiones en entornos multi-robot. Mediante la discretización del mapa en celdas y su representación como lugares en la red, se logró un modelo claro y funcional que garantiza la ocupación exclusiva de regiones con capacidad limitada. La implementación de políticas de prioridad dinámica, donde los robots con mayor recorrido tienen preferencia, permitió resolver conflictos de manera eficiente, evitando bloqueos y asegurando un flujo continuo de operaciones.

En síntesis, el sistema desarrollado no solo cumple con los requerimientos funcionales y no funcionales planteados, sino que también sienta las bases para extensiones más complejas, destacando la versatilidad de las redes de Petri en el ámbito del control y la automatización.

\newpage
\section{Iteración 5: Compensación de posición y trayectoria}

\subsection{Introducción}

Hasta el momento se ha implementado exitosamente un prototipo que se ubica en un mapa gobernado por una red de Petri junto a un Monitor, lo que brinda una representación formal y estructurada del espacio operativo.

En la presente etapa, el objetivo principal es implementar una técnica que permita estimar con precisión la posición del robot dentro de su entorno modelado. Entre las posibles soluciones, se considera la aplicación de un Filtro de Kalman, dada su capacidad para optimizar estimaciones en sistemas dinámicos y con incertidumbre. Este enfoque busca robustecer el desempeño del sistema, incrementando la precisión y confiabilidad en la localización del robot.

\subsection{Requerimientos}

En primer lugar, los requerimientos funcionales a tratar son:

\begin{center} \begin{tabular}{|c|c|} 
\hline
    ID & Descripción \\
\hline
    RF6 & El robot debe recibir y enviar información mediante comunicaciones inalámbricas. \\ 
\hline
    RF8 & Debe poder ubicarse al robot en el plano de forma precisa. \\
\hline
\end{tabular} \end{center}

Por otra parte, los requerimientos no funcionales que abordaremos son:

\begin{center} \begin{tabular}{|p{0.15\linewidth}|p{0.55\linewidth}|} 
\hline
    ID & Descripción \\
\hline
    RNF1 & Debería tener tiempos de respuesta aceptables para el buen funcionamiento del sistema de control. \\
\hline
    RNF2 & El software debería contar con pruebas unitarias y de integración. \\
\hline
    RNF4 & El código debería contar con documentación. \\
\hline
\end{tabular} \end{center}


\subsection{Desarrollo}

\subsubsection{Filtro de Kalman}

El filtro de Kalman es un algoritmo de estimación utilizado para predecir el estado de un sistema dinámico mediante la combinación de datos provenientes de uno o mas sensores, considerando la incertidumbre y el ruido de medición. En el contexto del robot omnidireccional del presente proyecto, este filtro tiene un rol crucial, ya que permite corregir errores en la localización, estimar velocidades más precisas y mejorar la trayectoria mediante la integración de señales provenientes de múltiples sensores.

El funcionamiento del filtro de Kalman se basa en dos etapas: la predicción, donde se estima el próximo estado del sistema utilizando el modelo matemático exacto o teórico; y la actualización, en la que se corrige esa estimación en función de las mediciones actuales, aplicando un modelo probabilístico para minimizar el error. Mediante su modelo probabilístico, es capaz de filtrar el ruido y las imprecisiones de las mediciones, lo que permite obtener una representación confiable del estado.

En cuanto a su implementación, se desarrollará una integración entre el modelo cinemático existente y el filtro de Kalman en Python. Los resultados esperados incluyen un aumento significativo en la precisión de la navegación del robot y una mejora en su capacidad de adaptarse a entornos ruidosos.


\paragraph{Estimación del estado} \mbox{} \vspace{8pt}

A través de un modelo probabilístico, el filtro combina las predicciones del sistema basadas en el modelo cinemático con mediciones reales provenientes de sensores. Esto permite minimizar el impacto del ruido y las imprecisiones inherentes a las lecturas de los sensores. El proceso de estimación sigue dos etapas principales:

\begin{itemize}
    \item Predicción: Utilizando el modelo de movimiento del robot, se genera una estimación inicial del estado en el instante de tiempo dado.
    \item Actualización: Esta estimación se corrige comparándola con las mediciones obtenidas en tiempo real, ajustando los valores según la incertidumbre de cada fuente de datos. El resultado es una representación más precisa del estado actual del robot.
\end{itemize}

La información estimada por el filtro de Kalman se utiliza para compensar trayectorias y corregir vectores de movimiento. En caso de detectar desviaciones respecto a la trayectoria deseada, se ajustan dinámicamente los comandos de control (distancia y velocidad) enviados al robot, compensando errores como ser resbalamiento, impactos externos e irregularidades del terreno. Esto asegura que el robot mantenga su rumbo previsto con alta precisión y adaptabilidad a entornos variables, además, la capacidad de compensación mejora la fluidez de los desplazamientos.

En términos prácticos, la implementación del filtro de Kalman no solo refuerza la fiabilidad del sistema de navegación, sino que también sienta una base sólida para futuros desarrollos, como la integración con sensores más avanzados o la aplicación en sistemas autónomos.


\paragraph{Matrices de covarianza} \mbox{} \vspace{8pt}

Las matrices de covarianza expresan la relación estadística entre las distintas variables del sistema, cuantificando tanto la magnitud del ruido como la correlación entre las mediciones obtenidas de diferentes sensores. Cada elemento de la matriz de covarianza representa cómo dos variables están relacionadas en términos de su varianza compartida; por lo tanto, si las variables son independientes, sus valores excepto los de la diagonal principal de la matriz son nulos.

El propósito de estas matrices en el filtro de Kalman y en sistemas similares radica en su capacidad para modelar la incertidumbre. Se utilizan tanto en la etapa de predicción, como en la de actualización del filtro para ponderar el impacto de los errores de medición, mejorando las estimaciones del estado. \cite{tzafestas2013introduction} \cite{Rigatos01062007}

La calibración de las matrices de covarianza es crucial para garantizar su correcta representación del ruido real del sistema. Esto puede lograrse mediante técnicas experimentales, analizando mediciones históricas para calcular las varianzas y covarianzas directamente. También se pueden implementar métodos estadísticos que ajusten las matrices iterativamente según los resultados obtenidos durante el funcionamiento del robot. Otra técnica consiste en realizar simulaciones repetitivas, diseñadas para identificar los patrones de error y ajustar las matrices en consecuencia. \cite{Bang18}

\paragraph{Cambio de coordenadas medición/estimación} \mbox{} \vspace{8pt}

Dado que en la próxima iteración se va a colocar una cámara fija en el robot que queremos que esté siempre apuntando hacia el frente, cuando el robot realiza trayectorias y necesita hacer un cambio de dirección, primero realiza un giro sobre su propio eje hasta orientar al robot correctamente.

Por otro lado, el robot al moverse siempre a lo largo sobre su propio eje $y$, va a reportar siempre mediciones donde la velocidad $v_y$ es distinta de cero y $v_x$ es cercana a cero. Es por ello que se agrega una variable que lleva la cuenta de la orientación del robot en grados $[0^{\circ}, 90^{\circ}, 180^{\circ}, 270^{\circ}]$ respecto al sistema de coordenadas global. Con esta información realizamos un cambio de coordenadas para adaptar el sistema de coordenadas locales del robot y sus mediciones, a las globales del mapa.

\paragraph{Implementación} \mbox{} \vspace{8pt}

El filtro de Kalman se implementa de modo que registra velocidad y posición en un eje determinado, por lo que replicando el filtro 2 veces podemos tener una representación de un plano $XY$. Esta implementación se realiza en Python y se integra al funcionamiento del sistema.

Del filtro de Kalman 2D registramos la siguiente tupla:
$$ [x, v_x, y, v_y] $$

Siendo $x$ la posición absoluta en el eje $x$, $v_x$ la velocidad en el eje $x$, $y$ la posición absoluta en el eje $y$ y $v_y$ la velocidad en el eje $y$. Cada Filtro de Kalman se encarga de cada eje $X$ e $Y$ en particular.
Es importante que la entrada y salida del filtro sean del mismo formato, es decir, tenga como entrada y salida la tupla $[x, v_x, y, v_y]$. En este caso introducimos las mediciones acumuladas en distancia y las velocidades en cada eje cartesiano.

La secuencia de procesos a realizar cada vez que se recibe una medición es el siguiente:
\begin{enumerate}
    \item Cálculo del Estado Predicho $X_{kp}$.
    \item Cálculo de la Matriz de Covarianza de Proceso Predicha $P_{kp}$.
    \item Cálculo de la ganancia de Kalman $K$.
    \item Cálculo del Estado Estimado $X_k$.
    \item Actualización de valores.
    \item Actualización de la Matriz de Covarianza de Proceso $P_k$.
\end{enumerate}

Por otra parte, esperamos que el funcionamiento del Filtro de Kalman sea tal que podamos actualizar el estado por cada medición recibida del robot y calcular el vector de compensación según el estado estimado por el filtro. Este proceso se describe en el diagrama de secuencia de la Figura \ref{fig:diagsecuenciafiltrokalman}. \\


\textbf{Adaptación de datos de entrada} \mbox{} \vspace{8pt}

Para comenzar trataremos el proceso para adaptar los datos de las mediciones. En primer lugar, tenemos que el estado observado $Y_k$ puede ser expresado como:

$$ Y_k = C \cdot Y_{km} + Z_k $$

Donde la matriz $C$ transforma la medición al formato necesario, en este caso es una matriz identidad por no necesitar transformarse. Además se tiene la matriz $Z_k$ que modela el error en las observaciones, como ser error en los dispositivos, retardos, fallas mecánicas, entre otras, tambien lo consideramos despreciable.

Por lo que tenemos que el estado observado se puede expresar como:

$$ Y_k =
    \begin{bmatrix}
    1 & 0 \\
    0 & 1  \\
    \end{bmatrix}
    \cdot
    \begin{bmatrix} x_{km} \\ v_{km} \\ \end{bmatrix}
    +
    \begin{bmatrix}
    0 & 0  \\
    0 & 0  \\
    \end{bmatrix}
    =
    \begin{bmatrix} x_{km} \\ v_{km} \\ \end{bmatrix}
$$

Siendo $x_{km}$ y $v_{km}$ los valores de posición y velocidad medidos. La posición es acumulada y la velocidad es la instantánea. \\

\begin{figure}[H]
    \centering
    \includegraphics[width=1.1\linewidth]{images/diag_secuencia_filtro_de_kalman_con_esp32.png}
    \caption{Diagrama de secuencia con el Filtro de Kalman}
    \label{fig:diagsecuenciafiltrokalman}
\end{figure}


\textbf{Modelo del filtro} \mbox{} \vspace{8pt}

El filtro de Kalman requiere que modelemos matemáticamente nuestro sistema. En nuestro caso registraremos la posición y velocidad en un eje determinado. El modelo matemático es en base a la ecuación de movimiento rectilíneo acelerado, por lo que para la posición y velocidad instantáneas se tiene:

$$ x = x_0 + x_0 \Delta t + \frac{1}{2} a \Delta t^2 $$
$$ v_x = v_0 + a\Delta t $$

\textbf{Cálculo del Estado Predicho $X_{kp}$} \mbox{} \vspace{8pt}

Este término significa en donde debería estar el robot según el modelo ideal y la distancia entre muestras $\Delta t$. En base a lo anterior en forma matricial, tenemos que el estado predicho está dado por:

$$ X_{kp} =
    \begin{bmatrix} x_{kp} \\ v_{kp} \\ \end{bmatrix}
    =
    A \cdot X_{k-1} + B \cdot U_k + \omega_k
$$

Las matrices A y B representan al modelo en base a las ecuaciones de arriba, mientras que $\omega_k$ representa algún ruido o perturbación presente en el proceso de cálculo del estado predicho. En nuestro caso lo consideramos despreciable, por lo que es nulo. Además, el vector $U_k$ se considera nulo dado que no llevaremos registro de la aceleración del robot.

Donde:

$$
X_{k-1} = \begin{bmatrix} x_{k-1} \\ v_{k-1} \\ \end{bmatrix}
;
A = \begin{bmatrix}
    1 & \Delta t  \\
    0 & 1         \\
    \end{bmatrix} \\
;
B = \begin{bmatrix}
    \frac{1}{2} \Delta t^2 \\
    \Delta t               \\
    \end{bmatrix}          \\
;
U_k = \begin{bmatrix} a_0 \\ \end{bmatrix} \\
;
\omega_k = \begin{bmatrix} 0 \\ 0 \\ \end{bmatrix}
$$

Entonces la ecuación para obtener el nuevo estado predicho:

$$
X_{kp} = 
    \begin{bmatrix}
    1 & \Delta t  \\
    0 & 1         \\
    \end{bmatrix} \\
    \cdot \begin{bmatrix} x_{k-1} \\ v_{k-1} \\ \end{bmatrix}
    +
    \begin{bmatrix}
    \frac{1}{2} \Delta t^2 \\
    \Delta t               \\
    \end{bmatrix}          \\
    \cdot
    \begin{bmatrix} 0 \\ \end{bmatrix} \\
    +
    \begin{bmatrix} 0 \\ 0 \\ \end{bmatrix}
$$ \\

\textbf{Cálculo de Matriz de Covarianza de Proceso Predicha $P_{kp}$} \mbox{} \vspace{8pt}

Esta matriz se modela teniendo por un lado el término $\Delta_{px}$ y por otro lado el término $\Delta_{pv_x}$ que representan errores en el proceso de calcular la posición y velocidad predichos.

En el caso que exista relación en la incerteza del proceso para calcular la posición y velocidad, el término $\Delta_{px} \cdot \Delta_{pv_x}$ es distinto de cero. En nuestro caso lo establecemos en $0$.

Por lo que se tiene la siguiente matriz de covarianza inicial:

$$
P_{k-1} = \begin{bmatrix}
    {\Delta_{px}}^2 & \Delta_{px} \cdot \Delta_{pv_x} \\
    \Delta_{px} \cdot \Delta_{pv_x} & {\Delta_{pv_x}}^2 \\
    \end{bmatrix} \\
    = \begin{bmatrix}
    {\Delta_{px}}^2 & 0 \\
    0 & {\Delta_{pv_x}}^2 \\
    \end{bmatrix} 
$$

Luego, para calcular la nueva matriz se parte de la siguiente expresión:

$$ P_{kp} = A \cdot P_{k-1} A^T + Q_k $$

Donde la matriz A es la misma expresada en el paso anterior y $Q_k$ modela el error en el proceso de cálculo de las matrices de covarianza, también establecido en $0$. Por lo que obtenemos:

$$ P_{kp} =
    \begin{bmatrix}
    1 & \Delta t  \\
    0 & 1         \\
    \end{bmatrix} \\
    \cdot 
    \begin{bmatrix}
    {\Delta_{px}}^2 & 0   \\
    0 & {\Delta_{pv_x}}^2 \\
    \end{bmatrix}
    \cdot
    \begin{bmatrix}
    1 & 0           \\
    \Delta t & 1    \\
    \end{bmatrix}   \\
    +
    \begin{bmatrix}
    0 & 0  \\
    0 & 0  \\
    \end{bmatrix}  \\
$$

\textbf{Cálculo de la ganancia de Kalman $K$} \mbox{} \vspace{8pt}

La ganancia de Kalman en pocas palabras es una media ponderada que le da mas o menos importancia a la medición versus la estimación según el error detectado. Si la ganancia $K$ tiende a 0, indica que el sistema no converge y que tiene un error considerable. Por otro lado si la ganancia $K$ tiende a 1 indica que tiene poco error.

Para calcular la ganancia de Kalman utilizamos la siguiente expresión:

$$ K = \frac{P_{kp} \cdot H^T}
            {H \cdot P_{kp} \cdot H^T + R}
$$

Donde la matriz $H$ es una matriz identidad $I$ que se utiliza para convertir el dominio de los datos al formato que necesita $K$. La matriz $R$ es una matriz de covarianzas que representa los errores en la observación, que se puede simplificar del mismo modo que $P_{kp}$ y se puede expresar como:

 $$ R =
    \begin{bmatrix}
    {\Delta_x}^2 & \Delta_x \cdot \Delta_{v_x} \\
    \Delta_x \cdot \Delta_{v_x} & {\Delta_{v_x}}^2 \\
    \end{bmatrix} \\ 
    =
    \begin{bmatrix}
    {\Delta_x}^2 & 0 \\
    0 & {\Delta_{v_x}}^2 \\
    \end{bmatrix} \\ 
$$

Por lo que para calcular la ganancia de Kalman:

$$ K =
    \frac{
        \begin{bmatrix}
        {\Delta_{px}}^2 & 0 \\
        0 & {\Delta_{pv_x}}^2 \\
        \end{bmatrix}
        \cdot
        \begin{bmatrix}
        1 & 0 \\
        0 & 1  \\
        \end{bmatrix}
        }
    {
        \begin{bmatrix}
        1 & 0 \\
        0 & 1  \\
        \end{bmatrix}
        \cdot
        \begin{bmatrix}
        {\Delta_{px}}^2 & 0 \\
        0 & {\Delta_{pv_x}}^2 \\
        \end{bmatrix}
        \cdot
        \begin{bmatrix}
        1 & 0 \\
        0 & 1  \\
        \end{bmatrix}
        +
        \begin{bmatrix}
        {\Delta_x}^2 & 0 \\
        0 & {\Delta_{v_x}}^2 \\
        \end{bmatrix} \\ 
    }
$$ \\

\textbf{Cálculo del Estado Estimado $X_k$} \mbox{} \vspace{8pt}

Para obtener la nueva estimación del estado en base a la última medición recibida, podemos expresar:

$$ X_k = X_{kp} + K \cdot ( Y_k - H \cdot X_{kp} ) $$

Donde $X_{kp}$ es el estado predecido, $K$ es la ganancia de Kalman, $Y_k$ es la medición observada y $H$ una matriz de conversión. Esta ultima resulta ser una matriz identidad por no ser necesario convertir los datos del estado predicho.

$$ X_k =
    X_{kp}
    +
    K
    \cdot (
        \begin{bmatrix} x_{km} \\ v_{km} \\ \end{bmatrix}
        -
        \begin{bmatrix}
        1 & 0 \\
        0 & 1  \\
        \end{bmatrix}
        \cdot
        X_{kp}
    )
$$ \\

\textbf{Actualización de la Matriz de Covarianza de Proceso $P_k$} \mbox{} \vspace{8pt}

Eventualmente, cuando el error se reduce lo suficiente, podemos actualizar la matriz de covarianza de proceso mediante la siguiente expresión:

$$ P_k = ( I - K \cdot H ) \cdot P_{kp} $$

Donde $H$ nuevamente es una matriz identidad. Por lo que resulta:

$$ P_k =
    (
    \begin{bmatrix}
    1 & 0 \\
    0 & 1  \\
    \end{bmatrix}
    -
    K
    \cdot
    \begin{bmatrix}
    1 & 0 \\
    0 & 1  \\
    \end{bmatrix}
    )
    \cdot
    P_{kp}
$$

\textbf{Actualización de valores} \mbox{} \vspace{8pt}

Antes de comenzar una nueva iteración debemos actualizar los valores de la matriz de covarianza y la de estado predecido:

$$ P_{k-1} = P_k $$
$$ X_{k-1} = X_k $$

\subsubsection{Compensación con Filtro de Kalman}

Dado que el filtro de Kalman puede estimar con precisión un estado en base a sucesivas mediciones, podemos utilizarlo para ubicar en todo momento al robot en el plano con su coordenada y velocidad estimadas. Esta información la podemos utilizar para determinar la corrección necesaria a aplicar para que el robot corrija su trayectoria, procederemos a describir este proceso.

Vale aclarar que el robot recibe comandos del siguiente formato, donde se especifica la distancia a recorrer con un vector determinado:
$$ [distancia, v_x, v_y, v_\theta] $$

Por otra parte el robot reporta el feedback de modo que es apto para el filtro de Kalman:
$$ [x, v_x, y, v_y] $$

\paragraph{Compensación espacial y vectorial} \mbox{} \vspace{10pt}

La compensación espacial y vectorial del robot omnidireccional se logra aprovechando las capacidades del filtro de Kalman para estimar el estado actual del sistema en tiempo real. Con esta información, el sistema puede ajustar los vectores de movimiento en las ruedas, compensando automáticamente irregularidades en el desplazamiento.

La corrección de la \textit{posición} del robot se realiza principalmente a través del filtro de Kalman, que combina el modelo cinemático con las mediciones reales para determinar en todo momento la ubicación exacta del robot en el espacio y su velocidad. Cuando se detectan desviaciones en la posición estimada, el filtro ajusta la estimación para reducir el error acumulado y garantizar un control óptimo.

Por otro lado, la \textit{trayectoria} planeada se corrige utilizando el modelo cinemático del robot. Este modelo define las relaciones entre los vectores de movimiento de las ruedas y los desplazamientos deseados en el espacio. Cuando el filtro de Kalman detecta desviaciones en la posición real, el sistema utiliza esta información para recalcular y corregir la trayectoria, asegurando que el robot siga el camino previamente definido.

\paragraph{Implementación} \mbox{} \vspace{10pt}

\textbf{Compensación espacial} \mbox{} \vspace{10pt}

Para la compensación espacial necesitamos determinar $d_c$, que es la distancia a recorrer para corregir la posición y llegar a $B$.

Definimos la compensación para un desplazamiento en el eye $y$ del mapa, luego se comprueba que el proceso es el mismo para un desplazamiento en $x$. En primer lugar se definen tres puntos claves:

$$ A = (x_1, y_1) : \text{Punto de partida} $$
$$ B = (x_2, y_2) : \text{Punto de llegada} $$
$$ X_a = (x_a, y_a) : \text{Coordenada actual del robot estimada por Kalman} $$
$$ d_c = (d_{xc}, d_{yc}) : \text{Distancia de compensación} $$

\begin{figure}[H]
    \centering
    \includegraphics[width=0.4\linewidth]{images/compensacion_vector_distancia_kalman.png}
    \caption{Diagrama de compensación espacial}
    \label{fig:diagcompespacial}
\end{figure}

Para ello podemos definir:

$$ d_c = \sqrt{ {d_{xc}}^2 + {d_{yc}}^2 } $$
$$ d_{xc} = x_2 - x_a $$
$$ d_{yc} = y_2 - y_a $$

Con lo que podemos calcular $d_c$ para completar la tupla enviada al robot.

Ahora bien, en este punto podemos obtener el valor de $\alpha$:

$$ tan(\alpha) = \frac{d_{xc}}{d_{yc}} $$
$$ \alpha = arctan(\frac{d_{xc}}{d_{yc}}) $$

Es posible comprobar que para un desplazamiento sobre el eje $x$, la diferencia es que se calcula $\alpha$ de la siguiente manera:

$$ \alpha = arctan(\frac{d_{yc}}{d_{xc}}) $$

\textbf{Compensación vectorial} \mbox{} \vspace{10pt}

Para la compensación vectorial, en primera instancia definimos:

$$ V_c = (V_{xc}, V_{yc}) : \text{Vector de compensación} $$

\begin{figure}[H]
    \centering
    \includegraphics[width=0.4\linewidth]{images/compensacion_vector_velocidad_kalman.png}
    \caption{Diagrama de compensación vectorial}
    \label{fig:diagcompvectorial}
\end{figure}

Ahora bien, se establece que el valor de velocidad de compensación $V_c$ es constante, pero aún así necesitamos descomponer el vector para obtener $V_{xc}$ y $V_{yc}$ para enviarlo al robot:

$$ V_{xc} = V_c \cdot sen(\alpha) $$
$$ V_{yc} = V_c \cdot cos(\alpha) $$

Por lo que podemos definir:

$$ tan(\alpha) = \frac{V_{xc}}{V_{yc}} $$
$$ \alpha = arctan(\frac{V_{xc}}{V_{yc}}) $$

Con lo que ya podemos calcular $V_{xc}$ y $V_{yc}$ para enviarlo en la tupla que recibe el robot. En las tuplas del vector de compensación $v_\theta = 0$ dado que nuestro modelo de Kalman no contempla desplazamiento rotacional.

Es posible comprobar que para un desplazamiento sobre el eje $x$, la diferencia es que calculamos $\alpha$ de la siguiente manera:

$$ \alpha = arctan(\frac{V_{yc}}{V_{xc}}) $$

\subsection{Testing y pruebas}

Todas las pruebas realizadas en esta iteración parten del desarrollo realizado hasta el momento del prototipo, la interfaz y el modelo del mapa.

\begin{testtableformat}
    \hline \rowcolor{test_header_color}
        Test ID             & TC\_05\_00 \\
    \hline
        Tipo de test        & Test unitario \\
    \hline
        Objeto de prueba    & Filtro de Kalman \\
    \hline
        Requerimiento       & RF8 - RF6 \\
    \hline
        Nombre              & Estimación del Filtro de Kalman \\
    \hline
        Descripción         & Determinar la estimación del Filtro de Kalman luego de N iteraciones de mediciones, con trayectorias en linea recta sin cambio de dirección \\
    \hline
        Precondición        & PRECOND\_H \\
    \hline
        Pasos del test      & \begin{enumerate}
                                \item Enviar al robot setpoints en linea recta y recolectar las mediciones
                                \item Introducir las mediciones en el Filtro de Kalman y verificar que la estimación realizada se condice con la posición real final del robot
                                \item Repetir desde el paso 1) con diferentes valores
                            \end{enumerate} \\
    \hline
        Resultado esperado  & La estimación hecha por el Filtro de Kalman luego de N iteraciones debe aproximarse a la coordenada real del robot en el espacio \\
    \hline
        Resultado obtenido  & Se obtiene que la estimación de Kalman es correcta con un error de alrededor de $\pm3[cm]$ \\
    \hline
        Observaciones       & - \\
    \hline
\end{testtableformat}


\begin{testtableformat}
    \hline \rowcolor{test_header_color}
        Test ID             & TC\_05\_01 \\
    \hline
        Tipo de test        & Test unitario \\
    \hline
        Objeto de prueba    & Filtro de Kalman \\
    \hline
        Requerimiento       & RF8 - RF6 \\
    \hline
        Nombre              & Estimación del Filtro de Kalman \\
    \hline
        Descripción         & Determinar la estimación del Filtro de Kalman luego de N iteraciones de mediciones, con trayectorias en linea recta con cambio de dirección respetando la restricción de movimiento \\
    \hline
        Precondición        & PRECOND\_H \\
    \hline
        Pasos del test      & \begin{enumerate}
                                \item Enviar al robot setpoints en linea recta con cambios de dirección y recolectar las mediciones
                                \item Introducir las mediciones en el Filtro de Kalman y verificar que la estimación realizada se condice con la posición real final del robot
                                \item Repetir desde el paso 1) con diferentes valores
                            \end{enumerate} \\
    \hline
        Resultado esperado  & La estimación hecha por el Filtro de Kalman luego de N iteraciones debe aproximarse a la coordenada real del robot en el espacio \\
    \hline
        Resultado obtenido  & Se obtiene que la estimación de Kalman se aproxima a la real con un radio de $\pm3[cm]$ en las coordenadas XY \\
    \hline
        Observaciones       & - \\
    \hline
\end{testtableformat}


\begin{testtableformat}
    \hline \rowcolor{test_header_color}
        Test ID             & TC\_05\_02 \\
    \hline
        Tipo de test        & Test de integración \\
    \hline
        Objeto de prueba    & Compensador del Filtro de Kalman \\
    \hline
        Requerimiento       & RF8 - RF6 - RF5 - RF3 \\
    \hline
        Nombre              & Compensación del Filtro de Kalman apagada \\
    \hline
        Descripción         & Se debe tomar un punto de referencia sobre el cual comparar la acción de la compensación del filtro. Realizar compensaciones en base a las mediciones acumuladas sin ser procesadas por el filtro. Se realizan los experimentos con trayectorias en línea recta sin cambio de dirección \\
    \hline
        Precondición        & PRECOND\_G \\
    \hline
        Pasos del test      & \begin{enumerate}
                                \item Enviar al robot setpoints en linea recta y recolectar las mediciones
                                \item Calcular compensaciones sin utilizar el Filtro de Kalman y comprobar la posición final del robot
                                \item Repetir desde el paso 1) con diferentes valores
                            \end{enumerate} \\
    \hline
        Resultado esperado  & El robot debe lograr llegar a destino con un comportamiento similar al obtenido con las pruebas del Modelo Cinemático en línea recta, con su error asociado \\
    \hline
        Resultado obtenido  & Al realizar la trayectoria se observan desviaciones considerables, comparables a las pruebas solo con Modelo Cinemático compensado \\
    \hline
        Observaciones       & - \\
    \hline
\end{testtableformat}


\begin{testtableformat}
    \hline \rowcolor{test_header_color}
        Test ID             & TC\_05\_03 \\
    \hline
        Tipo de test        & Test de integración \\
    \hline
        Objeto de prueba    & Compensador del Filtro de Kalman \\
    \hline
        Requerimiento       & RF8 - RF6 - RF5 - RF3 \\
    \hline
        Nombre              & Compensación del Filtro de Kalman apagada \\
    \hline
        Descripción         & Se debe tomar un punto de referencia sobre el cual comparar la acción de la compensación del filtro. Realizar compensaciones en base a las mediciones acumuladas sin ser procesadas por el filtro. Se realizan los experimentos con trayectorias en línea recta con cambio de dirección \\
    \hline
        Precondición        & PRECOND\_G \\
    \hline
        Pasos del test      & \begin{enumerate}
                                \item Enviar al robot setpoints en linea recta con cambios de dirección y recolectar las mediciones
                                \item Calcular compensaciones sin utilizar el Filtro de Kalman y comprobar la posición final del robot
                                \item Repetir desde el paso 1) con diferentes valores
                            \end{enumerate} \\
    \hline
        Resultado esperado  & El robot debe lograr llegar a destino con un comportamiento similar al obtenido con las pruebas del Modelo Cinemático compensado en línea recta, con su error asociado \\
    \hline
        Resultado obtenido  & Al realizar la trayectoria se observan desviaciones aún mas considerables, especialmente en los momentos de cambio de dirección \\
    \hline
        Observaciones       & - \\
    \hline
\end{testtableformat}


\begin{testtableformat}
    \hline \rowcolor{test_header_color}
        Test ID             & TC\_05\_04 \\
    \hline
        Tipo de test        & Test de integración \\
    \hline
        Objeto de prueba    & Compensador del Filtro de Kalman \\
    \hline
        Requerimiento       & RF8 - RF6 - RF5 - RF3 \\
    \hline
        Nombre              & Compensación del Filtro de Kalman encendida \\
    \hline
        Descripción         & Realizar compensaciones en base a las estimaciones hechas por el filtro. Se realizan los experimentos con trayectorias en línea recta sin cambio de dirección \\
    \hline
        Precondición        & PRECOND\_I \\
    \hline
        Pasos del test      & \begin{enumerate}
                                \item Enviar al robot setpoints en linea recta sin cambios de dirección y recolectar las mediciones
                                \item Introducir cada nueva medición al Filtro de Kalman y obtener el vector de compensación
                                \item Enviar el vector de compensación calculado al robot
                                \item Comprobar que el robot corrige su posición en los momentos donde se envía el vector compensado
                                \item Repetir desde el paso 1) con diferentes valores
                            \end{enumerate} \\
    \hline
        Resultado esperado  & El robot debe lograr llegar a la coordenada destino con una mejor aproximación y menor error, no solo al final, sino que durante todo el recorrido \\
    \hline
        Resultado obtenido  & Se observa una notoria mejoría en la aproximación al punto destino y también es notable que la trayectoria del robot continuamente se intenta aproximar a la ideal durante el recorrido \\
    \hline
        Observaciones       & - \\
    \hline
\end{testtableformat}


\begin{testtableformat}
    \hline \rowcolor{test_header_color}
        Test ID             & TC\_05\_05 \\
    \hline
        Tipo de test        & Test de integración \\
    \hline
        Objeto de prueba    & Compensador del Filtro de Kalman \\
    \hline
        Requerimiento       & RF8 - RF6 - RF5 - RF3 \\
    \hline
        Nombre              & Compensación del Filtro de Kalman encendida \\
    \hline
        Descripción         & Realizar compensaciones en base a las estimaciones hechas por el filtro. Se realizan los experimentos con trayectorias en línea recta con cambio de dirección \\
    \hline
        Precondición        & PRECOND\_I \\
    \hline
        Pasos del test      & \begin{enumerate}
                                \item Enviar al robot setpoints en linea recta con cambios de dirección y recolectar las mediciones
                                \item Introducir cada nueva medición al Filtro de Kalman y obtener el vector de compensación
                                \item Enviar el vector de compensación calculado al robot
                                \item Comprobar que el robot corrige su posición en los momentos donde se envía el vector compensado
                                \item Repetir desde el paso 1) con diferentes valores
                            \end{enumerate} \\
    \hline
        Resultado esperado  & El robot debe lograr llegar a la coordenada destino con una mejor aproximación y menor error, no solo al final, sino que durante todo el recorrido \\
    \hline
        Resultado obtenido  & Se observa que la estimación y la posición real resulta ser más acertada y con un menor error, incluso en los cambios de dirección \\
    \hline
        Observaciones       & - \\
    \hline
\end{testtableformat}


\begin{testtableformat}
    \hline \rowcolor{test_header_color}
        Test ID             & TC\_05\_06 \\
    \hline
        Tipo de test        & Test de sistema \\
    \hline
        Objeto de prueba    & Comunicación inalámbrica - PID - Modelo cinemático compensado - Odometría - Seguidor de línea magnética - Modelo del Mapa - Calculador de trayectorias - Interfaz de usuario - Red de Petri - Monitor - Filtro de Kalman \\
    \hline
        Requerimiento       & RF1 - RF2 - RF3 - RF4 - RF5 - RF6 - RF7 - RF8 - RF10 \\
    \hline
        Nombre              & Prueba de sistema integrado \\
    \hline
        Descripción         & Verificar que la interfaz, el robot y todos los componentes involucrados funcionan de manera adecuada \\
    \hline
        Precondición        & PRECOND\_I \\
    \hline
        Pasos del test      & \begin{enumerate}
                                \item En la interfaz determinar la coordenada origen y destino, calcular la trayectoria y enviar los setpoints
                                \item Comprobar que el robot se mueve a lo largo de la trayectoria definida y al desviarse se corrige su posición
                                \item Repetir desde el paso 1) con diferentes valores
                            \end{enumerate} \\
    \hline
        Resultado esperado  & La interfaz calcula las trayectorias del robot para un determinado par de puntos de origen y destino, las envía al robot y éste realiza las trayectorias. Durante el recorrido el robot reporta información de mediciones que son utilizadas para la estimación y compensación de Kalman. El robot consigue corregir su trayectoria en todo el recorrido y se conoce con precisión su posición. \\
    \hline
        Resultado obtenido  & El robot y la interfaz se comportan de manera esperada. El robot realiza las trayectorias dentro de los límites observados en las pruebas unitarias y de integración hechas anteriormente. \\
    \hline
        Observaciones       & - \\
    \hline
\end{testtableformat}

\subsection{Resultados}

Como principal resultado de esta iteración, el uso del Filtro de Kalman, basado en las entradas proporcionadas por los sensores del sistema y complementado con el PID y el Modelo Cinemático del robot, permitió una mejora drástica en las trayectorias lineales del robot. Estas mejoras se tradujeron en una compensación efectiva que optimiza el desempeño del sistema.

A través de este enfoque, se logró reducir significativamente el error en las trayectorias estimadas, garantizando un alto nivel de precisión en el movimiento del robot. Esto se evidencia en las representaciones gráficas obtenidas, donde la trayectoria deseada, indicada en fucsia, y la trayectoria real estimada, marcada en verde, muestran en la Figura \ref{fig:ejemplofiltrokalman} que se tiene un bajo error para todos los puntos de la misma. Esta trayectoria esta compuesta por la secuencia de coordenadas:

$$ [(3,3), (2,3), (1,3), (1,2), (1,1), (2,1), (3,1)] $$

\begin{figure}[H]
    \centering
    \includegraphics[width=1\linewidth]{images/ejemplo_trayectoria_compensacion_kalman.png}
    \caption{Ejemplo de una trayectoria realizada con el Filtro de Kalman}
    \label{fig:ejemplofiltrokalman}
\end{figure}

Sin embargo, durante las pruebas se identificó que el Filtro de Kalman, al estar implementado en la interfaz de Python, necesita operar a través de la red con un intercambio constante de información con el robot mediante MQTT, generando un cuello de botella en el sistema. A futuro, se considera como una mejora significativa trasladar la ejecución del Filtro de Kalman directamente al robot, eliminando la dependencia de la comunicación por red para la compensación de vectores y distancia.

\subsection{Riesgos superados}
El desarrollo de la presente iteración se corresponde con unas de las mas importantes del proyecto dado que termina de superar riesgos importantes como ser RI-04 y RI-03.

Al mismo tiempo, se logra una integración efectiva entre los nuevos y ya existentes componentes, por lo que se logran progresos sobre el riesgo RI-02.

\begin{center} \begin{tabular}{|c|c|} 
    \hline
        ID & Riesgo \\
    \hline
        RI-02 & Intercomunicación de componentes ineficiente o ineficaz \\
    \hline
        RI-03 & Prestaciones insuficientes de componentes \\
    \hline
        RI-04 & Modificación de los requerimientos del proyecto \\
    \hline
\end{tabular} \end{center}

\subsection{Conclusiones}
En esta iteración del proyecto, se alcanzaron importantes avances que fortalecen tanto la precisión como la estabilidad del sistema de control del robot omnidireccional. La implementación del Filtro de Kalman y proceso de compensación en base a él ha demostrado ser fundamental para mejorar las trayectorias y reducir los errores en el desplazamiento del robot.

\newpage
\section{Iteración 6: Lectura de códigos QR}

\subsection{Introducción}

El procesamiento de imágenes es una disciplina ampliamente utilizada en la actualidad en diversas aplicaciones, que van desde la visión artificial en vehículos autónomos hasta el reconocimiento facial en dispositivos móviles. Su aplicación permite extraer información valiosa de imágenes para la toma de decisiones automatizadas.

En el contexto de esta iteración, nos enfocamos en la captura y análisis de imágenes para la detección de códigos QR. Este proceso es fundamental para la intervención en la navegación del robot, ya que le permite recibir y procesar información del entorno de manera eficiente. La integración de este sistema representa un avance significativo en la capacidad del robot para interactuar con su entorno y mejorar su movilidad.

\subsection{Requerimientos}
En esta iteración abordaremos los siguientes requerimientos funcionales:

\begin{center}
    \begin{tabular} {
        | >{\centering\arraybackslash}m{1cm}
        | >{\centering\arraybackslash}m{13cm} |}
        \hline
            ID & Descripción \\
        \hline
            RF8 & Debe poder ubicarse al robot en el plano de forma precisa \\
        \hline
            RF9 & El robot debe identificar su ambiente mediante el uso de una cámara \\ 
        \hline
    \end{tabular}
\end{center}

    Por otra parte, los requerimientos no funcionales que abordaremos son:

\begin{center}
    \begin{tabular} {
        | >{\centering\arraybackslash}m{1cm}
        | >{\centering\arraybackslash}m{13cm} |}
        \hline
            ID & Descripción \\
        \hline
            RNF1 & Debería tener tiempos de respuesta aceptables para el buen funcionamiento del sistema de control \\
        \hline
    \end{tabular}
\end{center}

\subsection{Desarrollo}

\subsubsection{Mediciones con imágenes}

Para poder corregir la trayectoria del robot de forma dinámica, es decir interfiriendo cual va a ser el vector de movimiento, cuando este se encuentra realizando un recorrido es necesario tener un conocimiento del entorno en el cual se va a mover. El beneficio que trae esta incorporación es poder aumentar la precisión de los movimientos que va a realizar cuando el robot ejecuta un trayecto, en realidad, la denominación correcta sería poder disminuir el error de la trayectoria a efectuar.

Esta medición y corrección en tiempo real se suma a las iteraciones de ajuste de trayectoria mediante la medición de imanes por sensores hall y filtro de Kalman, estas iteraciones por supuesto mejoran el movimiento que efectúa el robot pero también siempre es importante tratar de disminuir el error los mas posible ya que sin ningún tipo de medición sobre el entorno y los movimientos, el único parámetro medible es la distancia recorrida por las ruedas, y como sabemos esto no es suficiente para determinar con precisión la ubicación del robot.

El reconocimiento del entorno de desplazamiento del robot se puede realizar de muchas maneras, desde sensores ultrasónicos hasta cámaras fotográficas, y por supuesto dependiendo cual se selecciona la precisión de la medición va a aumentar o disminuir.
En base a los modelos anteriores del robot que usaron sensores de ultrasonido para determinar la distancia de los objetos que se encuentran en frente del robot, notamos que la precisión de los mismos no nos es suficiente para influir de manera conveniente en la trayectoria. Es por ello que decidimos usar como alternativa una cámara fotográfica, y por ende el procesamiento de imágenes para identificar los elementos que creamos necesarios dentro de la foto para determinar qué información podemos extraer de ahí, cómo interpretarla y finalmente influir en el comportamiento del robot.

El microcontrolador ESP-32 cuenta con un modelo que incorpora la posibilidad de añadirle un cámara de manera extraíble, desafortunadamente no es el mismo modelo el cual maneja la lógica del comportamiento del robot por lo cual fue necesario la incorporación de un nuevo microcontrolador y sumarle un sostén por encima del robot para que este sostenga la cámara y podamos obtener imágenes de manera más cómoda.

\begin{figure}[H]
   \centering
   \includegraphics[width=1.0\linewidth]{images/robot_camara.png}
   \caption{Robot con la ESP32-CAM incorporada}
   \label{fig:robot_camara}
\end{figure}

Con el solo hecho de tomar una fotografía con el microcontrolador, usamos la mayor de su procesamiento en esa tarea. Realizando distintas pruebas con las resoluciones de imágenes que la cámara pone a disposición nos dimos cuenta que hasta es necesario añadirle un opción para habilitar una memoria RAM externa para almacenar las imágenes. Esto nos sirvió para darnos cuenta que no íbamos a poder realizar el procesamiento de las imágenes dentro del microcontrolador y lo mejor sería seguir acoplado al paradigma de IoT el cual nos plantea que este dispositivo solo va a servir para tomar la fotografía y enviarla a otro servidor el cual se va a encargar de procesar la imagen.

\subsubsection{Mediciones con códigos QR}

Dentro de una imagen podemos tener miles de referencias por las cual guiarnos o tomar como objetivo para establecer una referencia de la ubicación de la cámara, una buena alternativa nos pareció el uso de códigos QR que en su payload contengan información relevante sobre la ubicación del objeto, entonces, si dentro de las varias fotos que capturamos realizando una trayectoria podemos identificar códigos QR en ellas, el procesamiento de imágenes se centra en la tarea puntual de identificar los códigos, y por lo tanto es más rápida y menos pesada computacionalmente.

Ya especificamos en secciones anteriores que al mapa del robot lo proyectamos sobre una superficie en dos dimensiones con distintas celdas que podemos identificar mediante un sistema de ejes cartesianos. Teniendo una referencia a gran escala de su ubicación podemos determinar fácilmente en qué parte del mapa se encuentra ya que van a ser un par coordenadas en el eje $X$ y el eje $Y$. Si a este par de coordenadas lo colocamos como payload dentro del código QR, vamos a poder identificar donde se encuentra el robot o al menos tener una referencia real de donde puede llegar a encontrarse.

Esto por supuesto no es suficiente ya que al ser una referencia generalizada, no obtenemos una precisión de la cual confiarnos. Es por ello que decidimos medir la distancia hacia el código QR, tanto en el eje $X$ como en el eje $Y$, sabiendo que si el robot se encuentra posicionado de forma central con el código QR, se encuentra exactamente ubicado donde marca el par de ejes cartesianos.

Para las pruebas y la determinación de los parámetros al calcular una distancia efectiva a las cuales se pueden detectar los códigos QR se tomó un objetivo fijo con distintas ya definidas y a partir de ahí poder determinar cual es la distancia mínima de reconocimiento. Se implemento la siguiente experiencia para determinar los parámetros de la cámara.

\begin{figure}[H]
   \includegraphics[trim={0 0 0 6cm}, clip, width=1.0\linewidth]{images/robot_medicion_qr.png}
   \caption{Experiencia para la obtención de parámetros de calibración de la cámara}
   \label{fig:robot_medicion_qr}
\end{figure}

El objetivo de esta calibración es encontrar la longitud focal, la cual es una constante que representa la distancia desde el centro del lente objetivo hasta el sensor de imagen. Siempre tomando como el ejemplo dimensiones de las fotografías obtenidas por la ESP32 las cuales se detallan en la siguiente imagen y donde el ancho es de $1280px$ mientras tanto que el alto es de $720px$.

\begin{figure}[H]
   \centering
   \includegraphics[width=0.5\linewidth]{images/ejemplo_foto.drawio.png}
   \caption{Ejemplo de las dimensiones tomadas por la cámara}
   \label{fig:qr}
\end{figure}

Por lo tanto se calcula de la siguiente manera \cite{wu2018size}:

\begin{equation}
   Longitud\ focal [px] = \frac{Ancho\ del\ QR\ en\ la\ imagen\ [px] \times Distancia\ real\ [mm]}{Dimensi\acute{o}n\ real\ del\ c\acute{o}digo\ QR\ [mm]}
   \label{ec:logitud_focal}
\end{equation}

Donde:
\begin{itemize}
   \item El ancho del QR en la imagen se representa en cantidad de pixeles.
   \item La distancia real es entre el código QR y la cámara medida en milímetros, en este caso 400$[mm]$.
   \item La dimensión real del código QR es el ancho del código QR, en este caso es de 50$[mm]$.
\end{itemize}

Al consultar el tamaño de pixel dado por la hoja de datos de la cámara OV2640, obtenemos que es de $2.2\mu m$. Por lo que podemos convertir la longitud focal de $[px]$ a $[mm]$ mediante la siguiente expresión:

\begin{equation}
 Longitud\ focal [mm] = Longitud\ focal [px] \times Dimensi\acute{o}n\ del\ pixel [mm/px]
\end{equation}

Una vez obtenido este parámetro lo vamos a utilizar para comparar la dimensión en píxeles del código QR en la imagen con el código QR en sí, y de esa manera obtener la distancia que existe hacia el objeto con la fórmula:

\begin{equation}
   Distancia[mm] = \frac{Dimensi\acute{o}n\ real\ del\ c\acute{o}digo\ QR\ [mm] \times Longitud\ focal\ [px]}{Ancho\ del\ QR\ en\ la\ imagen\ [px]}
   \label{ec:distancia_qr}
\end{equation}

Por supuesto que esto se aplica únicamente cuando en la fotografía se logra identificar un código QR, si durante el procesamiento de la imagen esto no sucede, no se va a poder determinar la distancia. Realizando varias pruebas se logró determinar que si la cámara se encuentra entre los $500mm$ y $600mm$ del objetivo, el código QR es identificable en la imagen. Una distancia mayor al rango descrito ya se considera fuera del rango.

A partir del posicionamiento del robot se obtuvo la siguiente imagen, la cual se usó de referencia para medir la distancia en todas las demás fotografías utilizando las expresiones desarrolladas anteriormente.

\begin{figure}[H]
   \centering
   \includegraphics[width=0.8\linewidth]{images/img_centro.jpg}
   \caption{Imagen de referencia obtenida por el robot}
   \label{fig:img_centro}
\end{figure}

\subsubsection{Intervención en la estimación de posición del robot}

Mencionamos anteriormente que el payload de los códigos QR contienen las coordenadas en el eje $X$e $Y$ en donde se encuentra el QR en el mapa. Al medir la distancia hacia el código lo que estamos haciendo es sumarle precisión a la determinación de la ubicación del robot. Entonces si tomamos como ejemplo ideal a una fotografía donde el código QR se encuentra perfectamente centrado, significa que el robot no presenta desplazamiento alguno el eje $X$.

\begin{figure}[H]
   \centering
   \includegraphics[width=0.7\linewidth]{images/ejemplo_foto_centro.jpg}
   \caption{Ejemplo de una fotografía con el código QR centrado}
   \label{fig:ejemplo_foto_centro}
\end{figure}

Entonces si nos basamos en la Figura \ref{fig:ejemplo_foto_centro} para determinar la posición del robot, obtenemos una posición mas precisa ya que si durante el procesamiento de la fotografía verificamos que el código QR se encuentra desplazado hacía la derecha, el robot va a estar desplazado hacía la izquierda sobre el eje $X$. Y si también el código QR detectado es de mayor dimensión en la imagen, el robot se encuentra posicionado más cerca del código QR.

Además cómo se mencionó mas arriba, el contenido del payload incluye las coordenadas en donde se encuentra el código dentro del plano. Entonces si tenemos el ejemplo donde un robot posicionado en la coordenada $(4,2)$ desea llegar a la coordenada $(4,0)$ donde se encuentra ubicado un código QR con las coordenadas descritas, nos encontramos en la situación en la que, por su posición actual, le es imposible al robot reconocer el código QR por la gran distancia que existe entre él y la cámara del robot, por lo tanto estimamos que el robot mientras realiza desplazamientos entre celda y celda siempre va a terminar posicionado al centro de ella.

\begin{figure}[H]
   \centering
   \includegraphics[width=0.5\linewidth]{images/robot_posicion_0.jpg}
   \caption{Robot posicionado en la coordenada $(4,2)$}
   \label{fig:robot_posicion_0}
\end{figure}

Ahora si el robot se encuentra llegando a su destino, la coordenada $(4,0)$ y se encuentra dentro del rango de distancia de reconocimiento de códigos QR, va a calcular la distancia tanto en el eje $Y$ como en el $X$ y sumarla al payload del código QR para después poder reportar esa coordenada. Entonces en ese momento, así como muestra la figura \ref{fig:robot_posicion_1} el robot en realidad se encuentra ubicado en la coordenada $(4.2,0.5)$ y no el centro de la celda como uno supone cuando no tiene el reporte de su ubicación en todo momento.

\begin{figure}[H]
   \centering
   \includegraphics[width=0.5\linewidth]{images/robot_posicion_1.jpg}
   \caption{Robot posicionado en la coordenada $(4.2,0.5)$}
   \label{fig:robot_posicion_1}
\end{figure}

La siguiente imagen \ref{fig:qrcamararobot} muestra cómo es una fotografía captada en movimiento por el robot y la lectura que hace del payload, cómo se puede observar es una imagen bastante nítida y con buen enfoque lo que permite que el procesamiento de la misma se realice con mayor precisión.

\begin{figure}[H]
   \centering
   \includegraphics[width=0.7\linewidth]{images/qr.png}
   \caption{Imagen obtenida por el robot en movimiento}
   \label{fig:qrcamararobot}
\end{figure}

\subsection{Pruebas y testing}

\begin{testtableformat}
   \hline \rowcolor{test_header_color}
       Test ID             & TC\_06\_00 \\
   \hline
       Tipo de test        & Test unitario \\
   \hline
       Objeto de prueba    & Calibración de la cámara \\
   \hline
       Requerimiento       & RF4 \\
   \hline
       Nombre              & Medición de la distancia hacia un objetivo \\
   \hline
       Descripción         & Lograr capturar la fotografía de un objetivo en un ambiente controlado para tomar obtener los parámetros de calibración de la cámara.\\
   \hline
       Precondición        & PRECOND\_H\\
   \hline
       Pasos del test      & \begin{enumerate}
                             \item Capturar una fotografía con la ESP32-CAM.
                             \item Envíar la imagen por comunicación inalámbrica hacia el servidor que aloja la imagen.
                             \item Verificar que la imagen llegó completa.
                             \item Procesar la imagen para obtener los parámetros de la imagen.
                             \end{enumerate} \\
   \hline
       Resultado esperado  & La imagen debe llegar de forma completa y en el procesamiento se debe detectar el código QR. \\
   \hline
       Resultado obtenido  & Tanto la imagen como el procesamiento fueron obtenidas de forma correcta. \\
   \hline
       Observaciones       & - \\
   \hline
\end{testtableformat}

\begin{testtableformat}
   \hline \rowcolor{test_header_color}
       Test ID             & TC\_06\_01 \\
   \hline
       Tipo de test        & Test unitario \\
   \hline
       Objeto de prueba    & Envío del contenido del payload \\
   \hline
       Requerimiento       & RF4 \\
   \hline
       Nombre              & Coordenadas del payload \\
   \hline
       Descripción         & Capturar una fotografía, detectar si existe un código QR en la imagen y enviar el contenido del payload.\\
   \hline
       Precondición        & PRECOND\_H\\
   \hline
       Pasos del test      & \begin{enumerate}
                             \item Capturar una fotografía con la ESP32-CAM.
                             \item Envíar la imagen por comunicación inalámbrica hacia el servidor que aloja la imagen.
                             \item Verificar que la imagen llegó completa.
                             \item Procesar la imagen para obtener los parámetros de la imagen.
                             \end{enumerate} \\
   \hline
       Resultado esperado  & Se debe poder llegar a leer el payload de forma completa. \\
   \hline
       Resultado obtenido  & El contenido del payload se lee correctamente. \\
   \hline
       Observaciones       & - \\
   \hline
\end{testtableformat}

\begin{testtableformat}
   \hline \rowcolor{test_header_color}
       Test ID             & TC\_06\_02 \\
   \hline
       Tipo de test        & Test integración \\
   \hline
       Objeto de prueba    & Realizar procesamiento de los códigos QR mientras el robot se encuentra en movimiento\\
   \hline
       Nombre              & Captura de fotografías con el robot en movimiento\\
   \hline
       Descripción         & La idea es capturar y enviar las fotografías de los códigos QR que se encuentran dispersos a lo largo del mapa mientras el robot realiza un desplazamiento\\
   \hline
       Precondición        & PRECOND\_H \\
   \hline
       Pasos del test      & \begin{enumerate}
                             \item Validar que se reciben las fotografías
                             \item Inicializar el proceso de desplazamiento del robot
                             \item Capturar y envíar las fotografías tomadas
                             \item Validar que el procesamiento de las fotos es correcto y el payload es legible
                             \end{enumerate} \\
   \hline
       Resultado esperado  & Recibir las fotografías de forma correcta y realizar la lectura y procesamiento del payload\\
   \hline
       Resultado obtenido  & Todas las fotos se reciben de forma completa, es decir, no contienen errores y por lo tanto el procesamiento de los códigos QR es efectiva y la información contenida en su payload es procesable\\
   \hline
       Observaciones       & \\
   \hline
\end{testtableformat}

\begin{testtableformat}
    \hline \rowcolor{test_header_color}
        Test ID             & TC\_06\_03 \\
    \hline
        Tipo de test        & Test de sistema \\
    \hline
        Objeto de prueba    & comunicación inalámbrica - PID - Modelo cinemático compensado - Odometría - Seguidor de línea magnética - Modelo del Mapa - Calculador de trayectorias - Red de Petri - Monitor - Filtro de Kalman - Lectura de códigos QR\\
    \hline
        Nombre              & Prueba de sistema integrado\\
    \hline
        Descripción         & Verificar que la interfaz, el robot y todos los componentes involucrados funcionan de manera adecuada\\
    \hline
        Precondición        & PRECOND\_I \\
    \hline
        Pasos del test      & \begin{enumerate}
                              \item En la interfaz determinar la coordenada origen y destino
                              \item Calcular la trayectorias
                              \item Enviar los setpoints
                              \item Comprobar que el robot se mueve a lo largo de la trayectoria definida, al desviarse se corrige su posición y reporta la lectura de los códigos QR
                              \item Repetir la prueba desde el paso 1 con distintos valores
                              \end{enumerate} \\
    \hline
        Resultado esperado  & El robot pueda completar el desplazamiento definido por el usuario usando la interfaz como medio de control y el robot pueda reportar la lectura de los códigos QR de manera efectiva\\
    \hline
        Resultado obtenido  & El robot y la interfaz se comportan de manera esperada. El robot realiza las trayectorias dentro de los límites observados en las pruebas unitarias y de integración hechas anteriormente\\
    \hline
        Observaciones & Se probó recorridos de hasta 4 metros por limitaciones de espacio\\
    \hline
 \end{testtableformat}

\subsection{Resultados}
Los resultados son satisfactorios ya que se pudo cumplir con creces el objetivo de poder enviar y realizar el procesamiento de las imágenes tomadas por la cámara del microcontrolador ESP-32. Como este microcontrolador sólo se encarga de tomar fotos, pudimos generar imágenes de gran calidad llegando a una resolución HD y a una frecuencia de envío de 2 FPS (Frame per second), es decir, 2 imágenes por segundos. Por supuesto esto no llega a ser un procesamiento de video ya que no se arma un objeto con esas imágenes, solamente se realiza el procesamiento de las mismas de forma particular.
Además no implicó realizar cambios grandes en el sistema principal de movilización del robot, esto porque al estar todo conectado vía las comunicaciones inalámbricas, el sistema explicado puede considerarse como un módulo externo, que, por supuesto ayuda al robot a tomar conocimiento de su entorno.

\subsection{Riesgos superados}
\begin{center}
    \begin{tabular} {
        | c| c |}
        \hline \rowcolor{test_header_color}
            ID & Riesgo \\
        \hline
            RI-02 & Intercomunicación de componentes ineficiente o ineficaz \\
        \hline
            RI-03 & Prestaciones insuficientes de componentes \\
        \hline
    \end{tabular}
\end{center}

\subsection{Conclusiones}
En este capítulo se explicaron los motivos que nos llevaron a implementar este tipo de procesamiento y porque consideramos que el conocimiento del entorno y sus parámetros son importantes para mejorar la movilización del robot en un entorno real. Lamentablemente, los tiempos del proyecto no fueron suficientes para realizar todas las pruebas necesarias y verificar fehacientemente que el comportamiento del robot cambian para mejor con la ayuda de las mediciones tomadas de las imágenes, esto teniendo en cuenta que no se añaden de forma directa ni se suman como entrada al filtro de Kalman.
Esperamos que una próxima iteración de la serie Hermes esto se pueda continuar, y plasmar en el mundo real lo que planteamos en la teoría de la implementación.
