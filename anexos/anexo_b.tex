\newpage
\section{Anexo B: Repositorios usados en el proyecto}

A continuación se listan los repositorios utilizados y desarrollados durante el presente trabajo. En ellos se encuentra el código fuente para la implementación, simulación y validación de los algoritmos descritos en el informe:

\begin{itemize}
  \item \textbf{Firmware del robot compatible con el microcontrolador ESP32}:
  \begin{itemize}
    \item URL: \url{https://github.com/facu-tolay/PPS-Robot}
    \item Descripción: Contiene todo el firmware para que el robot pueda funcionar
  \end{itemize}

    \item \textbf{Interfaz gráfica para controlar los movimientos del robot}:
  \begin{itemize}
    \item URL: \url{https://github.com/francovaira/RdPMonitor}
    \item Descripción: Contiene el código en Python que gestiona la creación del mapa, la generación de redes de Petri, el Monitor, el calculador de trayectorias y la interfaz gráfica para el control de los robots
  \end{itemize}

    \item \textbf{Firmware para el envío de fotografías compatible con el microcontrolador ESP32-CAM}:
  \begin{itemize}
    \item URL: \url{https://github.com/facu-tolay/ModuloCamara}
    \item Descripción: Contiene el código en C que hace uso del módulo de la cámara del microcontrolador ESP32 para capturas fotografías cada 2 segundos y enviarlas por la red
  \end{itemize}

    \item \textbf{Interfaz de procesamiento de códigos QR}:
  \begin{itemize}
    \item URL: \url{https://github.com/facu-tolay/ReconocimientoQR}
    \item Descripción: Contiene el código en Python para la interfaz que recibe los códigos QR, realiza los cálculos para las mediciones y envía reportes de las mismas.
  \end{itemize}

    \item \textbf{Repositorio que contiene el informe del proyecto}:
  \begin{itemize}
    \item URL: \url{https://github.com/francovaira/informe-robot-omnidireccional}
    \item Descripción: Contiene el informe de este Proyector Integrador
  \end{itemize}
\end{itemize}