\newpage
\chapter*{Motivación}

En el ámbito industrial, los robots autónomos han demostrado una capacidad sobresaliente para ubicarse en el espacio y resolver conflictos emergentes de manera eficiente. Sin embargo, dichos sistemas presentan barreras significativas para el acceso por parte del usuario final y carecen de una documentación adecuada, lo que limita su integración en sectores más amplios que no sean exclusivamente industriales. Este trabajo se propone investigar y desarrollar un prototipo de robot autónomo accesible y versátil, diseñado para operar en distintos ámbitos como la medicina, el rescate y el uso doméstico. La motivación principal de este proyecto radica en democratizar la tecnología robótica, promoviendo su uso en contextos donde actualmente es limitada o inexistente.

El desarrollo de robots autónomos enfrenta un desafío importante debido a los elevados costos asociados con los sistemas existentes. Sin embargo, los avances tecnológicos recientes han abierto la posibilidad de diseñar prototipos funcionales utilizando componentes disponibles en el mercado, con un diseño simplificado y económico que minimiza el número de piezas mecánicas. Esto contribuye no solo a la reducción de los costos iniciales, sino también a una disminución significativa en los requerimientos de mantenimiento, lo que hace que la solución sea más sostenible y escalable en el largo plazo.

El estudio se centra en el comportamiento y modelado de sistemas compuestos por múltiples robots, los cuales seguirán trayectorias calculadas previamente mediante algoritmos de planificación de rutas. Esto permite investigar técnicas avanzadas de navegación autónoma y coordinación, además de abordar desafíos clave como la prevención de colisiones y la optimización de movimientos en entornos compartidos. La implementación de estas técnicas en un prototipo funcional representa una contribución relevante al campo de la robótica móvil, con un alto potencial para ser aplicada en diversos escenarios prácticos.

En resumen, este proyecto responde a la necesidad de desarrollar soluciones robóticas accesibles y eficientes que puedan ser adoptadas en sectores variados y de forma práctica. La combinación de avances tecnológicos con un enfoque en la reducción de costos y la simplificación del diseño no solo busca avanzar en el conocimiento técnico, sino también facilitar la integración de esta tecnología en contextos donde actualmente no está disponible, ampliando así su impacto y utilidad.

% En la industria existen robots autónomos que son capaces de ubicarse en el espacio donde se mueven y poder resolver conflictos emergentes. Estos robots no suelen ser de disponibilidad a un usuario final ni tampoco hay mucha documentación al respecto, por lo que una de nuestras mayores motivaciones es investigar y desarrollar un robot de similares características que pueda ser utilizado en diversos sectores, como ser medicina, rescate, doméstico, etc.

% Al mismo tiempo es importante notar que los robots ya existentes tienen un alto costo, pero por otro lado la tecnología ha avanzado lo suficiente para que el proyecto sea factible y accesible. Nos motiva lograr un prototipo funcional construido con componentes de alta disponibilidad en el mercado, de bajo costo, y con pocas piezas mecánicas, disminuyendo así el mantenimiento requerido.

% En un nivel más alto este trabajo se centra en el estudio del comportamiento y modelado de un sistema compuesto por varios robots que tienen que seguir cada uno una trayectoria calculada previamente utilizando algoritmos de planificación de rutas.