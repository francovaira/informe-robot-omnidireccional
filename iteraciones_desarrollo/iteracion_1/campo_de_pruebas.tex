\subsubsection{Campo de pruebas}

En cualquier estudio experimental, es vital controlar todas las variables posibles para aislar el efecto de las variables independientes. La superficie sobre la que se desplaza el robot puede influir significativamente en su comportamiento, ya que las irregularidades de la misma pueden inducir deslizamientos o atascos.

La elección de una superficie determinada, consistente y constante en cada experimento es crucial para asegurar la fiabilidad de los resultados obtenidos, al proporcionar un entorno controlado. Mantener una superficie constante permite replicar las condiciones experimentales y comparar resultados de manera precisa y predecible. Si la superficie varía, se introducen variables adicionales que pueden afectar el desempeño del robot, dificultando la comparación entre experimentos.

En primera instancia, la superficie sobre la cual comenzamos las pruebas del prototipo fue el suelo del Laboratorio. En los sucesivos experimentos notamos cierta inconsistencia entre cada iteración, manteniendo los parámetros de funcionamiento constantes. En búsqueda de mejorar la consistencia entre los experimentos, hicimos algunas pruebas sobre un tablón de madera lisa y notamos que la mejoría es sustancial. En consecuencia, se dispone de una superficie de madera sobre la cual realizamos las subsiguientes pruebas.
