\subsection{Testing y pruebas}

Todas las pruebas realizadas en esta iteración parten del desarrollo realizado hasta el momento del prototipo, la interfaz y el modelo del mapa.

\begin{testtableformat}
    \hline \rowcolor{test_header_color}
        Test ID             & TC\_05\_00 \\
    \hline
        Tipo de test        & Test unitario \\
    \hline
        Objeto de prueba    & Filtro de Kalman \\
    \hline
        Requerimiento       & RF8 - RF6 \\
    \hline
        Nombre              & Estimación del Filtro de Kalman \\
    \hline
        Descripción         & Determinar la estimación del Filtro de Kalman luego de N iteraciones de mediciones, con trayectorias en linea recta sin cambio de dirección \\
    \hline
        Precondición        & PRECOND\_H \\
    \hline
        Pasos del test      & \begin{enumerate}
                                \item Enviar al robot setpoints en linea recta y recolectar las mediciones
                                \item Introducir las mediciones en el Filtro de Kalman y verificar que la estimación realizada se condice con la posición real final del robot
                                \item Repetir desde el paso 1) con diferentes valores
                            \end{enumerate} \\
    \hline
        Resultado esperado  & La estimación hecha por el Filtro de Kalman luego de N iteraciones debe aproximarse a la coordenada real del robot en el espacio \\
    \hline
        Resultado obtenido  & Se obtiene que la estimación de Kalman es correcta con un error de alrededor de $\pm3[cm]$ \\
    \hline
        Observaciones       & - \\
    \hline
\end{testtableformat}


\begin{testtableformat}
    \hline \rowcolor{test_header_color}
        Test ID             & TC\_05\_01 \\
    \hline
        Tipo de test        & Test unitario \\
    \hline
        Objeto de prueba    & Filtro de Kalman \\
    \hline
        Requerimiento       & RF8 - RF6 \\
    \hline
        Nombre              & Estimación del Filtro de Kalman \\
    \hline
        Descripción         & Determinar la estimación del Filtro de Kalman luego de N iteraciones de mediciones, con trayectorias en linea recta con cambio de dirección respetando la restricción de movimiento \\
    \hline
        Precondición        & PRECOND\_H \\
    \hline
        Pasos del test      & \begin{enumerate}
                                \item Enviar al robot setpoints en linea recta con cambios de dirección y recolectar las mediciones
                                \item Introducir las mediciones en el Filtro de Kalman y verificar que la estimación realizada se condice con la posición real final del robot
                                \item Repetir desde el paso 1) con diferentes valores
                            \end{enumerate} \\
    \hline
        Resultado esperado  & La estimación hecha por el Filtro de Kalman luego de N iteraciones debe aproximarse a la coordenada real del robot en el espacio \\
    \hline
        Resultado obtenido  & Se obtiene que la estimación de Kalman se aproxima a la real con un radio de $\pm3[cm]$ en las coordenadas XY \\
    \hline
        Observaciones       & - \\
    \hline
\end{testtableformat}


\begin{testtableformat}
    \hline \rowcolor{test_header_color}
        Test ID             & TC\_05\_02 \\
    \hline
        Tipo de test        & Test de integración \\
    \hline
        Objeto de prueba    & Compensador del Filtro de Kalman \\
    \hline
        Requerimiento       & RF8 - RF6 - RF5 - RF3 \\
    \hline
        Nombre              & Compensación del Filtro de Kalman apagada \\
    \hline
        Descripción         & Se debe tomar un punto de referencia sobre el cual comparar la acción de la compensación del filtro. Realizar compensaciones en base a las mediciones acumuladas sin ser procesadas por el filtro. Se realizan los experimentos con trayectorias en línea recta sin cambio de dirección \\
    \hline
        Precondición        & PRECOND\_G \\
    \hline
        Pasos del test      & \begin{enumerate}
                                \item Enviar al robot setpoints en linea recta y recolectar las mediciones
                                \item Calcular compensaciones sin utilizar el Filtro de Kalman y comprobar la posición final del robot
                                \item Repetir desde el paso 1) con diferentes valores
                            \end{enumerate} \\
    \hline
        Resultado esperado  & El robot debe lograr llegar a destino con un comportamiento similar al obtenido con las pruebas del Modelo Cinemático en línea recta, con su error asociado \\
    \hline
        Resultado obtenido  & Al realizar la trayectoria se observan desviaciones considerables, comparables a las pruebas solo con Modelo Cinemático compensado \\
    \hline
        Observaciones       & - \\
    \hline
\end{testtableformat}


\begin{testtableformat}
    \hline \rowcolor{test_header_color}
        Test ID             & TC\_05\_03 \\
    \hline
        Tipo de test        & Test de integración \\
    \hline
        Objeto de prueba    & Compensador del Filtro de Kalman \\
    \hline
        Requerimiento       & RF8 - RF6 - RF5 - RF3 \\
    \hline
        Nombre              & Compensación del Filtro de Kalman apagada \\
    \hline
        Descripción         & Se debe tomar un punto de referencia sobre el cual comparar la acción de la compensación del filtro. Realizar compensaciones en base a las mediciones acumuladas sin ser procesadas por el filtro. Se realizan los experimentos con trayectorias en línea recta con cambio de dirección \\
    \hline
        Precondición        & PRECOND\_G \\
    \hline
        Pasos del test      & \begin{enumerate}
                                \item Enviar al robot setpoints en linea recta con cambios de dirección y recolectar las mediciones
                                \item Calcular compensaciones sin utilizar el Filtro de Kalman y comprobar la posición final del robot
                                \item Repetir desde el paso 1) con diferentes valores
                            \end{enumerate} \\
    \hline
        Resultado esperado  & El robot debe lograr llegar a destino con un comportamiento similar al obtenido con las pruebas del Modelo Cinemático compensado en línea recta, con su error asociado \\
    \hline
        Resultado obtenido  & Al realizar la trayectoria se observan desviaciones aún mas considerables, especialmente en los momentos de cambio de dirección \\
    \hline
        Observaciones       & - \\
    \hline
\end{testtableformat}


\begin{testtableformat}
    \hline \rowcolor{test_header_color}
        Test ID             & TC\_05\_04 \\
    \hline
        Tipo de test        & Test de integración \\
    \hline
        Objeto de prueba    & Compensador del Filtro de Kalman \\
    \hline
        Requerimiento       & RF8 - RF6 - RF5 - RF3 \\
    \hline
        Nombre              & Compensación del Filtro de Kalman encendida \\
    \hline
        Descripción         & Realizar compensaciones en base a las estimaciones hechas por el filtro. Se realizan los experimentos con trayectorias en línea recta sin cambio de dirección \\
    \hline
        Precondición        & PRECOND\_I \\
    \hline
        Pasos del test      & \begin{enumerate}
                                \item Enviar al robot setpoints en linea recta sin cambios de dirección y recolectar las mediciones
                                \item Introducir cada nueva medición al Filtro de Kalman y obtener el vector de compensación
                                \item Enviar el vector de compensación calculado al robot
                                \item Comprobar que el robot corrige su posición en los momentos donde se envía el vector compensado
                                \item Repetir desde el paso 1) con diferentes valores
                            \end{enumerate} \\
    \hline
        Resultado esperado  & El robot debe lograr llegar a la coordenada destino con una mejor aproximación y menor error, no solo al final, sino que durante todo el recorrido \\
    \hline
        Resultado obtenido  & Se observa una notoria mejoría en la aproximación al punto destino y también es notable que la trayectoria del robot continuamente se intenta aproximar a la ideal durante el recorrido \\
    \hline
        Observaciones       & - \\
    \hline
\end{testtableformat}


\begin{testtableformat}
    \hline \rowcolor{test_header_color}
        Test ID             & TC\_05\_05 \\
    \hline
        Tipo de test        & Test de integración \\
    \hline
        Objeto de prueba    & Compensador del Filtro de Kalman \\
    \hline
        Requerimiento       & RF8 - RF6 - RF5 - RF3 \\
    \hline
        Nombre              & Compensación del Filtro de Kalman encendida \\
    \hline
        Descripción         & Realizar compensaciones en base a las estimaciones hechas por el filtro. Se realizan los experimentos con trayectorias en línea recta con cambio de dirección \\
    \hline
        Precondición        & PRECOND\_I \\
    \hline
        Pasos del test      & \begin{enumerate}
                                \item Enviar al robot setpoints en linea recta con cambios de dirección y recolectar las mediciones
                                \item Introducir cada nueva medición al Filtro de Kalman y obtener el vector de compensación
                                \item Enviar el vector de compensación calculado al robot
                                \item Comprobar que el robot corrige su posición en los momentos donde se envía el vector compensado
                                \item Repetir desde el paso 1) con diferentes valores
                            \end{enumerate} \\
    \hline
        Resultado esperado  & El robot debe lograr llegar a la coordenada destino con una mejor aproximación y menor error, no solo al final, sino que durante todo el recorrido \\
    \hline
        Resultado obtenido  & Se observa que la estimación y la posición real resulta ser más acertada y con un menor error, incluso en los cambios de dirección \\
    \hline
        Observaciones       & - \\
    \hline
\end{testtableformat}


\begin{testtableformat}
    \hline \rowcolor{test_header_color}
        Test ID             & TC\_05\_06 \\
    \hline
        Tipo de test        & Test de sistema \\
    \hline
        Objeto de prueba    & Comunicación inalámbrica - PID - Modelo cinemático compensado - Odometría - Seguidor de línea magnética - Modelo del Mapa - Calculador de trayectorias - Interfaz de usuario - Red de Petri - Monitor - Filtro de Kalman \\
    \hline
        Requerimiento       & RF1 - RF2 - RF3 - RF4 - RF5 - RF6 - RF7 - RF8 - RF10 \\
    \hline
        Nombre              & Prueba de sistema integrado \\
    \hline
        Descripción         & Verificar que la interfaz, el robot y todos los componentes involucrados funcionan de manera adecuada \\
    \hline
        Precondición        & PRECOND\_I \\
    \hline
        Pasos del test      & \begin{enumerate}
                                \item En la interfaz determinar la coordenada origen y destino, calcular la trayectoria y enviar los setpoints
                                \item Comprobar que el robot se mueve a lo largo de la trayectoria definida y al desviarse se corrige su posición
                                \item Repetir desde el paso 1) con diferentes valores
                            \end{enumerate} \\
    \hline
        Resultado esperado  & La interfaz calcula las trayectorias del robot para un determinado par de puntos de origen y destino, las envía al robot y éste realiza las trayectorias. Durante el recorrido el robot reporta información de mediciones que son utilizadas para la estimación y compensación de Kalman. El robot consigue corregir su trayectoria en todo el recorrido y se conoce con precisión su posición. \\
    \hline
        Resultado obtenido  & El robot y la interfaz se comportan de manera esperada. El robot realiza las trayectorias dentro de los límites observados en las pruebas unitarias y de integración hechas anteriormente. \\
    \hline
        Observaciones       & - \\
    \hline
\end{testtableformat}