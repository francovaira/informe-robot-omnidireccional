\newpage
\subsection{Pruebas y testing}

En cada una de las iteraciones del proyecto se detallan pruebas unitarias para cada módulo individual, las pruebas de integración entre módulos y las pruebas de sistema. Por otra parte, cada test se vincula con uno o más requerimientos. \cite{sommerville_ingenieria}

\begin{testtableformat}
    \hline \rowcolor{test_header_color}
        Test ID             & TC\_01\_00 \\
    \hline
        Tipo de test        & Test unitario \\
    \hline
        Objeto de prueba    & Determinar cuanto se desplazó el robot\\
    \hline
        Requerimiento       & RF3\\
    \hline
        Nombre              & Cálculo de la distancia recorrida \\
    \hline
        Descripción         & Se pretende calcular la distancia recorrida por el robot contando las cantidad de ranuras que se atraviesan utilizando el contador de pulsos del microcontrolador ESP32 \\
    \hline
        Precondición        & PRECOND\_A \\
    \hline
        Pasos del test      & \begin{enumerate}
                                \item Crear un programa que habilite el contador de pulsos incorporado en el microcontrolador ESP32
                                \item Compilar el proyecto
                                \item Grabar el programa en el microcontrolador
                                \item Contar la cantidad de ranuras que atraviesa el opto acoplador y mediante la formula $distancia\_recorrida = 2.\pi.r\ /\ ranuras\_del\_encoder$ obtener la distancia recorrida por la rueda del robot
                            \end{enumerate} \\
    \hline
        Resultado esperado  & Obtener de forma precisa el desplazamiento realizado por la rueda del robot \\
    \hline
        Resultado obtenido  & Se pudo determinar cuanto es el desplazamiento de la rueda mediante el contador de pulsos pero en frecuencias baja pierde mucha precisión por lo que se decidió realizar varias lecturas de la medición y realizar el promedio de las mismas \\
    \hline
        Observaciones       & - \\
    \hline
 \end{testtableformat}

\begin{testtableformat}
   \hline \rowcolor{test_header_color}
       Test ID             & TC\_01\_01 \\
   \hline
       Tipo de test        & Test unitario \\
   \hline
       Objeto de prueba    & Ejecutar un programa simple en el microcontrolador ESP32 \\
   \hline
       Requerimiento       & RF6 \\
   \hline
       Nombre              & Programa en C que logra establecer conexión WiFi \\
   \hline
       Descripción         & Crear, compilar y grabar un programa simple que logre establecer una conexión inalambrica usando el protocolo WiFi \\
   \hline
       Precondición        & PRECOND\_A \\
   \hline
       Pasos del test      & \begin{enumerate}
                               \item Crear un proyecto en C para el microcontrolador ESP32 usando la librería provista por Espressif
                               \item Compilar el proyecto
                               \item Grabar el programa en el microcontrolador
                               \item Establecer una conexión inalambrica WiFi
                           \end{enumerate} \\
   \hline
       Resultado esperado  &  El microcontrolador debe establecer la conexión inalámbrica sin errores \\
   \hline
       Resultado obtenido  & El microcontrolador establece la conexión WiFi \\
   \hline
       Observaciones       & - \\
   \hline
\end{testtableformat}