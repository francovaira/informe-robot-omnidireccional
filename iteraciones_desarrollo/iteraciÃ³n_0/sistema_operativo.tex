\subsubsection{Sistema Operativo} \mbox{} \vspace{10pt} \\
La decisión de utilizar FreeRTOS como sistema operativo para el desarrollo de este proyecto se fundamenta en varias consideraciones claves. Primero, FreeRTOS está integrado de manera nativa en el framework ESP-IDF de Espressif, lo que facilita enormemente el desarrollo y la integración de funcionalidades en tiempo real. Al utilizar la API de Espressif, se aprovechan componentes y servicios que ya están optimizados para trabajar con FreeRTOS, lo que se traduce en una mayor estabilidad y rendimiento en aplicaciones críticas.

FreeRTOS permite una gestión eficiente de tareas, permitiendo la concurrencia y la sincronización de procesos de manera robusta y controlada. Esto es esencial para aplicaciones que requieren una respuesta rápida a eventos externos, como la gestión de sensores, la comunicación inalámbrica y la realización de múltiples operaciones simultáneamente. La modularidad y escalabilidad que ofrece FreeRTOS permite desarrollar soluciones complejas sin incurrir en sobre costos de recursos, lo que es especialmente valioso en sistemas embebidas con limitaciones de hardware.

El uso de FreeRTOS está respaldado por un amplio ecosistema de documentación y soporte, facilitando el desarrollo, la depuración y el mantenimiento de aplicaciones en el microcontrolador ESP32. Esta sinergia entre el hardware y el sistema operativo garantiza una integración fluida y optimizada para las exigencias de aplicaciones modernas en sistemas embebidos y soluciones IoT, lo que se ajusta perfectamente al desarrollo de este proyecto.