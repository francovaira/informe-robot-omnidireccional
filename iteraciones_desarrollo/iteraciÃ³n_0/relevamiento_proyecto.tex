\subsection{Relevamiento del Proyecto Integrador Hermes III}

El robot Hermes III es la última versión de la familia de robots desarrollada dentro del laboratorio de arquitectura de computadoras. Este también incorpora sistemas del modelo anterior y plantea una mejora sobre ellos, en este caso sobre el sistema de control que comanda el robot móvil y la implementación de un sistema operativo robótico (ROS) el cual es un framework para Linux que está diseñado exclusivamente para el desarrollo de robots. \cite{micolini2022hermes}

El uso del sistema operativo ROS abre la posibilidad de usar software dedicado al desarrollo de robots móviles y también a la incorporación de componentes como ser el módulo de cámara Kinect desarrollado por Microsoft para la consola Xbox.

El conjunto ya mencionado da paso a la implementación e innovación más importante del proyecto, la localización y mapeo en simultáneo (SLAM), brindándole al robot la posibilidad de moverse sobre un superficie y crear un mapa de su entorno, provocando que en cada nueva iteración pueda mejorar su desplazamiento y localización en la superficie de movimiento.

El software se ejecuta sobre una placa NVIDIA Jetson TK1, la cual ofrece un gran poder de cómputo que es necesario para sostener y correr de forma efectiva tanto el sistema operativo como el algoritmo de SLAM. La desventaja es su alto precio, lo que aumenta el costo de construcción del robot y por lo tanto disminuye la posibilidad de pensar en un flota de estos equipos.