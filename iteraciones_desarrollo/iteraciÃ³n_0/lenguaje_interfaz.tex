\subsubsection{Elección de lenguaje para interfaz} \mbox{} \vspace{10pt} \\
En cuanto a las elección del lenguaje se refiere, teníamos dos caminos para elegir, ya que nos encontrábamos limitados por el microcontrolador que soporta tanto C como microPython. Habiendo hecho un análisis y una lectura de las distintas APIs que el fabricante nos ofrece, optamos por C ya que es un lenguaje conocido por nosotros. Esto porque durante el transcurso de la carrera nos tocó desarrollar varios trabajos prácticos en este lenguaje, por lo que sentíamos que no necesitabamos adquirir muchos nuevos conocimientos, es un entorno cómodo para trabajar y la documentación es que se encuentra es amplia por lo que los problemas que podían llegar a surgir los íbamos a poder sortear con cierta facilidad.
En el escenario de seguimiento del robot nos encontramos con el desafió de realizar una interfaz amigable, entendible y que pueda desarrollarse de la forma mas rápida posible sin la necesidad de tener una curva de aprendizaje muy grande. Es por ello que optamos por Python para poder cumplir todos los requisitos ya mencionados. Ambos integrantes ya habíamos trabajado con este lenguaje, y ademas, últimamente, se lo encuentra seguido entre los lenguajes mas utilizados por toda la comunidad de la programación, esto por ser un lenguaje sumamente adaptable ante cualquier necesidad y contar con una variedad muy extensa de librerías. Nos pareció una buena elección que nos daba la opción de sortear cualquier obstáculo que surgiera en el camino.