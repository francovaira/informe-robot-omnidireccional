\subsection{Requerimientos}

En esta iteración abordaremos los siguientes requerimientos funcionales:

\begin{center}
    \begin{tabular} {
        | >{\centering\arraybackslash}m{1cm}
        | >{\centering\arraybackslash}m{13cm} |}
        \hline \rowcolor{test_header_color}
            ID & Descripción \\
        \hline
            RF4 & El robot debe poder realizar trayectorias en línea recta y curvas. \\
        \hline
            RF6 & El robot debe recibir y enviar información mediante comunicaciones inalámbricas. \\
        \hline
            RF8 & Debe poder ubicarse al robot en el plano de forma precisa. \\
        \hline
    \end{tabular}
\end{center}

   Por otra parte, el requerimiento no funcional que abordaremos es:

\begin{center}
    \begin{tabular} {
        | >{\centering\arraybackslash}m{1cm}
        | >{\centering\arraybackslash}m{13cm} |}
        \hline \rowcolor{test_header_color}
            ID & Descripción \\
        \hline
            RNF1 & Debería tener tiempos de respuesta aceptables para el buen funcionamiento del sistema de control. \\
        \hline
    \end{tabular}
\end{center}