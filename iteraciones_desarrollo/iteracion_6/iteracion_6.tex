\newpage
\section{Iteración 6: Lectura de códigos QR}

\subsection{Introducción}

El procesamiento de imágenes es una disciplina ampliamente utilizada en la actualidad en diversas aplicaciones, que van desde la visión artificial en vehículos autónomos hasta el reconocimiento facial en dispositivos móviles. Su aplicación permite extraer información valiosa de imágenes para la toma de decisiones automatizadas.

En el contexto de esta iteración, nos enfocamos en la captura y análisis de imágenes para la detección de códigos QR. Este proceso es fundamental para la intervención en la navegación del robot, ya que le permite recibir y procesar información del entorno de manera eficiente. La integración de este sistema representa un avance significativo en la capacidad del robot para interactuar con su entorno y mejorar su movilidad.

\subsection{Requerimientos}
En esta iteración abordaremos los siguientes requerimientos funcionales:

\begin{center}
    \begin{tabular} {
        | >{\centering\arraybackslash}m{1cm}
        | >{\centering\arraybackslash}m{13cm} |}
        \hline
            ID & Descripción \\
        \hline
            RF8 & Debe poder ubicarse al robot en el plano de forma precisa \\
        \hline
            RF9 & El robot debe identificar su ambiente mediante el uso de una cámara \\ 
        \hline
    \end{tabular}
\end{center}

    Por otra parte, los requerimientos no funcionales que abordaremos son:

\begin{center}
    \begin{tabular} {
        | >{\centering\arraybackslash}m{1cm}
        | >{\centering\arraybackslash}m{13cm} |}
        \hline
            ID & Descripción \\
        \hline
            RNF1 & Debería tener tiempos de respuesta aceptables para el buen funcionamiento del sistema de control \\
        \hline
    \end{tabular}
\end{center}

\subsection{Desarrollo}

\subsubsection{Mediciones con imágenes}

Para poder corregir la trayectoria del robot de forma dinámica, es decir interfiriendo cual va a ser el vector de movimiento, cuando este se encuentra realizando un recorrido es necesario tener un conocimiento del entorno en el cual se va a mover. El beneficio que trae esta incorporación es poder aumentar la precisión de los movimientos que va a realizar cuando el robot ejecuta un trayecto, en realidad, la denominación correcta sería poder disminuir el error de la trayectoria a efectuar.

Esta medición y corrección en tiempo real se suma a las iteraciones de ajuste de trayectoria mediante la medición de imanes por sensores hall y filtro de Kalman, estas iteraciones por supuesto mejoran el movimiento que efectúa el robot pero también siempre es importante tratar de disminuir el error los mas posible ya que sin ningún tipo de medición sobre el entorno y los movimientos, el único parámetro medible es la distancia recorrida por las ruedas, y como sabemos esto no es suficiente para determinar con precisión la ubicación del robot.

El reconocimiento del entorno de desplazamiento del robot se puede realizar de muchas maneras, desde sensores ultrasónicos hasta cámaras fotográficas, y por supuesto dependiendo cual se selecciona la precisión de la medición va a aumentar o disminuir.
En base a los modelos anteriores del robot que usaron sensores de ultrasonido para determinar la distancia de los objetos que se encuentran en frente del robot, notamos que la precisión de los mismos no nos es suficiente para influir de manera conveniente en la trayectoria. Es por ello que decidimos usar como alternativa una cámara fotográfica, y por ende el procesamiento de imágenes para identificar los elementos que creamos necesarios dentro de la foto para determinar qué información podemos extraer de ahí, cómo interpretarla y finalmente influir en el comportamiento del robot.

El microcontrolador ESP-32 cuenta con un modelo que incorpora la posibilidad de añadirle un cámara de manera extraíble, desafortunadamente no es el mismo modelo el cual maneja la lógica del comportamiento del robot por lo cual fue necesario la incorporación de un nuevo microcontrolador y sumarle un sostén por encima del robot para que este sostenga la cámara y podamos obtener imágenes de manera más cómoda.

\begin{figure}[H]
   \centering
   \includegraphics[width=1.0\linewidth]{images/robot_camara.png}
   \caption{Robot con la ESP32-CAM incorporada}
   \label{fig:robot_camara}
\end{figure}

Con el solo hecho de tomar una fotografía con el microcontrolador, usamos la mayor de su procesamiento en esa tarea. Realizando distintas pruebas con las resoluciones de imágenes que la cámara pone a disposición nos dimos cuenta que hasta es necesario añadirle un opción para habilitar una memoria RAM externa para almacenar las imágenes. Esto nos sirvió para darnos cuenta que no íbamos a poder realizar el procesamiento de las imágenes dentro del microcontrolador y lo mejor sería seguir acoplado al paradigma de IoT el cual nos plantea que este dispositivo solo va a servir para tomar la fotografía y enviarla a otro servidor el cual se va a encargar de procesar la imagen.

\subsubsection{Mediciones con códigos QR}

Dentro de una imagen podemos tener miles de referencias por las cual guiarnos o tomar como objetivo para establecer una referencia de la ubicación de la cámara, una buena alternativa nos pareció el uso de códigos QR que en su payload contengan información relevante sobre la ubicación del objeto, entonces, si dentro de las varias fotos que capturamos realizando una trayectoria podemos identificar códigos QR en ellas, el procesamiento de imágenes se centra en la tarea puntual de identificar los códigos, y por lo tanto es más rápida y menos pesada computacionalmente.

Ya especificamos en secciones anteriores que al mapa del robot lo proyectamos sobre una superficie en dos dimensiones con distintas celdas que podemos identificar mediante un sistema de ejes cartesianos. Teniendo una referencia a gran escala de su ubicación podemos determinar fácilmente en qué parte del mapa se encuentra ya que van a ser un par coordenadas en el eje $X$ y el eje $Y$. Si a este par de coordenadas lo colocamos como payload dentro del código QR, vamos a poder identificar donde se encuentra el robot o al menos tener una referencia real de donde puede llegar a encontrarse.

Esto por supuesto no es suficiente ya que al ser una referencia generalizada, no obtenemos una precisión de la cual confiarnos. Es por ello que decidimos medir la distancia hacia el código QR, tanto en el eje $X$ como en el eje $Y$, sabiendo que si el robot se encuentra posicionado de forma central con el código QR, se encuentra exactamente ubicado donde marca el par de ejes cartesianos.

Para las pruebas y la determinación de los parámetros al calcular una distancia efectiva a las cuales se pueden detectar los códigos QR se tomó un objetivo fijo con distintas ya definidas y a partir de ahí poder determinar cual es la distancia mínima de reconocimiento. Se implemento la siguiente experiencia para determinar los parámetros de la cámara.

\begin{figure}[H]
   \includegraphics[trim={0 0 0 6cm}, clip, width=1.0\linewidth]{images/robot_medicion_qr.png}
   \caption{Experiencia para la obtención de parámetros de calibración de la cámara}
   \label{fig:robot_medicion_qr}
\end{figure}

El objetivo de esta calibración es encontrar la longitud focal, la cual es una constante que representa la distancia desde el centro del lente objetivo hasta el sensor de imagen. Siempre tomando como el ejemplo dimensiones de las fotografías obtenidas por la ESP32 las cuales se detallan en la siguiente imagen y donde el ancho es de $1280px$ mientras tanto que el alto es de $720px$.

\begin{figure}[H]
   \centering
   \includegraphics[width=0.5\linewidth]{images/ejemplo_foto.drawio.png}
   \caption{Ejemplo de las dimensiones tomadas por la cámara}
   \label{fig:qr}
\end{figure}

Por lo tanto se calcula de la siguiente manera \cite{wu2018size}:

\begin{equation}
   Longitud\ focal [px] = \frac{Ancho\ del\ QR\ en\ la\ imagen\ [px] \times Distancia\ real\ [mm]}{Dimensi\acute{o}n\ real\ del\ c\acute{o}digo\ QR\ [mm]}
   \label{ec:logitud_focal}
\end{equation}

Donde:
\begin{itemize}
   \item El ancho del QR en la imagen se representa en cantidad de pixeles.
   \item La distancia real es entre el código QR y la cámara medida en milímetros, en este caso 400$[mm]$.
   \item La dimensión real del código QR es el ancho del código QR, en este caso es de 50$[mm]$.
\end{itemize}

Al consultar el tamaño de pixel dado por la hoja de datos de la cámara OV2640, obtenemos que es de $2.2\mu m$. Por lo que podemos convertir la longitud focal de $[px]$ a $[mm]$ mediante la siguiente expresión:

\begin{equation}
 Longitud\ focal [mm] = Longitud\ focal [px] \times Dimensi\acute{o}n\ del\ pixel [mm/px]
\end{equation}

Una vez obtenido este parámetro lo vamos a utilizar para comparar la dimensión en píxeles del código QR en la imagen con el código QR en sí, y de esa manera obtener la distancia que existe hacia el objeto con la fórmula:

\begin{equation}
   Distancia[mm] = \frac{Dimensi\acute{o}n\ real\ del\ c\acute{o}digo\ QR\ [mm] \times Longitud\ focal\ [px]}{Ancho\ del\ QR\ en\ la\ imagen\ [px]}
   \label{ec:distancia_qr}
\end{equation}

Por supuesto que esto se aplica únicamente cuando en la fotografía se logra identificar un código QR, si durante el procesamiento de la imagen esto no sucede, no se va a poder determinar la distancia. Realizando varias pruebas se logró determinar que si la cámara se encuentra entre los $500mm$ y $600mm$ del objetivo, el código QR es identificable en la imagen. Una distancia mayor al rango descrito ya se considera fuera del rango.

A partir del posicionamiento del robot se obtuvo la siguiente imagen, la cual se usó de referencia para medir la distancia en todas las demás fotografías utilizando las expresiones desarrolladas anteriormente.

\begin{figure}[H]
   \centering
   \includegraphics[width=0.8\linewidth]{images/img_centro.jpg}
   \caption{Imagen de referencia obtenida por el robot}
   \label{fig:img_centro}
\end{figure}

\subsubsection{Intervención en la estimación de posición del robot}

Mencionamos anteriormente que el payload de los códigos QR contienen las coordenadas en el eje $X$e $Y$ en donde se encuentra el QR en el mapa. Al medir la distancia hacia el código lo que estamos haciendo es sumarle precisión a la determinación de la ubicación del robot. Entonces si tomamos como ejemplo ideal a una fotografía donde el código QR se encuentra perfectamente centrado, significa que el robot no presenta desplazamiento alguno el eje $X$.

\begin{figure}[H]
   \centering
   \includegraphics[width=0.7\linewidth]{images/ejemplo_foto_centro.jpg}
   \caption{Ejemplo de una fotografía con el código QR centrado}
   \label{fig:ejemplo_foto_centro}
\end{figure}

Entonces si nos basamos en la Figura \ref{fig:ejemplo_foto_centro} para determinar la posición del robot, obtenemos una posición mas precisa ya que si durante el procesamiento de la fotografía verificamos que el código QR se encuentra desplazado hacía la derecha, el robot va a estar desplazado hacía la izquierda sobre el eje $X$. Y si también el código QR detectado es de mayor dimensión en la imagen, el robot se encuentra posicionado más cerca del código QR.

Además cómo se mencionó mas arriba, el contenido del payload incluye las coordenadas en donde se encuentra el código dentro del plano. Entonces si tenemos el ejemplo donde un robot posicionado en la coordenada $(4,2)$ desea llegar a la coordenada $(4,0)$ donde se encuentra ubicado un código QR con las coordenadas descritas, nos encontramos en la situación en la que, por su posición actual, le es imposible al robot reconocer el código QR por la gran distancia que existe entre él y la cámara del robot, por lo tanto estimamos que el robot mientras realiza desplazamientos entre celda y celda siempre va a terminar posicionado al centro de ella.

\begin{figure}[H]
   \centering
   \includegraphics[width=0.5\linewidth]{images/robot_posicion_0.jpg}
   \caption{Robot posicionado en la coordenada $(4,2)$}
   \label{fig:robot_posicion_0}
\end{figure}

Ahora si el robot se encuentra llegando a su destino, la coordenada $(4,0)$ y se encuentra dentro del rango de distancia de reconocimiento de códigos QR, va a calcular la distancia tanto en el eje $Y$ como en el $X$ y sumarla al payload del código QR para después poder reportar esa coordenada. Entonces en ese momento, así como muestra la figura \ref{fig:robot_posicion_1} el robot en realidad se encuentra ubicado en la coordenada $(4.2,0.5)$ y no el centro de la celda como uno supone cuando no tiene el reporte de su ubicación en todo momento.

\begin{figure}[H]
   \centering
   \includegraphics[width=0.5\linewidth]{images/robot_posicion_1.jpg}
   \caption{Robot posicionado en la coordenada $(4.2,0.5)$}
   \label{fig:robot_posicion_1}
\end{figure}

La siguiente imagen \ref{fig:qrcamararobot} muestra cómo es una fotografía captada en movimiento por el robot y la lectura que hace del payload, cómo se puede observar es una imagen bastante nítida y con buen enfoque lo que permite que el procesamiento de la misma se realice con mayor precisión.

\begin{figure}[H]
   \centering
   \includegraphics[width=0.7\linewidth]{images/qr.png}
   \caption{Imagen obtenida por el robot en movimiento}
   \label{fig:qrcamararobot}
\end{figure}

\subsection{Pruebas y testing}

\begin{testtableformat}
   \hline \rowcolor{test_header_color}
       Test ID             & TC\_06\_00 \\
   \hline
       Tipo de test        & Test unitario \\
   \hline
       Objeto de prueba    & Calibración de la cámara \\
   \hline
       Requerimiento       & RF4 \\
   \hline
       Nombre              & Medición de la distancia hacia un objetivo \\
   \hline
       Descripción         & Lograr capturar la fotografía de un objetivo en un ambiente controlado para tomar obtener los parámetros de calibración de la cámara.\\
   \hline
       Precondición        & PRECOND\_H\\
   \hline
       Pasos del test      & \begin{enumerate}
                             \item Capturar una fotografía con la ESP32-CAM.
                             \item Envíar la imagen por comunicación inalámbrica hacia el servidor que aloja la imagen.
                             \item Verificar que la imagen llegó completa.
                             \item Procesar la imagen para obtener los parámetros de la imagen.
                             \end{enumerate} \\
   \hline
       Resultado esperado  & La imagen debe llegar de forma completa y en el procesamiento se debe detectar el código QR. \\
   \hline
       Resultado obtenido  & Tanto la imagen como el procesamiento fueron obtenidas de forma correcta. \\
   \hline
       Observaciones       & - \\
   \hline
\end{testtableformat}

\begin{testtableformat}
   \hline \rowcolor{test_header_color}
       Test ID             & TC\_06\_01 \\
   \hline
       Tipo de test        & Test unitario \\
   \hline
       Objeto de prueba    & Envío del contenido del payload \\
   \hline
       Requerimiento       & RF4 \\
   \hline
       Nombre              & Coordenadas del payload \\
   \hline
       Descripción         & Capturar una fotografía, detectar si existe un código QR en la imagen y enviar el contenido del payload.\\
   \hline
       Precondición        & PRECOND\_H\\
   \hline
       Pasos del test      & \begin{enumerate}
                             \item Capturar una fotografía con la ESP32-CAM.
                             \item Envíar la imagen por comunicación inalámbrica hacia el servidor que aloja la imagen.
                             \item Verificar que la imagen llegó completa.
                             \item Procesar la imagen para obtener los parámetros de la imagen.
                             \end{enumerate} \\
   \hline
       Resultado esperado  & Se debe poder llegar a leer el payload de forma completa. \\
   \hline
       Resultado obtenido  & El contenido del payload se lee correctamente. \\
   \hline
       Observaciones       & - \\
   \hline
\end{testtableformat}

\begin{testtableformat}
   \hline \rowcolor{test_header_color}
       Test ID             & TC\_06\_02 \\
   \hline
       Tipo de test        & Test integración \\
   \hline
       Objeto de prueba    & Realizar procesamiento de los códigos QR mientras el robot se encuentra en movimiento\\
   \hline
       Nombre              & Captura de fotografías con el robot en movimiento\\
   \hline
       Descripción         & La idea es capturar y enviar las fotografías de los códigos QR que se encuentran dispersos a lo largo del mapa mientras el robot realiza un desplazamiento\\
   \hline
       Precondición        & PRECOND\_H \\
   \hline
       Pasos del test      & \begin{enumerate}
                             \item Validar que se reciben las fotografías
                             \item Inicializar el proceso de desplazamiento del robot
                             \item Capturar y envíar las fotografías tomadas
                             \item Validar que el procesamiento de las fotos es correcto y el payload es legible
                             \end{enumerate} \\
   \hline
       Resultado esperado  & Recibir las fotografías de forma correcta y realizar la lectura y procesamiento del payload\\
   \hline
       Resultado obtenido  & Todas las fotos se reciben de forma completa, es decir, no contienen errores y por lo tanto el procesamiento de los códigos QR es efectiva y la información contenida en su payload es procesable\\
   \hline
       Observaciones       & \\
   \hline
\end{testtableformat}

\begin{testtableformat}
    \hline \rowcolor{test_header_color}
        Test ID             & TC\_06\_03 \\
    \hline
        Tipo de test        & Test de sistema \\
    \hline
        Objeto de prueba    & comunicación inalámbrica - PID - Modelo cinemático compensado - Odometría - Seguidor de línea magnética - Modelo del Mapa - Calculador de trayectorias - Red de Petri - Monitor - Filtro de Kalman - Lectura de códigos QR\\
    \hline
        Nombre              & Prueba de sistema integrado\\
    \hline
        Descripción         & Verificar que la interfaz, el robot y todos los componentes involucrados funcionan de manera adecuada\\
    \hline
        Precondición        & PRECOND\_I \\
    \hline
        Pasos del test      & \begin{enumerate}
                              \item En la interfaz determinar la coordenada origen y destino
                              \item Calcular la trayectorias
                              \item Enviar los setpoints
                              \item Comprobar que el robot se mueve a lo largo de la trayectoria definida, al desviarse se corrige su posición y reporta la lectura de los códigos QR
                              \item Repetir la prueba desde el paso 1 con distintos valores
                              \end{enumerate} \\
    \hline
        Resultado esperado  & El robot pueda completar el desplazamiento definido por el usuario usando la interfaz como medio de control y el robot pueda reportar la lectura de los códigos QR de manera efectiva\\
    \hline
        Resultado obtenido  & El robot y la interfaz se comportan de manera esperada. El robot realiza las trayectorias dentro de los límites observados en las pruebas unitarias y de integración hechas anteriormente\\
    \hline
        Observaciones & Se probó recorridos de hasta 4 metros por limitaciones de espacio\\
    \hline
 \end{testtableformat}

\subsection{Resultados}
Los resultados son satisfactorios ya que se pudo cumplir con creces el objetivo de poder enviar y realizar el procesamiento de las imágenes tomadas por la cámara del microcontrolador ESP-32. Como este microcontrolador sólo se encarga de tomar fotos, pudimos generar imágenes de gran calidad llegando a una resolución HD y a una frecuencia de envío de 2 FPS (Frame per second), es decir, 2 imágenes por segundos. Por supuesto esto no llega a ser un procesamiento de video ya que no se arma un objeto con esas imágenes, solamente se realiza el procesamiento de las mismas de forma particular.
Además no implicó realizar cambios grandes en el sistema principal de movilización del robot, esto porque al estar todo conectado vía las comunicaciones inalámbricas, el sistema explicado puede considerarse como un módulo externo, que, por supuesto ayuda al robot a tomar conocimiento de su entorno.

\subsection{Riesgos superados}
\begin{center}
    \begin{tabular} {
        | c| c |}
        \hline \rowcolor{test_header_color}
            ID & Riesgo \\
        \hline
            RI-02 & Intercomunicación de componentes ineficiente o ineficaz \\
        \hline
            RI-03 & Prestaciones insuficientes de componentes \\
        \hline
    \end{tabular}
\end{center}

\subsection{Conclusiones}
En este capítulo se explicaron los motivos que nos llevaron a implementar este tipo de procesamiento y porque consideramos que el conocimiento del entorno y sus parámetros son importantes para mejorar la movilización del robot en un entorno real. Lamentablemente, los tiempos del proyecto no fueron suficientes para realizar todas las pruebas necesarias y verificar fehacientemente que el comportamiento del robot cambian para mejor con la ayuda de las mediciones tomadas de las imágenes, esto teniendo en cuenta que no se añaden de forma directa ni se suman como entrada al filtro de Kalman.
Esperamos que una próxima iteración de la serie Hermes esto se pueda continuar, y plasmar en el mundo real lo que planteamos en la teoría de la implementación.