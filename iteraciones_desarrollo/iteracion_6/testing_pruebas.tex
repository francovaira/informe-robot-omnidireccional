\subsection{Pruebas y testing}

\begin{testtableformat}
   \hline \rowcolor{test_header_color}
       Test ID             & TC\_06\_00 \\
   \hline
       Tipo de test        & Test unitario \\
   \hline
       Objeto de prueba    & Calibración de la cámara \\
   \hline
       Requerimiento       & RF4 \\
   \hline
       Nombre              & Medición de la distancia hacia un objetivo \\
   \hline
       Descripción         & Lograr capturar la fotografía de un objetivo en un ambiente controlado para tomar obtener los parámetros de calibración de la cámara.\\
   \hline
       Precondición        & PRECOND\_H\\
   \hline
       Pasos del test      & \begin{enumerate}
                             \item Capturar una fotografía con la ESP32-CAM.
                             \item Envíar la imagen por comunicación inalámbrica hacia el servidor que aloja la imagen.
                             \item Verificar que la imagen llegó completa.
                             \item Procesar la imagen para obtener los parámetros de la imagen.
                             \end{enumerate} \\
   \hline
       Resultado esperado  & La imagen debe llegar de forma completa y en el procesamiento se debe detectar el código QR. \\
   \hline
       Resultado obtenido  & Tanto la imagen como el procesamiento fueron obtenidas de forma correcta. \\
   \hline
       Observaciones       & - \\
   \hline
\end{testtableformat}

\begin{testtableformat}
   \hline \rowcolor{test_header_color}
       Test ID             & TC\_06\_01 \\
   \hline
       Tipo de test        & Test unitario \\
   \hline
       Objeto de prueba    & Envío del contenido del payload \\
   \hline
       Requerimiento       & RF4 \\
   \hline
       Nombre              & Coordenadas del payload \\
   \hline
       Descripción         & Capturar una fotografía, detectar si existe un código QR en la imagen y enviar el contenido del payload.\\
   \hline
       Precondición        & PRECOND\_H\\
   \hline
       Pasos del test      & \begin{enumerate}
                             \item Capturar una fotografía con la ESP32-CAM.
                             \item Envíar la imagen por comunicación inalámbrica hacia el servidor que aloja la imagen.
                             \item Verificar que la imagen llegó completa.
                             \item Procesar la imagen para obtener los parámetros de la imagen.
                             \end{enumerate} \\
   \hline
       Resultado esperado  & Se debe poder llegar a leer el payload de forma completa. \\
   \hline
       Resultado obtenido  & El contenido del payload se lee correctamente. \\
   \hline
       Observaciones       & - \\
   \hline
\end{testtableformat}

\begin{testtableformat}
   \hline \rowcolor{test_header_color}
       Test ID             & TC\_06\_02 \\
   \hline
       Tipo de test        & Test integración \\
   \hline
       Objeto de prueba    & Realizar procesamiento de los códigos QR mientras el robot se encuentra en movimiento\\
   \hline
       Nombre              & Captura de fotografías con el robot en movimiento\\
   \hline
       Descripción         & La idea es capturar y enviar las fotografías de los códigos QR que se encuentran dispersos a lo largo del mapa mientras el robot realiza un desplazamiento\\
   \hline
       Precondición        & PRECOND\_H \\
   \hline
       Pasos del test      & \begin{enumerate}
                             \item Validar que se reciben las fotografías
                             \item Inicializar el proceso de desplazamiento del robot
                             \item Capturar y envíar las fotografías tomadas
                             \item Validar que el procesamiento de las fotos es correcto y el payload es legible
                             \end{enumerate} \\
   \hline
       Resultado esperado  & Recibir las fotografías de forma correcta y realizar la lectura y procesamiento del payload\\
   \hline
       Resultado obtenido  & Todas las fotos se reciben de forma completa, es decir, no contienen errores y por lo tanto el procesamiento de los códigos QR es efectiva y la información contenida en su payload es procesable\\
   \hline
       Observaciones       & \\
   \hline
\end{testtableformat}

\begin{testtableformat}
    \hline \rowcolor{test_header_color}
        Test ID             & TC\_06\_03 \\
    \hline
        Tipo de test        & Test de sistema \\
    \hline
        Objeto de prueba    & comunicación inalámbrica - PID - Modelo cinemático compensado - Odometría - Seguidor de línea magnética - Modelo del Mapa - Calculador de trayectorias - Red de Petri - Monitor - Filtro de Kalman - Lectura de códigos QR\\
    \hline
        Nombre              & Prueba de sistema integrado\\
    \hline
        Descripción         & Verificar que la interfaz, el robot y todos los componentes involucrados funcionan de manera adecuada\\
    \hline
        Precondición        & PRECOND\_I \\
    \hline
        Pasos del test      & \begin{enumerate}
                              \item En la interfaz determinar la coordenada origen y destino
                              \item Calcular la trayectorias
                              \item Enviar los setpoints
                              \item Comprobar que el robot se mueve a lo largo de la trayectoria definida, al desviarse se corrige su posición y reporta la lectura de los códigos QR
                              \item Repetir la prueba desde el paso 1 con distintos valores
                              \end{enumerate} \\
    \hline
        Resultado esperado  & El robot pueda completar el desplazamiento definido por el usuario usando la interfaz como medio de control y el robot pueda reportar la lectura de los códigos QR de manera efectiva\\
    \hline
        Resultado obtenido  & El robot y la interfaz se comportan de manera esperada. El robot realiza las trayectorias dentro de los límites observados en las pruebas unitarias y de integración hechas anteriormente\\
    \hline
        Observaciones & Se probó recorridos de hasta 4 metros por limitaciones de espacio\\
    \hline
 \end{testtableformat}