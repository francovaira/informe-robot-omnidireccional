\subsection{Testing y pruebas}

\begin{testtableformat}
    \hline \rowcolor{test_header_color}
        Test ID             & TC\_03\_00 \\
    \hline
        Tipo de test        & Test unitario \\
    \hline
        Objeto de prueba    & Calculador de trayectorias \\
    \hline
        Requerimiento       & RF7 \\
    \hline
        Nombre              & Cálculo de trayectorias en un mapa con obstáculos, trayectoria resoluble \\
    \hline
        Descripción         & Calcular la secuencia de coordenadas a seguir para llegar a un punto A a un punto B del mapa, con obstáculos dispuestos de modo que existe al menos una trayectoria posible \\
    \hline
        Precondición        & PRECOND\_E \\
    \hline
        Pasos del test      & \begin{enumerate}
                                \item Enviar al PathFinder una coordenada origen y una destino de modo que exista un camino soluble entre ellas
                                \item Verificar que la salida del PathFinder consiste en la secuencia de coordenadas del mapa que se deben visitar para llegar del origen al destino
                                \item Repetir desde el paso 1) con diferentes valores
                            \end{enumerate} \\
    \hline
        Resultado esperado  & Se obtiene la secuencia adecuada para los puntos dados y se cumple la restricción de solo movimientos horizontales o verticales \\
    \hline
        Resultado obtenido  & La secuencia de coordenadas calculadas se corresponden a las más óptimas para cada par de puntos y se cumple la restricción de movimiento \\
    \hline
        Observaciones       & - \\
    \hline
\end{testtableformat}


\begin{testtableformat}
    \hline \rowcolor{test_header_color}
        Test ID             & TC\_03\_01 \\
    \hline
        Tipo de test        & Test unitario \\
    \hline
        Objeto de prueba    & Calculador de trayectorias \\
    \hline
        Requerimiento       & RF7 \\
    \hline
        Nombre              & Cálculo de trayectorias en un mapa con obstáculos, trayectoria irresoluble \\
    \hline
        Descripción         & Calcular la secuencia de coordenadas a seguir para llegar a un punto A a un punto B del mapa, con obstáculos dispuestos de modo que no existe ninguna trayectoria posible \\
    \hline
        Precondición        & PRECOND\_F \\
    \hline
        Pasos del test      & \begin{enumerate}
                                \item Enviar al PathFinder una coordenada origen y una destino de modo que no exista camino posible entre ellas
                                \item Verificar que la salida del PathFinder es un error
                                \item Repetir desde el paso 1) con diferentes valores
                            \end{enumerate} \\
    \hline
        Resultado esperado  & Se obtiene un mensaje de error informando que no existe trayectoria posible \\
    \hline
        Resultado obtenido  & El error se presenta solo cuando se define el mapa de modo que entre un punto A y un punto B no existe camino posible \\
    \hline
        Observaciones       & - \\
    \hline
\end{testtableformat}
