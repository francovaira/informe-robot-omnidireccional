\subsection{Testing y pruebas}

\begin{testtableformat}
    \hline \rowcolor{test_header_color}
        Test ID             & TC\_03\_00 \\
    \hline
        Tipo de test        & Test unitario \\
    \hline
        Objeto de prueba    & Calculador de trayectorias \\
    \hline
        Requerimiento       & RF7 \\
    \hline
        Nombre              & Cálculo de trayectorias en un mapa con obstáculos, trayectoria resoluble \\
    \hline
        Descripción         & Calcular la secuencia de coordenadas a seguir para llegar a un punto A a un punto B del mapa, con obstáculos dispuestos de modo que existe al menos una trayectoria posible \\
    \hline
        Precondición        & PRECOND\_E \\
    \hline
        Pasos del test      & \begin{enumerate}
                                \item Enviar al PathFinder una coordenada origen y una destino de modo que exista un camino soluble entre ellas
                                \item Verificar que la salida del PathFinder consiste en la secuencia de coordenadas del mapa que se deben visitar para llegar del origen al destino
                                \item Repetir desde el paso 1) con diferentes valores
                            \end{enumerate} \\
    \hline
        Resultado esperado  & Se obtiene la secuencia adecuada para los puntos dados y se cumple la restricción de solo movimientos horizontales o verticales \\
    \hline
        Resultado obtenido  & La secuencia de coordenadas calculadas se corresponden a las más óptimas para cada par de puntos y se cumple la restricción de movimiento \\
    \hline
        Observaciones       & - \\
    \hline
\end{testtableformat}


\begin{testtableformat}
    \hline \rowcolor{test_header_color}
        Test ID             & TC\_03\_01 \\
    \hline
        Tipo de test        & Test unitario \\
    \hline
        Objeto de prueba    & Calculador de trayectorias \\
    \hline
        Requerimiento       & RF7 \\
    \hline
        Nombre              & Cálculo de trayectorias en un mapa con obstáculos, trayectoria irresoluble \\
    \hline
        Descripción         & Calcular la secuencia de coordenadas a seguir para llegar a un punto A a un punto B del mapa, con obstáculos dispuestos de modo que no existe ninguna trayectoria posible \\
    \hline
        Precondición        & PRECOND\_F \\
    \hline
        Pasos del test      & \begin{enumerate}
                                \item Enviar al PathFinder una coordenada origen y una destino de modo que no exista camino posible entre ellas
                                \item Verificar que la salida del PathFinder es un error
                                \item Repetir desde el paso 1) con diferentes valores
                            \end{enumerate} \\
    \hline
        Resultado esperado  & Se obtiene un mensaje de error informando que no existe trayectoria posible \\
    \hline
        Resultado obtenido  & El error se presenta solo cuando se define el mapa de modo que entre un punto A y un punto B no existe camino posible \\
    \hline
        Observaciones       & - \\
    \hline
\end{testtableformat}


\begin{testtableformat}
    \hline \rowcolor{test_header_color}
        Test ID             & TC\_03\_02 \\
    \hline
        Tipo de test        & Test de integración \\
    \hline
        Objeto de prueba    & Interfaz de usuario y calculador de trayectorias \\
    \hline
        Requerimiento       & RF7 - RF10 \\
    \hline
        Nombre              & Interacción con la interfaz y cálculo de trayectorias con obstáculos \\
    \hline
        Descripción         & Verificar que al interactuar con la interfaz es posible establecer puntos de origen y de destino, para luego calcular la secuencia de coordenadas a seguir en base a esos puntos \\
    \hline
        Precondición        & PRECOND\_E \\
    \hline
        Pasos del test      & \begin{enumerate}
                                \item Establecer un punto de origen y un punto de destino en el mapa de la interfaz
                                \item Dar la orden de calcular la secuencia de coordenadas y verificar que son validas
                                \item Repetir desde el paso 1) para otra combinación de coordenadas origen y destino
                            \end{enumerate} \\
    \hline
        Resultado esperado  & En todos los casos se obtiene una lista con las coordenadas a recorrer desde el punto de origen hasta el final \\
    \hline
        Resultado obtenido  & La lista de coordenadas obtenidas representa las coordenadas del mapa por las cuales se debe pasar para llegar a destino \\
    \hline
        Observaciones       & - \\
    \hline
\end{testtableformat}


\begin{testtableformat}
    \hline \rowcolor{test_header_color}
        Test ID             & TC\_03\_03 \\
    \hline
        Tipo de test        & Test de sistema \\
    \hline
        Objeto de prueba    & Comunicación inalámbrica - PID - Modelo cinemático compensado - Odometría - Seguidor de línea magnética - Modelo del Mapa - Calculador de trayectorias - Interfaz de usuario \\
    \hline
        Requerimiento       & RF1 - RF2 - RF3 - RF4 - RF5 - RF6 - RF7 - RF10 \\
    \hline
        Nombre              & Prueba de sistema integrado \\
    \hline
        Descripción         & Comprobar que el robot realiza trayectorias en una dirección y longitud determinadas \\
    \hline
        Precondición        & PRECOND\_C - PRECOND\_E \\
    \hline
        Pasos del test      & \begin{enumerate}
                                \item Establecer en la interfaz un par de puntos origen y destino y calcular la trayectoria
                                \item Enviar al robot un setpoint por cada coordenada a recorrer
                                \item Repetir desde el paso 1) con diferentes valores
                            \end{enumerate} \\
    \hline
        Resultado esperado  &  El robot se mantiene centrado a lo largo de la línea magnética y que es compensado por el Modelo Cinemático. Además, la interfaz debe mostrar en tiempo real la celda que ocupa el robot y debe reportar mediciones de velocidad y distancia \\
    \hline
        Resultado obtenido  & El robot realiza las trayectorias de manera acorde dentro de los límites observados en las pruebas unitarias y de integración. Se logra recolectar la información enviada por el robot y se visualiza la posición del robot en la interfaz \\
    \hline
        Observaciones       & - \\
    \hline
\end{testtableformat}