\subsubsection{Diseño de la pista}

El diseño de la pista debe tener en cuenta las limitaciones impuestas a movimientos que puede y no puede hacer el robot en el plano. Dado que tendrá montada una cámara fija y ésta deberá tomar imágenes que son perpendiculares a la dirección de movimiento, es por lo que se establece una restricción al movimiento del robot sobre el plano. Aunque el robot tiene la capacidad de realizar movimientos rectos en cualquier ángulo e incluso hacer trayectorias curvas, se ha decidido que solo realizará trayectorias en línea recta a 90° y que puede rotar sobre su eje, pero no realizar trayectorias curvas.

Se procede a diseñar la pista dibujando un plano detallado del recorrido y la ubicación de los imanes considerando la distancia entre ellos.

\begin{figure}[H]
    \centering
    \includegraphics[width=0.5\linewidth]{images/pista_diseño_full.png}
    \caption{Diseño de la pista}
    \label{fig:disenopista}
\end{figure}

Procedimos a realizar un corte láser sobre madera MDF de 5mm. Se instalaron los imanes en la pista de modo que están correctamente alineados todos en la misma polaridad para que el robot pueda seguir la línea magnética sin problemas.

Finalmente, se realizó la calibración iterativa del robot en la pista diseñada. Se ajustan los sensores magnéticos del robot y se realizan múltiples pruebas en diferentes condiciones para evaluar su desempeño y hacer los ajustes necesarios.
