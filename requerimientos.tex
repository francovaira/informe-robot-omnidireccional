\chapter{Requerimientos}

\section{Requerimientos funcionales}

Los requerimientos funcionales son aquellas declaraciones de los servicios que debe proporcionar el sistema, cómo reacciona ante determinados eventos y cómo se comporta en situaciones específicas.

\begin{center} 
    \begin{tabular} {
        | >{\centering\arraybackslash}m{1cm}
        | >{\centering\arraybackslash}m{13cm}|}
        \hline
            ID & Descripción \\
        \hline
            RF1 & El robot debe contar con un sistema de control para las 4 ruedas \\ 
        \hline
            RF2 & El robot debe tener un sistema de locomoción omnidireccional \\ 
        \hline
            RF3 & El robot debe poder medir la distancia recorrida \\ 
        \hline
            RF4 & El robot debe poder realizar trayectorias en línea recta y curvas \\ 
        \hline
            RF5 & El robot debe poder corregir su trayectoria mediante el uso de sensores \\ 
        \hline
            RF6 & El robot debe recibir y enviar información mediante comunicaciones inalámbricas \\ 
        \hline
            RF7 & Debe existir un modo de calcular trayectorias automáticamente \\
        \hline
            RF8 & Debe poder ubicarse al robot en el plano de forma precisa \\
        \hline
            RF9 & El robot debe identificar su ambiente mediante el uso de una cámara \\ 
        \hline
            RF10 & Debe existir una interfaz de usuario para control y monitoreo \\ 
        \hline
    \end{tabular}
\end{center}

\section{Requerimientos no funcionales}

Los requerimientos no funcionales especifican las características o cualidades que el sistema debe tener, pero no definen directamente las funciones del sistema, sino a aquellas propiedades que emergen de ella como la fiabilidad, rendimiento, seguridad, usabilidad, etc. También definen las restricciones del sistema como los dispositivos de entrada/salida y las interfaces de comunicación del sistema.

\begin{center}
    \begin{tabular} {
        | >{\centering\arraybackslash}m{1cm}
        | >{\centering\arraybackslash}m{13cm}|}
        \hline
            ID & Descripción \\
        \hline
            RNF1 & Debería tener tiempos de respuesta aceptables para el buen funcionamiento del sistema de control \\
        \hline
            RNF2 & El software debería contar con pruebas unitarias y de integración \\
        \hline
            RNF3 & La comunicación inalámbrica debería usar protocolos de comunicación encriptados \\
        \hline
            RNF4 & El código debería contar con documentación \\
        \hline
    \end{tabular}
\end{center}