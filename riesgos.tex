\chapter{Análisis de riesgos}

Para el análisis de los riesgos del proyecto, se ha decidido utilizar la estrategia de Mitigación y Gestión de Riesgos, conocido por sus siglas en inglés como RMMM (Risk Mitigation, Monitoring and Management). Para llevar a cabo cada una de las tareas anteriormente mencionadas, se ideará un plan de prevención, un plan de monitorización y un plan de contingencia respectivamente.

Se consideran riesgos aquellas situaciones que, en caso de suceder, afectan negativamente determinados aspectos del proyecto. Por esta razón es importante identificarlos con anticipación para mitigarlos.

El plan de prevención intenta ser un escudo para mantener la cantidad de riesgos latentes al mínimo y que no logren materializarse, por lo que su objetivo es entorpecer lo máximo posible el flujo a través del circuito RMMM.

El plan de monitorización hace la gestión del riesgo, es decir, que una vez que se concreta un evento desafortunado asociado a un riesgo, se procede inmediatamente a solucionarlo y luego poder continuar con el proyecto. Las soluciones en esta etapa no deben ser tomadas apresuradamente porque pueden incrementar el nivel de riesgos de todo el proyecto y hacerlo inestable.

La ponderación de riesgos será calculada en base a la siguiente ecuación:

$$ \text{Estimación de la probabilidad $\times$ Estimación del impacto = Exposición al riesgo} $$

\section{Listado de riesgos}

\begin{center}
\begin{tabular} {|m{2.5cm}|m{11.5cm}|}
	\hline
	RI-01 & Incompatibilidad o avería de componentes \\
	\hline
    	Condición & Elección incorrecta del sistema embebido a usar. \\
	\hline
    	Consecuencias & Interrupciones en el funcionamiento del robot, comportamiento errático, pérdida de enlaces de comunicación, reinicios inesperados, entre otros. \\
	\hline
    	Efecto & No se logrará cumplir con las expectativas mínimas de robustez y estabilidad que se espera de un robot autónomo. \\
	\hline
\end{tabular}
\end{center}

\begin{center}
\begin{tabular} {|m{2.5cm}|m{11.5cm}|}
	\hline
	RI-02 & Intercomunicación de componentes ineficiente o ineficaz \\
	\hline
    	Condición & Comunicación entre componentes lenta, con interferencias y/o con ruido. \\
	\hline
    	Consecuencias & Sistema robótico con tiempo de respuestas grandes que imposibilitan un eficaz funcionamiento de un sistema de control. \\
	\hline
    	Efecto & El robot no podrá reaccionar a tiempo para evitar obstáculos que detecte en su camino y podrá desviarse del camino. Las respuestas del robot a los comandos de movimientos están muy desfasadas en el tiempo, lo que dificultará su manejo tanto manual como autónomo. A tiempo de respuesta mayores, menor será la velocidad máxima permitida de desplazamientos del robot. \\
	\hline
	\end{tabular} \end{center}

\begin{center}
\begin{tabular} {|m{2.5cm}|m{11.5cm}|}
	\hline
	RI-03 & Prestaciones insuficientes de componentes \\
	\hline
    	Condición & Sobrestimar las especificaciones técnicas de componentes. También puede ocurrir subestimar las necesidades tecnológicas de la solución. Falta documentación fiable de componentes. \\
	\hline
    	Consecuencias & Componentes que no son capaces de cumplir correctamente con su propósito y por lo tanto pasan a ser inservibles para los fines del proyecto. \\
	\hline
    	Efecto & Incremento del costo económico del proyecto por el descarte y reemplazo de los componentes por otros de mayor prestaciones. \\
	\hline
\end{tabular}
\end{center}

\begin{center}
\begin{tabular} {|m{2.5cm}|m{11.5cm}|}
	\hline
	RI-04 & Modificación de los requerimientos del proyecto \\
	\hline
    	Condición & Definición de nuevos requisitos o modificaciones de los existentes durante el desarrollo del proyecto. \\
	\hline
    	Consecuencias & Replanificación de las tareas a realizar. Desarrollos en etapas avanzadas en los que sí invirtió tiempo y esfuerzo pueden interrumpirse y quedar inservibles para los fines del proyecto. \\
	\hline
    	Efecto & Retrasos en la finalización del producto final del proyecto. \\
	\hline
\end{tabular}
\end{center}

\begin{center}
\begin{tabular} {|m{2.5cm}|m{11.5cm}|}
	\hline
	RI-05 & Dificultad en conseguir determinados componentes \\
	\hline
    	Condición & Componentes electrónicos que deben ser importados y provenientes de empresas que no trabajan con envíos al exterior. Altos costos de importación y/o envío. Grandes tiempo de demora en los envíos desde el exterior. \\
	\hline
    	Consecuencias & Análisis de componentes alternativos que puedan ser utilizados para el mismo fin que el componente que no puede conseguirse. En última instancia podrían adaptarse los requerimientos del proyecto. \\
	\hline
    	Efecto & Retrasos en la finalización del producto final del proyecto y/o degradación de las capacidades del robot. \\
	\hline
\end{tabular}
\end{center}

\begin{center}
\begin{tabular} {|m{2.5cm}|m{11.5cm}|}
	\hline
	RI-06 & Excesivo tiempo para cumplir los objetivos del proyecto \\
	\hline
    	Condición & Cualquier dificultad técnica que retrase significativamente el progreso del proyecto. Escasez de tiempo disponible de los integrantes del equipo de desarrollo \\
	\hline
    	Consecuencias & Robot con capacidades reducidas, menores a las esperadas. \\
	\hline
    	Efecto & Directores del proyecto podrán readaptar los objetivos a alcanzar dando una extensión de tiempo más allá de lo planificado desde un comienzo. \\
	\hline
\end{tabular}
\end{center}

\begin{center}
\begin{tabular} {|m{2.5cm}|m{11.5cm}|}
	\hline
	RI-07 & Reducción de la fuerza de trabajo \\
	\hline
    	Condición & Miembros del equipo que abandonen el proyecto. \\
	\hline
    	Consecuencias & Las tareas a realizar llevaran más tiempo concretarlas. \\
	\hline
    	Efecto & Retraso en la entrega final del proyecto. \\
	\hline
\end{tabular}
\end{center}

\section{Estimación de la probabilidad}
\begin{center}
\begin{tabular}{
	| >{\centering\arraybackslash}m{2cm}
	| >{\centering\arraybackslash}m{3cm}
	| >{\centering\arraybackslash}m{4cm}
	| >{\centering\arraybackslash}m{1.5cm}
	| >{\centering\arraybackslash}m{1.5cm} |
	}
	\hline
    	Rango de probabilidad & Promedio para el cálculo & Expresión en lenguaje natural & Valor numérico & Código de color \\
	\hline
    	1\% a 20\% & 10\% & Muy baja probabilidad & 1 & \cellcolor{blue!65}\\
	\hline
    	21\% a 40\% & 30\% & Baja probabilidad & 2 & \cellcolor{green!65}\\
	\hline
    	41\% a 60\% & 50\% & Mediana probabilidad & 3 & \cellcolor{yellow!65}\\
	\hline
    	61\% a 80\% & 70\% & Alta probabilidad & 4 & \cellcolor{orange!65}\\
	\hline
    	81\% a 99\% & 90\% & Muy alta probabilidad & 5 & \cellcolor{red!65}\\
	\hline
\end{tabular}
\end{center}

\begin{center}
\begin{tabular} {
	| >{\centering\arraybackslash}m{1cm}
	| >{\centering\arraybackslash}m{9cm}
	| >{\centering\arraybackslash}m{2.9cm} |
	}
	\hline
    	ID & Riesgo & Probabilidad \\
	\hline
    	RI-01 & Incompatibilidad o avería de componentes & Mediana \cellcolor{green!65}\\
	\hline
    	RI-02 & Intercomunicación de componentes ineficiente o ineficaz & Alta \cellcolor{yellow!65}\\
	\hline
    	RI-03 & Prestaciones insuficientes de componentes & Mediana \cellcolor{orange!65}\\
	\hline
    	RI-04 & Modificación de los requerimientos del proyecto & Baja \cellcolor{green!65}\\
	\hline
    	RI-05 & Dificultad en conseguir determinados componentes & Muy alta \cellcolor{red!65}\\
	\hline
    	RI-06 & Excesivo tiempo para cumplir los objetivos del proyecto & Alta \cellcolor{orange!65}\\
	\hline
    	RI-07 & Reducción de la fuerza de trabajo & Muy baja \cellcolor{blue!65}\\
	\hline
\end{tabular}
\end{center}

\section{Estimación de impacto}
\begin{center}
\begin{tabular} {
	| >{\centering\arraybackslash}m{3cm}
	| >{\centering\arraybackslash}m{5cm}
	| >{\centering\arraybackslash}m{1.5cm}
	| >{\centering\arraybackslash}m{1.5cm} |
	}
	\hline
    	Criterio & Retraso en la planificación & Valor numérico & Código de color \\
	\hline
    	Insignificante & 1 semana & 1 & \cellcolor{blue!65} \\
	\hline
    	Moderado & 2 a 3 semanas & 2 & \cellcolor{green!65} \\
	\hline
    	Medio & 4 a 5 semanas & 3 & \cellcolor{yellow!65} \\
	\hline
    	Crítico & 6 a 8 semanas & 4 & \cellcolor{orange!65} \\
	\hline
    	Catastrófico & Más de 8 semanas & 5 & \cellcolor{red!65} \\
	\hline
\end{tabular}
\end{center}

\begin{center}
\begin{tabular}{|c|c|c|}
	\hline
    	ID & Riesgo & Impacto \\
	\hline
    	RI-01 & Incompatibilidad o avería de componentes & Moderado \cellcolor{green!65} \\
	\hline
    	RI-02 & Intercomunicación de componentes ineficiente o ineficaz & Moderado \cellcolor{yellow!65} \\
	\hline
    	RI-03 & Prestaciones insuficientes de componentes & Medio \cellcolor{yellow!65} \\
	\hline
    	RI-04 & Modificación de los requerimientos del proyecto & Medio \cellcolor{orange!65} \\
	\hline
    	RI-05 & Dificultad en conseguir determinados componentes & Crítico \cellcolor{orange!65} \\
	\hline
    	RI-06 & Excesivo tiempo para cumplir los objetivos del proyecto & Catastrófico \cellcolor{red!65} \\
	\hline
    	RI-07 & Reducción de la fuerza de trabajo & Crítico \cellcolor{orange!65} \\
	\hline
\end{tabular}
\end{center}

\section{Exposición al riesgo}
\begin{center}
\begin{tabular} {
	| >{\centering\arraybackslash}m{1cm}
	| >{\centering\arraybackslash}m{7cm}
	| >{\centering\arraybackslash}m{2cm}
	| >{\centering\arraybackslash}m{1.4cm}
	| >{\centering\arraybackslash}m{1.6cm} |
	}
	\hline
    	ID & Riesgo & Probabilidad & Impacto & Exposición \\
	\hline
    	RI-01 & Incompatibilidad o avería de componentes & 50\% & 2 & 1 \\
	\hline
    	RI-02 & Intercomunicación de componentes ineficiente o ineficaz & 70\% & 2 & 2,1\\
	\hline
    	RI-03 & Prestaciones insuficientes de componentes & 50\% & 3 & 1,5 \\
	\hline
    	RI-04 & Modificación de los requerimientos del proyecto & 30\% & 4 & 1,2 \\
	\hline
    	RI-05 & Dificultad en conseguir determinados componentes & 90\% & 4 & 3,6 \\
	\hline
    	RI-06 & Excesivo tiempo para cumplir los objetivos del proyecto & 70\% & 5 & 3,5 \\
	\hline
    	RI-07 & Reducción de la fuerza de trabajo & 10\% & 4 & 0,4 \\
	\hline
\end{tabular}
\end{center}

\section{Conclusión}
El análisis de riesgos presentado permite identificar y priorizar los principales desafíos que podrían afectar el desarrollo del proyecto. Mediante la implementación de los planes de prevención, monitoreo y contingencia, se busca mitigar el impacto de estos riesgos, garantizando que el proyecto avance de manera estable y eficiente. La ponderación de riesgos mediante la ecuación de exposición al riesgo facilitará la toma de decisiones informadas y la asignación adecuada de recursos, contribuyendo al éxito del proyecto.