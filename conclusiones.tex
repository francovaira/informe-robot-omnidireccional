\chapter{Conclusiones}

Con los objetivos principales y secundarios en mente durante todo el proyecto, podemos concluir que todos los objetivos propuestos a lo largo del mismo han sido cumplidos. El desarrollo del robot omnidireccional presentado en este Proyecto Integrador requirió conocimientos multidisciplinarios sobre diseño e implementación de sistemas ciberfísicos, abordando problemáticas propias de la robótica. A través de un enfoque iterativo, se logró construir un prototipo funcional capaz de desplazarse de manera autónoma, integrando algoritmos de planificación de trayectorias y sistemas de control anidados.

Entre los hitos más destacados alcanzados durante el desarrollo, se logró implementar un mecanismo de manejo fino de hilos en Python dentro del Monitor de la Red de Petri, superando en su mayoría las restricciones impuestas por el Global Interpreter Lock (GIL).

Otra contribución significativa fue la implementación de políticas para la resolución de conflictos entre robots en el mapa al querer acceder simultáneamente a la misma celda. La utilización de Redes de Petri como herramienta de modelado resultó clave para representar el comportamiento concurrente del sistema y garantizar la seguridad en contextos multiagente, minimizando riesgos de colisión y mejorando la coordinación entre unidades.

También se integró un Filtro de Kalman para la compensación de la trayectoria y la posición del robot dentro del mapa, mejorando notablemente su precisión de navegación. A su vez, se incorporó un sistema de medición de distancia basado en el reconocimiento de códigos QR, lo que aporta medios adicionales para la percepción del entorno.

Paralelamente a todo lo anterior, el desarrollo de una interfaz de usuario y la comunicación bidireccional inalámbrica aportaron una gran robustez y flexibilidad al sistema.
